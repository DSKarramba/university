\documentclass[pscyr,titlepage]{hedreport}
\usepackage[russian]{babel}
\usepackage{hedmaths}
\usepackage{graphicx}
\graphicspath{{images/}}

\usepackage{color}
\usepackage[colorlinks,linkcolor=black,urlcolor=black]{hyperref}
\renewcommand{\UrlFont}{\rm\small}

\faculty{Факультет электроники и вычислительной техники}
\department{физики}
\subject{дисциплине\\<<Методы и средства физического эксперимента>>}
\topic{Измерение сопротивлений косвенными методами}
\student[m]{студент группы Ф-469\\Чечеткин И. А.}
\teacher[m]{старший преподаватель\\Аршинов А. В.}

\begin{document}
  \maketitle
  \tableofcontents

  \pagebreak
  \renewcommand{\bibname}{Список литературы}

  \begin{thebibliography}{9} \addcontentsline{toc}{section}{Список литературы}
    \bibitem{1} Раннев,~Г.~Г. Методы и средства измерений: Учебник для вузов~/
      Г.~Г.~Раннев, А.~П.~Тарасенко~-- М.:~Издательский центр <<Академия>>,
      2004.~-- 336~с.
    \bibitem{2}
    \bibitem{3} 
    \bibitem{4} Измерение сопротивления постоянному току. Режим доступа:\\
      \url{http://www.sonel.ru/ru/biblio/article/resistance-directcurrent/}\\
      (дата обращения 20.11.2013).
    \bibitem{5} Измерение ультрамалых сопротивлений. Режим доступа:\\
      \url{http://radioradar.net/articles/technics_measurements/%
      measurements_ultra.html}\\
      (дата обращения 20.11.2013).
  \end{thebibliography}
\end{document}
