\documentclass[a4paper,14pt]{extarticle}
\usepackage[utf8]{inputenc}
\usepackage[T2A]{fontenc}
\usepackage[russian]{babel}
\usepackage[left=2.5cm,right=1cm,top=2cm,bottom=2cm]{geometry}

\usepackage{color}
\usepackage[colorlinks,linkcolor=black,citecolor=black]{hyperref}
\renewcommand{\rmdefault}{ftm}

\makeatletter
  \renewcommand{\@evenhead}{\vbox{\hbox to \textwidth {\hfil\thepage\hfil}}}
  \renewcommand{\@oddhead} {\vbox{\hbox to \textwidth {\hfil\thepage\hfil}}}
  \renewcommand{\@oddfoot} {\@empty}
  \renewcommand{\@evenfoot}{\@empty}
  
  \renewcommand{\maketitle}[5]{
    \def\@fletter{f}
    \def\@studentfemale{#3}
    \begin{titlepage}
      \begin{center}
        Министерство образования и науки Российской Федерации \\
        Федеральное государственное бюджетное образовательное учреждение \\
        высшего профессионального образования \\
        <<Волгоградский государственный технический университет>> \\
        Кафедра <<История, культура и социология>>
      \end{center}
      \vspace{9em}
      \begin{center}
        СЕМЕСТРОВАЯ РАБОТА \\
        ПО КУРСУ СОЦИОЛОГИИ \\
        ТЕМА №#1 \\
        \textbf{#2}
      \end{center}
      \vspace{3em}
      \begin{flushright}
        \begin{minipage}{.40\textwidth}
          \ifx\@studentfemale\@fletter
            \textbf{Выполнила:} \\
            студентка группы Ф-469 \\
          \else
            \textbf{Выполнил:} \\
            студент группы Ф-469 \\
          \fi
          #4 \\
          (№ зачётки #5) \\
          \vspace{1em} \\
          \textbf{Проверил:} \\
          Преподаватель кафедры ИКС \\
          доцент, кандидат соц.~наук \\
          Овчар~Н.~А.
        \end{minipage}
      \end{flushright}
      \vspace{\fill}
      \begin{center}
        Волгоград, \the\year
      \end{center}
    \end{titlepage}
    \global\let\@studentfemale\@empty
    \global\let\studentfemale\relax
    \global\let\@fletter\@empty
    \global\let\fletter\relax
  }
  
  \renewcommand{\thesection}      {\arabic{section}.}
  \renewcommand{\thesubsection}   {\thesection\arabic{subsection}.}
  \renewcommand{\thesubsubsection}{\thesubsection\arabic{subsubsection}.}
  
  \renewcommand{\l@section}      {\@dottedtocline{1}{0em}{1em}}
  \renewcommand{\l@subsection}   {\@dottedtocline{2}{1em}{2em}}
  \renewcommand{\l@subsubsection}{\@dottedtocline{3}{2em}{3em}}
\makeatother

\usepackage{setspace}

\usepackage{titlesec}
\titleformat{\section}{\bf\normalsize\center}
  {Глава \thesection\hspace*{-5pt}} {1em}{\vspace{-.5em}}{}
\titleformat{\subsection}{\bf\normalsize\center}
  {\thesubsection\hspace*{-10pt}}   {1em}{\vspace{-.5em}}{}
\titleformat{\subsubsection}{\bf\normalsize\center}
  {\thesubsubsection\hspace*{-10pt}}{1em}{\vspace{-.5em}}{}


\begin{document}
  \maketitle{26}{ЭКОНОМИЧЕСКИЕ КРИЗИСЫ И СОЦИАЛЬНАЯ МОБИЛЬНОСТЬ}
    {m}{Чечеткин~И.~А.}{20101026}
  \onehalfspacing
  \setcounter{page}{2}
  \tableofcontents

  \newpage

  \section*{Введение}
  \addcontentsline{toc}{section}{Введение}
  На сегодняшний день современный мировой финансовый экономический кризис
  является широко распространенной проблемой. Ведь он коснулся каждого человека
  не только в нашей стране, но и затронул весь мир в целом. Проблема
  экономического кризиса является достаточно актуальной в настоящий момент, так
  как он повлиял на жизнь населения и в какой-то степени даже изменил социальный
  уровень некоторых граждан, не только нашей необъятной родины, но и за ее
  пределами.

  Финансовый кризис связан с теми или иными различными элементами и механизмами
  повседневности нашего общества. И в первую очередь он как мне кажется, связан
  с социальной мобильностью, так как в свою очередь непосредственно оказывает
  существенное на нее влияние.

  П.~А.~Сорокин определял социальную мобильность как <<\ldotsлюбой переход
  индивида или социального объекта (ценности), т. е. всего того, что создано или
  модифицировано человеческой деятельностью, из одной социальной позиции в
  другую>>\footnote{Сорокин,~П.~А. Социальная стратификация и социальная
  мобильность~/ П.~А.~Сорокин~// Человек. Цивилизация. Общество.~-- 1992.~--
  С.~373}.

  Объектом данного исследования являются жители России в общем и города
  Волгограда и Волгоградской области в частности. В качестве предмета
  исследования в работе будет рассматриваться социальная мобильность в условиях
  экономического кризиса.

  Эмпирическую основу для семестровой работы составляют данные из журнала
  общественные науки и современность, статьи и данные всероссийского центра
  изучения общественного мнения, а также данные социологических исследований,
  размещенные а официальных сайтах соответствующих российских исследовательских
  центров.

  Основные цели и задачи которые необходимо решить:
  \begin{itemize}
    \itemsep -.75ex
    \item определить сущность понятий социальной мобильности и экономического
      кризиса;
    \item определить с чего же начался современный мировой финансовый кризис;
    \item выяснить как кризис сказался на образе жизни россиян;
    \item определить какое социальное самочувствие жителей Волгограда во время
      кризиса.
  \end{itemize}

  Объектом данного исследования являются жители России в общем и города
  Волгограда в частности. В качестве предмета исследования в работе будет
  рассматриваться социальная мобильность в условиях экономического кризиса.

  Практическая значимость моей работе заключается в получении общего
  представления о поведении граждан моего города в условиях кризиса. Анализ
  имеющихся данных может послужить хорошим основанием для разработки грамотной
  стратегии в области социальной политики города. Правильная социальная политика
  может помочь в адаптации жителей к условиям социального кризиса.

  Семестровая работа состоит из введения, четырех параграфов, заключения, списка
  использованной литературы и приложения в котором объясняется и представляется
  социальное самочувствие жителей Волгограда и области, в приложении
  используются аналитические данные которые, представлены в виде таблиц и
  графиков.

  \section{Сущность определений социальной мобильности и экономического кризиса}

  Из всех многочисленных формулировок этого понятия, наиболее близкая и
  отражающая суть следующая:
  \begin{quote}
    Кризис~-- состояние, при котором существующие средства достижения целей
    становятся неадекватными, в результате чего возникают непредсказуемые
    ситуации и проблемы.\footnote{Словарь по общественным наукам.
    Глоссарий.ru [Электронный ресурс].~-- Режим доступа: \url{www.glossary.ru}}
  \end{quote}
  Если говорить простыми словами с примерами, то это тяжелая <<болезнь>>
  организма экономики, к которой привел образ жизни современной экономики.
  Вообще, любой системный кризис не лечится имеющейся стандартной <<таблеткой>>,
  эффективной таблетки просто нет. Причем от <<приемов>> стандартной таблетки,
  ситуация только усугубляется. Например, при приеме <<от головы>> голова не
  проходит, а вот <<ноги>> перестают ходить. Нынешним имеющимся инструментарием
  его не излечить, ибо он же этот кризис системно и породил. Нельзя тушить огонь
  бензином, а причину ликвидировать следствием.
  
  Финансовый кризис~-- это реальность сегодняшнего дня, это высокие цены на
  продовольствие и недвижимость. Это рост увольнений в коммерческих и
  государственных структурах. Это неуверенность в завтрашнем дне каждого
  человека. Это изменение реалий окружающего мира, новые правила игры, условия
  и тенденции.
  
  В системе стратификации индивиды или группы могут перемещаться с одного
  уровня (слоя) на другой. Этот процесс называется социальной мобильностью.
  Социальное неравенство предполагает различия в распределении благ и
  ответственности, а социальная стратификация~-- структурированную систему
  неравенства, социальная мобильность проявляется в движении индивидов или
  групп от одного социального статуса к другому. 

  \section{Финансовый кризис в России}
  
  \subsection{Начало финансового кризиса}
  
  Кризис начался с паники на западных фондовых рынках и снижения цены на нефть.
  Первое привело к оттоку западного капитала с российского рынка и снижению
  стоимости акций, второй~-- к снижению стоимости акций напрямую. В итоге
  дешевых денег сразу стало не хватать. Если учесть, что снижение стоимости
  акций, оформленных в качестве залога по кредитам перед западными банками,
  повлекло за собой ряд платежных требований, то денег стало не хватать много.
  Это и есть тот самый кризис ликвидности~-- банковских активов просто
  недостаточно для эффективной работы. Итог~-- сворачивание программ
  кредитования для населения и бизнеса, повышение процентных ставок и другие
  меры по удержанию активов. Сокращения, безработица в финансовом секторе,
  непогашенные кредитные взносы\ldots
    
  Кризис ликвидности автоматически наносит удар по промышленности и особенно по
  сельскому хозяйству, которое без кредитов попросту не может существовать.
  Отсутствие кредита вкупе со снижением спроса на российском и мировом рынке
  влечет за собой общеэкономический кризис. Приметы этого кризиса уже
  известны~-- сокращения производства, неполная рабочая неделя, увольнения,
  опять же невыплаченные потребительские и ипотечные кредиты и~т.~д., и~т.~п.
  
  Уже сейчас спрос на некоторые виды продукции упал в 13--15 раз по сравнению
  прошлым годом. Неплохо чувствуют себя лишь те предприятия, которые не подсели
  на кредитную иглу: многие конкуренты выбиты из седла, а на бирже труда масса
  желающих получить работу. Среди них есть и ценнейшие кадры, заполучить которые
  еще год назад было просто невозможно ни за какие деньги. Проблема заключается
  в том, что ресурс таких предприятий также ограничен и снижение
  платежеспособного спроса ударит и по ним.

  \subsection{Образ жизни россиян во время кризиса}
  
  Финансовый кризис все сильнее сказывается на положении простых россиян.
  Сокращаются рабочие места, зарплаты заморожены, если не сокращены. Падение
  доходов граждан сопровождается удорожанием стоимости жизни: с начала 2009~года
  значительно выросли коммунальные платежи, стремительно увеличиваются цены на
  продукты питания. Согласно соцопросам, половина россиян констатирует падение
  уровня жизни за последние 3--4 месяца и заявляют о том, что кризис изменил их
  привычки в сфере потребления. Задаваясь вполне логичным, в такой ситуации,
  вопросом о том, как не допустить дальнейшего снижения уровня жизни, граждане
  пересматривают структуру своих расходов и уже осенью начали экономить на еде,
  отдавая предпочтение более дешевым продуктам. Как следствие: средний чек в
  супермаркетах уменьшился на треть, продукты премиум-класса стали менее
  востребованными, а одним из самых востребованных продуктов жителей регионов
  стали макароны, произведенные из низкосортной пшеницы. Показательно, что
  большинство наших соотечественников не считают, что в ближайшее время их
  экономические <<самочувствие>> улучшится.
  
  Как показал опрос <<Левада-центра>>, проведенный в конце января 2009~года в
  46-ти регионах РФ, семь из десяти россиян уже перешли на приобретение более
  дешевых продуктов питания. Лишь 27\% респондентов заявили, что за последние
  2--3 месяца они не стали отказывать себе в покупках привычных товаров.
  Социологи при этом отмечают, что для большинства россиян это отнюдь не отказ
  от излишеств и чрезмерностей, а урезание и так весьма скромного потребления.
  
  Опрос еще раз доказал, что в период кризисных явлений в экономике самыми
  уязвимыми оказались россияне с низким уровнем доходов. 43 процентов
  респондентов, определивших уровень своих доходов как низкий, признались, что в
  последнее время стали отказываться от <<привычных>> продуктов питания и
  предметов первой необходимости и покупать более дешевые.
  
  Жители крупных городов, а также обеспеченные россияне в меньшей степени
  пересматривали свой образ жизни. Лишь 13\% респондентов с высоким уровнем
  доходов серьезно изменили свои <<потребительские привычки>> в сторону перехода
  на более дешевые товары. Среди жителей больших городов (более 500~тыс.
  жителей) таких оказалось 20\%.
  
  Данные соцопроса подтверждаются и наблюдениями представителей торговых сетей,
  которые отмечают, что большим спросом стали пользоваться товары эконом-класса
  и низкого ценового сегмента. Теперь, в первую очередь для покупателя имеет
  значение цена, констатируют ритейлеры. Изменение структуры потребительской
  корзины россиян вынуждает торговые сети пересмотреть свою маркетинговую и
  рекламную политику~-- кризис толкает их отказаться от <<раскрученных>> брендов
  в пользу <<безымянных>>, но качественных и недорогих продуктов. Отечественный
  товаропроизводитель спешит воспользоваться сменой потребительских предпочтений
  россиян, надеясь увеличить объем и ассортимент выпускаемых товаров. Другое
  дело, что на продовольственном рынке РФ к настоящему моменту доля импорта
  составляет порядка 40\%. Российские производители мяса честно признаются, что
  не в состоянии сейчас полностью удовлетворить потребностей граждан в говядине
  и свинине. Поэтому совершенно очевидно, что мясная импортная продукция из-за
  роста курса доллара и евро будет дорожать сильнее, чем, например, мясо птицы
  (на рынке курятины приблизительно 80\% занимает отечественные производители)
  или молоко, а также хлеб.
  
  По данным Росстата, в январе 2009~года потребительские цены выросли на 2,4\%
  по сравнению с декабрем и на 13,4\% относительно уровня годичной давности.
  Аналитики прогнозируют, что новая волна подорожания начнется именно с
  продовольствия. За весь 2009 год цены выросли в среднем на 20\%, так как
  практически все отечественные производители в этом сегменте используют
  импортные ингредиенты, подорожавшие из-за девальвации.
  
  Россияне стали экономить на еде не только при посещении магазинов. По данным
  социологов, половина работающих граждан нашей страны с конца осени стали
  экономить на обедах, а каждый седьмой из респондентов признался, что и вовсе
  от них отказался.
  
  Граждане стали реже посещать рестораны и кафе. Владельцы ресторанов и кафе уже
  в конце прошлого года стали сокращать обслуживающих персонал /в среднем по
  Москве от 10 до 30 процентов.
  
  Кто не попал пока под сокращение~-- рискует получить уведомление о снижении
  зарплаты. Между тем, комментируя ситуацию с безработицей в столице, московские
  власти заявили, что, в первую очередь, под сокращения попадут иногородние
  работники, в частности, жители Московской области. Это заявление получило, как
  говорится, наглядное подтверждение. На московских улицах значительно
  сократилось число машин с немосковскими номерами: традиционных для мегаполиса
  дорожных пробок в часы пик с начала года практически не наблюдается.
  
  Кризис заставляет россиян экономить повсеместно. Граждане стали отказывать от
  поездок за рубеж, отложили покупку нового авто, решили повременить с ремонтом,
  а также сократили траты на обучение. Более пристальное внимание к проблеме
  сокращения россиянами расходов на еду объясняется отчасти тем, что до
  последнего времени наши соотечественники не спешили отказывать себе в питании,
  во многом оставаясь заложниками комплекса, приобретенного еще в советские
  времена, когда многие продукты питания были в дефиците, а пищевой рацион был
  весьма скудным.
  
  \section{Социальное самочувствие жителей Волгограда}

  За последнее десятилетие произошли существенные изменения в социальной
  структуре общества. Изменились объем и структура занятости населения,
  углубилось неравенство~-- экономическое, социальное и политическое,
  неравенство между регионами. Еще одной заметной тенденцией является
  маргинализация значительной части населения. Эти изменения привели к росту
  социальной напряженности и протестных настроений в обществе. Пик социальной
  напряженности пришелся на середину 90-х.
    
  Для разработки концепции сбалансированного развития общества необходимо иметь
  представление не только о статичном состоянии социальной системы, но и о
  динамике ее состояния.
  
  В 2009 году был проведен телефонный соцопрос волгоградцев, который показал,
  что уровень обеспокоенности граждан кризисом понемногу снижается. Как сообщили
  ИА <<Высота 102>> в агентстве маркетинговых коммуникаций <<Viewpoint>>, в
  исследовании, проводившемся в конце апреля текущего года, приняли участие 400
  волгоградцев. Согласно данным опроса, к концу второго весеннего месяца
  удовлетворенность жизнью ощущали 47\% опрошенных жителей областного центра.
  Тогда как еще месяц назад таковых было на 9\% меньше. При всем этом,
  количество респондентов, отметивших ухудшение своего материального положения
  увеличилось почти на 5\% и составило 30\% от общего числа участников опроса.
  
  Количество волгоградцев, считающих свое материальное положение <<очень
  плохим>>, с марта по апрель не изменилось (7\%). А вот граждан с <<хорошим>>
  достатком стало больше в два раза (18\%). Вдвойне стало больше и
  <<счастливчиков>>, считающих, что в денежном вопросе у них все <<очень
  хорошо>>: если в марте таковых не было совсем, то через месяц уже набрался
  целый 1\%.
  
  На уровне среднего оценили свое материальное благополучие 45\% респондентов.
  28\% опрошенных пока не видят свет в конце тоннеля и оценивают свое финансовое
  положение как <<плохое>>.
  
  За первые четыре месяца 2009 года, если верить исследованиям социологов, самым
  сложным для волгоградцев оказался февраль. Но уже на рубеже марта-апреля
  появились первые, пусть и не значительные, признаки улучшения благосостояния
  граждан.
  
  Как можно заметить, в обыденности, гражданам легче приспособиться к условиям
  кризиса, чем изменить свое социальное положение. Особенно это относится к
  семьям, проблемы с определением ребенка в детский садик, школу, смена
  окружения является не благоприятной перспективой.\cite{v102}
  
  \subsection{Инфраструктура организации среды обитания}  

  Определению качества жизни населения Волгограда было посвящено проведенное
  зимой--весной 2002 года прикладное социологическое исследование\cite{vstu}.
  Одним из ключевых понятий исследования была <<инфраструктура организации
  среды обитания>>. Она включает в себя в себя совокупность материально-бытовых
  и ситуационных элементов. Выделялось три уровня инфраструктуры организации
  среды обитания: микроуровень (квартира, дом, двор, улица), макроуровень
  (район, город) и метауровень (область, страна). В соответствии с целями и
  задачами исследования анализ восприятия жителями города
  социально-экономической ситуации проводился на макро- и метауровне.

  Волгоградская область занимает 58~место в России по уровню
  социально-экономического развития и включена в группу регионов с <<уровнем
  развития ниже среднего>>. Наиболее <<высокое>> место Волгоградская область
  занимает по показателю <<Доля населения с доходами ниже прожиточного
  минимума>>~-- 13 место.
  
  Отвечая на вопрос интервью об условиях жизни людей в Волгограде, большинство
  опрошенных отмечают, что условия жизни людей в городе либо такие же (33\%),
  либо хуже, чем в других регионах (30\%). Около 14\% респондентов считают, что
  условия жизни в городе лучше, чем в других регионах. Неоднородность мнения
  респондентов вполне естественна, но, при этом, позволяет строить гипотезы
  относительно мотивировки их мнения.
  
  Во-первых, социальная структура Волгограда в значительной степени
  дифференцирована: довольно большой процент жителей города с доходами ниже
  прожиточного минимума, неоднородная поселенческая структура в различных
  районах города.
  
  Во-вторых, при ответе на вопросы, связанные с оценкой каких-либо общественных
  явлений на более высоком уровне (регион, страна), респонденты склонны
  проецировать свои впечатления, получаемые из реальной жизни (на бытовом уровне).

  В-третьих, впечатление об условиях жизни в других регионах страны во многом
  детерминированы СМИ. Размерность информационного пространства СМИ существенно
  ниже размерности информационного пространства непосредственно наблюдаемой
  действительности. Кроме того, СМИ подают уже каким-либо образом обработанные
  и упорядоченные сведения, и формирует определенные стереотипы восприятия.
  
  Все эти гипотезы подтверждаются полученными данными. Основными причинами
  пессимизма и оптимизма высказываемого респондентами являются
  социально-экономические условия, сложившиеся в городе. При этом значимым
  является не тот факт, что где-то действительно лучше (хуже), а то, что здесь
  плохо (хорошо). Тот факт, что социально-экономические условия жизни в
  Волгограде обуславливают противоположные реакции респондентов, говорит о
  жестком расслоении жителей на социальные группы с различной степенью
  удовлетворенности потребностей.
  
  Еще одной причиной того, что Волгоград в оценках горожан занимает место ниже
  других регионов России по условиям жизни, является возможность
  самостоятельно, без посредничества СМИ, сравнивать условия жизни. Характерно, 
  что среди ответов респондентов на вопрос о причинах, по которым они считают,
  что в Волгограде условия жизни лучше, чем в других регионах, такая причина не
  упоминается.
  
  \subsection{Количественная и качественная оценки восприятия экономической
    ситуации страны, города, области}
    
  С целью количественной оценки восприятия жителями Волгограда сложившейся
  социальной и экономической ситуации респондентам предлагалось оценить
  социальную и экономическую ситуацию в России, области, городе и районе по
  девятибалльной шкале (1~балл~-- самая низкая (отрицательная) оценка,
  9~баллов~-- самая высокая (положительная) оценка).
  
  Социальная ситуация оценивается респондентами одинаково на всех уровнях;
  средние оценки респондентов лежат в середине оценочной шкалы (4~балла),
  разброс оценок респондентов не велик.
  
  Экономическая ситуация в городе и районе, по мнению опрошенных, не хуже, чем
  экономическая ситуация в России. При оценке экономической ситуации в
  Волгограде респонденты демонстрируют большее единство мнений, чем, например,
  при оценке ситуации в области~-- в первом случае 80\% ответов заключено в
  интервале от <<2>> до <<6>> баллов, во втором~-- от <<1>> до <<6>> баллов.
  
  На качественном уровне такие результаты могут свидетельствовать о том, что
  социальная и экономическая ситуация воспринимается респондентами как
  стабильная: нет пиковых, отрицательных или положительных оценок; ситуация
  однородна. Распределение мнений респондентов унимодальное (то есть
  распределение, характеризующееся одним пиком оценок). Такое распределение
  оценок близко к нормальному распределению (кривая Гаусса).
  
  Нормальное распределение значений характерно для переменных, на которые
  оказывают влияние большое количество не связанных между собой случайных
  факторов. Кроме того, это говорит об отсутствии общего знаменателя в
  отношении социальной ситуации: каждый респондент определяет это понятие
  по-своему. Исключая субъективный фактор в определении социальной ситуации,
  можно сказать, что наблюдаемая структура оценок характерна для стабильной
  модели общественного развития.
  
  \subsection{Факторы, влияющие на социальное самочувствие граждан}
  
  Проведенное исследование позволило выявить и классифицировать факторы,
  оказывающие наибольшее влияние на социальное самочувствие респондентов.
  Удалось выделить две группы факторов: внешние и внутренне факторы.
  
  Внешние факторы~-- это факторы, не зависящие от личности респондента. К ним
  относятся социально-экономические (уровень инфляции, обеспеченность
  медицинскими услугами и~т.~д.), экологические (качество воды и воздуха,
  загрязненность территорий промышленными отходами и~т.~д.) и климатические
  факторы (уровень влажности, годовой перепад температур и~т.~д.). Наиболее
  значимыми из них являются социально-экономические факторы.
  
  К внутренним факторам относятся характеристики респондента, которые
  опосредованно влияют на его ответы. Такими факторами являются гендерные
  отличия респондента (пол и возраст респондента), социальный статус
  респондента (имущественное положение, потребительская активность,
  общественное положение, род занятий и~т.~д.), психологические факторы (не
  осознаваемые респондентом детерминанты его поведения) и место жительства
  респондентов. Как показал анализ, влияние статуса респондента более выражено
  по сравнению с остальными факторами.
  
  В результате взаимодействия двух данных групп факторов возникают различия во
  взглядах высказываемых респондентами. Это позволяет структурировать
  восприятие социально-экономической ситуации различных групп респондентов и
  выявлять наиболее общественно значимые проблемы и методы их решения.~\cite{v-v}

  \section*{Заключение}
  \addcontentsline{toc}{section}{Заключение}

  Каждый человек перемещается в социальном пространстве, в обществе, в котором
  он живет. Иногда эти перемещения легко ощущаются и идентифицируются,
  например, при переезде индивида из одного места в другое, переход из одной
  религии в другую, изменение семейного положения. Это меняет позицию индивида
  в обществе и говорит о его перемещении в социальном пространстве. Однако
  существуют такие перемещения индивида, которые трудно определить не только
  окружающим его людям, но и ему самому. Например, сложно определить изменение
  положения индивида в связи с ростом престижа, увеличением или уменьшением
  возможностей использования власти, изменением дохода. 
  
  В своей работе я рассмотрел многие факторы, которые являются причиной такого
  перемещения, одним из них в глобальном масштабе является мировой финансовый
  экономический кризис. Вследствие которого изменяется доход населения, уровень
  и качество жизни людей.
  
  В ходе исследования архивов социологических исследований за различные годы, я
  заметил стабильную стратегию россиян. Было замечено, что соотечественники
  стараются адаптироваться к экономическим кризисам за счет экономии (что
  показывает снижение посещаемости развлекательных заведений, заведений общего
  питания, а также активное потребление товаров эконом класса) и такой
  стратегией пользуются более 60\% жителей страны. И лишь менее 20 процентов
  населения России стараются изменить свой образ жизни, например пройти
  переквалификацию, уехать в другой город для получения боле респектабельного
  места работы и~т.~д.~\cite{vcpor}
  
  Адаптация жителей страны и города не должна зависеть только от самих граждан,
  но и от грамотной социальной политики государства.

  \newpage

  \renewcommand{\bibname}{Список литературы}
  \begin{thebibliography}{9}
    \addcontentsline{toc}{section}{Список литературы}
    \bibitem{v-v} Авраамова,~Е.~М. Социальная мобильность в условиях российского
      кризиса~/ Е.~М.~Авраамова~// Общественные науки и современность.~--5
      1999.~-- №~3.~-- С.~5--12.
    \bibitem{vcpor} Сидят в кустах и ждут награды [Электронный ресурс]~//
      Всероссийский центр изучения общественного мнения~-- Режим доступа:
      \url{http://wciom.ru/index.php?id=266&uid=11798}
    \bibitem{v102} У волгоградцев вырабатывается иммунитет к кризису
      [Электронный ресурс]~// Информационное Агентство <<Высота 102.0>>~--
      Режим доступа: \url{http://v102.ru/analytic/11333.html}
    \bibitem{vstu} Шмельков,~А.~В. Социальное самочувствие жителей Волгограда~/
      А.~В.~Шмельков, Н.~В.~Дулина~// Социокультурные исследования:
      Межвузовский сборник научных трудов.~-- Волгоград: Волгоградский
      государственный технический университет, 2003.~-- Вып.~7~-- С.~97--99.
  \end{thebibliography}

\end{document}
