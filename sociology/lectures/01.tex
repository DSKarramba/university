\chapter{Социология как наука об обществе}
\section{Предпосылки возникновения социологии}

  Впервые термин <<социология>> (\emph{societas}~-- общество,
  \( \lambda o\gamma o\varsigma \)~-- слово, понятие) используется в работе
  Огюста~Конта в 1832~г. в одной из лекций <<Курса позитивной философии>>.
  Изначально, социология строилась как одна из естественных наук.

  Возникновению социологии способствует ряд причин:
  \begin{enumerate}
    \item политическая~-- смена политического строя во Франции; формирование
      парламентских республик в ряду западно-европейских стран; необходимость
      власти нести ответственность за свои действия перед населением;
      потребность в знаниях об обществе как объекте управления.
    \item Экономическая~-- свободная конкуренция в экономике вынуждает опираться
      на знание конкретных социальных механизмов; углубляется дифференциация в
      обществе; появление новых профессий, предполагающих как физический, так и
      умственный труд.
    \item Научно-теоретическая~-- во время появления социологии главенствуют
      естественные науки; новые науки стремятся преодолеть умозрительность,
      абстрактность и субъективность философского знания.
  \end{enumerate}

\section{Объект и предмет социологии}

  Объектом любой науки является то, на что направлено изучение, некая часть
  внешней реальности, выбранная для исследования данной наукой.

  Предметом науки являются те стороны, связи и отношения объекта, которые
  изучаются данной наукой.

  Для социологии объектом являются социальные отношения и связи, а также способы
  их реализации. Предметом социологии являются социальные закономерности и
  законы.

\section{Функции социологии}

  \begin{itemize}
    \item Теоретико-познавательная функция~-- накопление знаний об обществе,
    \item описательная функция~-- систематизация знаний об обществе,
    \item прогностическая~-- социология позволяет прогнозировать изменения в
      обществе,
    \item преобразовательная~-- на основе теоретических исследований
      разрабатываются практические рекомендации,
    \item мировоззренческая~-- социология формирует мировоззрение человека,
    \item идеологическая~-- социологические теории и концепции выражают интересы
      определенных социальных групп,
    \item инструментальная~-- социология вырабатывает подходы и средства анализа
      социальной реальности.
  \end{itemize}

\section{Структура социологического знания}

  Существует множество различных моделей социологического знания. Одной из
  общепризнанных является модель Роберта Мертона:
  \begin{center}
    \begin{tabular}{rl}
      верхний уровень: & обще-социологические теории; \\
      средний уровень: & теории среднего уровня; \\
       нижний уровень: & конкретные социологические исследования. \\
    \end{tabular}
  \end{center}

  Все уровни тесно переплетены между собой информационными каналами, по которым
  передаются результаты исследований.

  \bigskip

  Также социологию делят на макро- и микросоциологию.

  Макросоциология рассматривает большие социальные образования: коллективы,
  большие общности людей, нации, государства и их взаимоотношения.

  Микросоциология же обращает внимание на немногие социальные образования:
  единичные коллективы, взаимоотношения между отдельными людьми.

\section{Связь социологии с другими науками}

  Науки делятся на три типа:
  \begin{enumerate}
    \item естественные~-- физика, химия, биология, \ldots;
    \item общественные~-- политология, психология, экономика, \ldots;
    \item гуманитарные~-- философия, культурология, филология, \ldots.
  \end{enumerate}

  Гуманитарные науки оперируют нестрогими моделями, оценочными суждениями и
  качественными методами; общественные науки~-- формализированными моделями,
  математическим аппаратом и опираются на количественное знание. Гуманитарные
  науки не опираются в такой значительной степени, как общественные науки, на
  эмпирические методы. Общественные, напротив, оперируют суждениями,
  объективность которых можно проверить на практике. Социология принадлежит и к
  гуманитарным, и к общественным наукам. С естественными науками социологию
  роднит использование математическим аппаратом теории вероятностей.

  \begin{thebibliography}{9}
    \addcontentsline{toc}{section}{Список литературы}
    \itemsep -.2em
    \bibitem{1} \href{http://www.isras.ru/files/File/Socis/2012_7/Gorshkov.pdf}
      {Горшков,~М.~К. Общество~-- социология~-- власть: к вопросу о
      взаимодействии~/ М.~К.~Горшков~// Социс.~-- 2012.~-- №~7.~-- С.~23-29.}
    \bibitem{2} \href{http://www.isras.ru/files/File/Socis/2012_7/Osipov.pdf}
      {Осипов,~Г.~В. Проблемы включения социологии в систему научного управления
      российским обществом~/ Г.~В.~Осипов~// Социс.~-- 2012.~-- №~7.~--
      С.~5-17.}
    \bibitem{3} \href{http://ecsocman.hse.ru/data/580/722/1217/010.OSPANOV.pdf}
      {Оспанов,~С.~И. Об объекте и предмете социологии: логико-гносеологическое
      осмысление~/ С.~И.~Оспанов~// Социс.~-- 1998.~-- №~3.~-- С.~62-72.}
    \bibitem{4} \href{http://www.isras.ru/files/File/Socis/2005-7/podvoyski.pdf}
      {Подвойский,~Д.~Г. О предпосылках и истоках рождения социологической
      науки~/ Д.~Г.~Подвойский~// Социс.~-- 2005.~-- №~7.~-- С.~3-12.}
    \bibitem{5} \href{http://www.isras.ru/files/File/Socis/2013_5/Popov.pdf}
      {Попов,~Е.~А. Язык социальности~-- конструкт эпохи или науки?~/
      Е.~А.~Попов~// Социс.~-- 2013.~-- №~5.~-- С.~13-19.}
    \bibitem{6} \href{http://vk.cc/2ivPZk}{Рабочая книга социолога~/ отв.~ред.
      Г.~В.~Осипов.~-- М.:~КомКнига, 2006.~-- 480~с. [DOC]}
    \bibitem{7} \href{http://2008.isras.ru/files/File/Socis/2007-9/toschenko.pdf}
      {Тощенко,~Ж.~Т. Парадигмы, структура и уровни социологического
      анализа~/ Ж.~Т.~Тощенко~// Социс.~-- 2007.~-- №~9.~-- С.~5-16.}
  \end{thebibliography}
