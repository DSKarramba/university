\documentclass[a4paper, 14pt]{extarticle}
\usepackage[utf8]{inputenc}
\usepackage[paper=a4paper, top=1cm, right=1cm, bottom=1.5cm, left=2cm]{geometry}
\usepackage{setspace}
\onehalfspacing

\usepackage{graphicx}
\graphicspath{{plots/}, {images/}}

\parindent=1.25cm

\usepackage{titlesec}

\titleformat{\section}
    {\normalsize\bfseries}
    {\thesection}
    {1em}{}

\titleformat{\subsection}
    {\normalsize\bfseries}
    {\thesubsection}
    {1em}{}

% Настройка вертикальных и горизонтальных отступов
\titlespacing*{\chapter}{0pt}{-30pt}{8pt}
\titlespacing*{\section}{\parindent}{*4}{*4}
\titlespacing*{\subsection}{\parindent}{*4}{*4}

\usepackage[square, numbers, sort&compress]{natbib}
\makeatletter
\bibliographystyle{unsrt}
\renewcommand{\@biblabel}[1]{#1.} 
\makeatother


\newcommand{\maketitlepage}[6]{
    \begin{titlepage}
        \singlespacing
        \newpage
        \begin{center}
            Министерство образования и науки Российской Федерации \\
            Федеральное государственное бюджетное образовательное \\
            учреждение высшего профессионального образования \\
            <<Волгоградский государственный технический университет>> \\
            #1 \\
            Кафедра #2
        \end{center}


        \vspace{14em}

        \begin{center}
            \large Семестровая работа #6 по дисциплине
            \\ <<#3>>
        \end{center}

        \vspace{5em}

        \begin{flushright}
            \begin{minipage}{.35\textwidth}
                Выполнила:\\#4
                \vspace{1em}\\
                Проверил:\\#5
                \\
                \\ Оценка \underline{\ \ \ \ \ \ \ \ \ \ \ \ \ \ \ \ }
            \end{minipage}
        \end{flushright}

        \vspace{\fill}

        \begin{center}
            Волгоград, \the\year
        \end{center}

    \end{titlepage}
    \setcounter{page}{2}
}

\newcommand{\maketitlepagewithvariant}[7]{
    \begin{titlepage}
        \singlespacing
        \newpage

        \begin{center}
            Министерство образования и науки Российской Федерации \\
            Федеральное государственное бюджетное образовательное \\
            учреждение высшего профессионального образования \\
            <<Волгоградский государственный технический университет>> \\
            #1 \\
            Кафедра #2
        \end{center}


        \vspace{8em}

        \begin{center}
            \large Семестровая работа #6 по дисциплине
            \\ <<#3>>
        \end{center}

        \vspace{1em}
        \begin{center}
            Вариант №#7
        \end{center}
        \vspace{4em}

        \begin{flushright}
            \begin{minipage}{.35\textwidth}
                Выполнила:\\#4
                \vspace{1em}\\
                Проверил:\\#5
                \\
                \\ Оценка \underline{\ \ \ \ \ \ \ \ \ \ \ \ \ \ \ \ }
            \end{minipage}
        \end{flushright}

        \vspace{\fill}

        \begin{center}
            Волгоград, \the\year
        \end{center}

    \end{titlepage}
    \setcounter{page}{2}
}

\input{../.preambles/10-russian}
\input{../.preambles/20-math}
\input{../.preambles/30-physics}

\renewcommand{\labelenumi}{\asbuk{enumi})}

\begin{document}
\maketitlepagewithvariant{Факультет электроники и вычислительной техники}
{физики}{Физика атомов}{студент группы Ф-369\\Чечеткин~И.~А.}
{доцент Еремин~А.~В.}{\!\!}{16}
\newpage
%-------------------------------------------------------------------------------
\emph{ИОФ 6.229.}
Имеются три параллельные друг другу абсолютно черные плоскости.
Найти установившуюся температуру \( T_x \):
\vspace*{-.8em}
\begin{enumerate} \itemsep-.5em
    \item внешних плоскостей, если внутреннюю плоскость поддерживают при
    температуре \( T \);
    \item внутренней плоскости, если внешние плоскости поддерживают при
    температурах \( T \) и \( 2T \).
\end{enumerate}

\vspace*{2em}
\emph{Решение:}
\begin{enumerate}
    \item Излучение, падающее на плоскости, равно излучению, которое испускает
    центральная плоскость. По закону Стефана-Больцмана:
    \[
        \sigma T_x^4 + \sigma T_x^4 = \sigma T^4,
    \]
    откуда:
    \[
        T_x = \frac{T}{\sqrt[4]{2}}.
    \]
    \item Излучение, падающее на две стороны центральной плоскости, равно
    излучению, которое испускают боковые плоскости. По закону Стефана-Больцмана:
    \[
        \sigma T^4 + 16\sigma T^4 = 2\sigma T_x^4,
    \]
    откуда:
    \[
        T_x = \sqrt[4]{\frac{17}{2}}T.
    \]
\end{enumerate}
\vspace*{2em}        
\emph{Ответ:} а) \( T_x = T/\sqrt[4]{2} \), б) \( T_x = T\cdot\sqrt[4]{17/2} \). 
\newpage
%-------------------------------------------------------------------------------
\emph{ИАЯФ 1.62.}
Фотон с энергией \( \hbar\omega \) испытал столкновение с электроном, который
двигался ему навстречу. В результате столкновения направление движения фотона
изменилось на противоположное, а его энергия осталась прежней. Найти скорость
электрона до и после столкновения (\( v \) и \( v' \)).

\vspace*{2em}
\emph{Решение:}

Из закона сохранения энергии
\[
    \hbar\omega + T = \hbar\omega + T'
\]
следует, что кинетическая энергия и, следовательно, скорость электрона, а также
импульс фотона не изменились: \( T = T' \), \( v = v' \),
\( p_\emph{ф} = p_\emph{ф}' \).

Из закона сохранения импульса
\[
    p_\emph{ф} - p_e = p_e - p_\emph{ф}
\]
следует, что импульс фотона равен импульсу электрона:
\[
    p_\emph{ф} = p_e, \quad \frac{\hbar\omega}{c} = \frac{mv}{\sqrt{1 -
    \left(\frac{v}{c}\right)}}.
\]
Выражая из последнего соотношения скорость электрона \( v \), получим:
\[
    v = \frac{\frac{\hbar\omega}{c}}{\sqrt{\left(\frac{\hbar\omega}{c^2}\right)^2 + m^2}}
\]
Окончательно, обозначая \( \eps = \hbar\omega/(mc^2) \):
\[
    v = \frac{c\eps}{\sqrt{\eps^2 + 1}}.
\]

\vspace*{2em}
\emph{Ответ:}
\vspace*{-1.6em}
\[
    v = \frac{c\eps}{\sqrt{\eps^2 + 1}}, \qquad\text{где }\ 
    \eps = \frac{\hbar\omega}{mc^2}.
\]
\newpage
%-------------------------------------------------------------------------------
\emph{ИОФ 5.42.}
Протон с кинетической энергией \( T \) и прицельным параметром \( b \) рассеялся
на кулоновском поле неподвижного ядра атома золота. Найти импульс, переданный
данному ядру.

\vspace*{2em}
\emph{Решение:}

По теореме косинусов:
\begin{align*}
    & \Delta p^2 = 2p^2 + 2p^2\cos\theta, \\
    & \Delta p = p\sqrt{2(1 + \cos\theta)}, \\
    & \Delta p = 2p\sin\frac{\theta}{2}.
\end{align*}

По формуле Резерфорда для угла рассеяния:
\[
    \ctg\frac{\theta}{2} = \frac{mv^2b}{ke^2Z} = \frac{2Tb}{ke^2Z}.
\]

Воспользовавшись известным тригонометрическим тождеством
\[
    \sin\frac{\theta}{2} = \frac{1}{\sqrt{1 + \ctg^2\frac{\theta}{2}}}
\]
и соотношениями \( p = \sqrt{2mT} \), \( mv^2 = 2T \), получим:
\[
    \Delta p = \frac{2p}{\sqrt{1 + \ctg^2\frac{\theta}{2}}} = \sqrt{\frac{4p^2}
    {1 + \left(\frac{mv^2b}{ke^2Z}\right)^2}} = \sqrt{\frac{8mT}{1 + \left(
    \frac{2Tb}{ke^2Z}\right)^2}}.
\]

\vspace*{2em}
\emph{Ответ:}
\vspace*{-1.7em}
\[
    \Delta p = \sqrt{\frac{8mT}{1 + \left(\frac{2Tb}{ke^2Z}\right)^2}}.
\]
\newpage
%-------------------------------------------------------------------------------
\emph{ИАЯФ 2.47.}
Вычислить для мезоатома водорода (в котором вместо электрона движется мезон,
имеющий тот же заряд, но массу в 207 раз больше):
\vspace*{-2em}
\begin{enumerate} \itemsep-.5em
    \item расстояние между мезоном и ядром в основном состоянии;
    \item длину волны резонансной линии;
    \item энергии связей основных состояний мезоатомов водорода, ядра которых
    протон и дейтрон.
\end{enumerate}

\vspace*{2em}
\emph{Решение:}
\begin{enumerate}
    \item По правилу квантования боровских орбит:
    \[
        L_n = \hbar n = mvr,
    \]
    откуда \( v = \hbar n/(mr) \).
    
    Подставляем значение \( v \) во второй закон Ньютона
    \[
        m\frac{v^2}{r} = k\frac{e^2}{r^2}, \quad \frac{\hbar^2n^2}{mr} = ke^2.
    \]
    Найдем значение \( r \) в основном состоянии:
    \[
        r = \frac{\hbar^2 n^2}{mke^2} = \frac{\hbar^2}{207m_eke^2}.
    \]
    \item Резонансная линия -- головная линия серии Лаймана:
    \[
        \omega = R\left(\frac{1}{1} - \frac{1}{4}\right) = \frac{3}{4}R.
    \]
    Постоянная Ридберга для мезоатома водорода:
    \[
        R = \frac{k^2e^4}{2\hbar^3}m = \frac{207k^2e^4}{2\hbar^3}m_e.
    \]
    Длина волны этой линии:
    \[
        \lambda = \frac{2\pi c}{\omega} = \frac{8\pi c}{3R} = \frac{16\pi c\hbar}{207m_ek^2e^4}.
    \]
    
    \item Энергия связи основного состояния мезоатома водорода, ядром которого
    является протон:
    \[
        E_\emph{п} = \hbar\omega_\infty = \hbar R\left(\frac{1}{1} - \frac{1}
        {\infty}\right) = \hbar R = \left(\frac{ke^2}{\sqrt{2}\hbar}\right)^2\cdot207m_e.
    \]
    Если ядром будет дейтрон, то постоянная Ридберга в этом случае:
    \[
        R_\emph{д} = \frac{k^2e^4}{2\hbar^2}\cdot\frac{207m_e}{1 + 207\frac{m_e}
        {m_\emph{д}}},
    \]
    где \( m_\emph{д} \) -- масса дейтрона.
    
    Тогда энергия связи:
    \[
        E_\emph{д} = \hbar\omega_{\emph{д}_\infty} = \hbar R_\emph{д} =
        \left(\frac{ke^2}{\sqrt{2}\hbar}\right)^2\cdot\frac{207m_em_\emph{д}}
        {m_\emph{д} + 207m_e}.
    \]
\end{enumerate}

\vspace*{2em}
\emph{Ответ:}
\newpage
%-------------------------------------------------------------------------------
\emph{ИАЯФ 3.32.}
Оценить минимально возможную энергию электронов в атоме He и
соответствующее расстояние электронов от ядра.

\vspace*{2em}
\emph{Решение:}

\vspace*{2em}
\emph{Ответ:}
\newpage
%-------------------------------------------------------------------------------
\emph{ИАЯФ 5.29.}
Выписать электронные конфигурации и с помощью правила Хунда найти основной
терм атомов: а) C и N; б) S и Cl. Электронные конфигурации этих атомов
соответствуют застройке электронных оболочек в нормальном порядке.

\vspace*{2em}
\emph{Решение:}

\vspace*{2em}
\emph{Ответ:}
\newpage
%-------------------------------------------------------------------------------
\emph{ИОФ 5.192.}
При увеличении напряжения на рентгеновской трубке от \( U_1 = 10 \)~кВ до
\( U_2 = 20 \)~кВ интервал длин волн между \( K_\alpha \)-линией и коротковолновой
границей сплошного рентгеновского спектра увеличился в \( n = 3,0 \)~раза.
Определить порядковый номер элемента антикатода этой трубки, имея в виду, что
данный элемент является легким.

\vspace*{2em}
\emph{Решение:}

Длина волны коротковолновой границы сплошного рентгеновского спектра
определяется выражением
\[
    \lambda_\emph{кг} = \frac{2\pi c\hbar}{eU},
\]
где \( U \) -- напряжение на рентгеновской трубке.

Длина волны \( K_\alpha \)-линии:
\[
    \lambda_{K\alpha} = \frac{2\pi c}{\omega_{K\alpha}} = \frac{8\pi c}{3R(Z - 1)^2},
\]
где \( \omega_{K\alpha} = \frac{3}{4}R(Z - 1)^2 \) -- частота \( K_\alpha \)-линии.

Интервал между этими двумя линиями при увеличении напряжения увеличился в
\( n \) раз:
\vspace*{-1em}
\begin{align*}
    \lambda_{K\alpha} - \lambda_\emph{кг\,2} & = n(\lambda_{K\alpha} - 
    \lambda_\emph{кг\,1}); \\
    \frac{8\pi c}{3R(Z - 1)^2} - \frac{2\pi c\hbar}{eU_2} & = n\frac{8\pi c}
    {3R(Z - 1)^2} - n\frac{2\pi c\hbar}{eU_1}; \\
    \frac{4(n - 1)}{3R(Z - 1)^2} & = \frac{\hbar(nU_2 - U_1)}{eU_1U_2}; \\
    3R(Z - 1)^2 & = \frac{4eU_1U_2(n - 1)}{\hbar (nU_2 - U_1)}.
\end{align*}

Окончательно:
\[
    Z = 2\sqrt{\frac{eU_1(n - 1)}{3R\hbar\left(n - \frac{U_1}{U_2}\right)}} + 1 = 29.
\]

\vspace*{2em}
\emph{Ответ:}
\vspace*{-1.8em}
\[
    Z = 2\sqrt{\frac{eU_1(n - 1)}{3R\hbar\left(n - \frac{U_1}{U_2}\right)}} + 1 = 29.
\]
\end{document}
