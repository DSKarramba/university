\documentclass[a4paper, 14pt]{extarticle}
\usepackage[utf8]{inputenc}
\usepackage[paper=a4paper, top=1cm, right=1cm, bottom=1.5cm, left=2cm]{geometry}
\usepackage{setspace}
\onehalfspacing

\usepackage{graphicx}
\graphicspath{{plots/}, {images/}}

\parindent=1.25cm

\usepackage{titlesec}

\titleformat{\section}
    {\normalsize\bfseries}
    {\thesection}
    {1em}{}

\titleformat{\subsection}
    {\normalsize\bfseries}
    {\thesubsection}
    {1em}{}

% Настройка вертикальных и горизонтальных отступов
\titlespacing*{\chapter}{0pt}{-30pt}{8pt}
\titlespacing*{\section}{\parindent}{*4}{*4}
\titlespacing*{\subsection}{\parindent}{*4}{*4}

\usepackage[square, numbers, sort&compress]{natbib}
\makeatletter
\bibliographystyle{unsrt}
\renewcommand{\@biblabel}[1]{#1.} 
\makeatother


\newcommand{\maketitlepage}[6]{
    \begin{titlepage}
        \singlespacing
        \newpage
        \begin{center}
            Министерство образования и науки Российской Федерации \\
            Федеральное государственное бюджетное образовательное \\
            учреждение высшего профессионального образования \\
            <<Волгоградский государственный технический университет>> \\
            #1 \\
            Кафедра #2
        \end{center}


        \vspace{14em}

        \begin{center}
            \large Семестровая работа #6 по дисциплине
            \\ <<#3>>
        \end{center}

        \vspace{5em}

        \begin{flushright}
            \begin{minipage}{.35\textwidth}
                Выполнила:\\#4
                \vspace{1em}\\
                Проверил:\\#5
                \\
                \\ Оценка \underline{\ \ \ \ \ \ \ \ \ \ \ \ \ \ \ \ }
            \end{minipage}
        \end{flushright}

        \vspace{\fill}

        \begin{center}
            Волгоград, \the\year
        \end{center}

    \end{titlepage}
    \setcounter{page}{2}
}

\newcommand{\maketitlepagewithvariant}[7]{
    \begin{titlepage}
        \singlespacing
        \newpage

        \begin{center}
            Министерство образования и науки Российской Федерации \\
            Федеральное государственное бюджетное образовательное \\
            учреждение высшего профессионального образования \\
            <<Волгоградский государственный технический университет>> \\
            #1 \\
            Кафедра #2
        \end{center}


        \vspace{8em}

        \begin{center}
            \large Семестровая работа #6 по дисциплине
            \\ <<#3>>
        \end{center}

        \vspace{1em}
        \begin{center}
            Вариант №#7
        \end{center}
        \vspace{4em}

        \begin{flushright}
            \begin{minipage}{.35\textwidth}
                Выполнила:\\#4
                \vspace{1em}\\
                Проверил:\\#5
                \\
                \\ Оценка \underline{\ \ \ \ \ \ \ \ \ \ \ \ \ \ \ \ }
            \end{minipage}
        \end{flushright}

        \vspace{\fill}

        \begin{center}
            Волгоград, \the\year
        \end{center}

    \end{titlepage}
    \setcounter{page}{2}
}

\input{../.preambles/10-russian}
\input{../.preambles/20-math}
\input{../.preambles/30-physics}

\renewcommand{\labelenumi}{\asbuk{enumi})}

\begin{document}
\maketitlepage{Факультет электроники и вычислительной техники}{физики}
{Физика атомов}{студент группы Ф-369\\Чечеткин~И.~А.}
{доцент Еремин~А.~В.}{\!\!}
\newpage
\emph{ИОФ 6.229.}
Имеются три параллельные друг другу абсолютно черные плоскости.
Найти установившуюся температуру \( T_x \):
\begin{enumerate}
    \item внешних плоскостей, если внутреннюю плоскость поддерживают при
    температуре \( T \);
    \item внутренней плоскости, если внешние плоскости поддерживают при
    температурах \( T \) и \( 2T \).
\end{enumerate}

\vspace*{2em}
\emph{Решение:}

\vspace*{2em}        
\emph{Ответ:}
\newpage
%-------------------------------------------------------------------------------
\emph{ИАЯФ 1.62.}

\vspace*{2em}
\emph{Решение:}

\vspace*{2em}
\emph{Ответ:}
\newpage
%-------------------------------------------------------------------------------
\emph{ИОФ 5.42.}
Протон с кинетической энергией \( T \) и прицельным параметром \( b \) рассеялся
на кулоновском поле неподвижного ядра атома золота. Найти импульс, переданный
данному ядру.

\vspace*{2em}
\emph{Решение:}

\vspace*{2em}
\emph{Ответ:}
\newpage
%-------------------------------------------------------------------------------
\emph{ИАЯФ 2.47.}

\vspace*{2em}
\emph{Решение:}

\vspace*{2em}
\emph{Ответ:}
\newpage
%-------------------------------------------------------------------------------
\emph{ИАЯФ 3.32.}

\vspace*{2em}
\emph{Решение:}

\vspace*{2em}
\emph{Ответ:}
\newpage
%-------------------------------------------------------------------------------
\emph{ИАЯФ 5.29.}

\vspace*{2em}
\emph{Решение:}

\vspace*{2em}
\emph{Ответ:}
\newpage
%-------------------------------------------------------------------------------
\emph{ИОФ 5.192.}
При увеличении напряжения на рентгеновской трубке от \( U_1 = 10 \)~кВ до
\( U_2 = 20 \)~кВ интервал длин волн между \( K_\alpha \)-линией и коротковолновой
границей сплошного рентгеновского спектра увеличился в \( n = 3,0 \)~раза.
Определить порядковый номер элемента антикатода этой трубки, имея в виду, что
данный элемент является легким.

\vspace*{2em}
\emph{Решение:}

Длина волны коротковолновой границы сплошного рентгеновского спектра
определяется выражением
\[
    \lambda_\emph{кг} = \frac{2\pi c\hbar}{eU},
\]
где \( U \) -- напряжение на рентгеновской трубке.

Длина волны \( K_\alpha \)-линии:
\[
    \lambda_{K\alpha} = \frac{2\pi c}{\omega_{K\alpha}} = \frac{8\pi c}{3R(Z - 1)^2},
\]
где \( \omega_{K\alpha} = \frac{3}{4}R(Z - 1)^2 \) -- частота \( K_\alpha \)-линии.

Интервал между этими двумя линиями при увеличении напряжения увеличился в
\( n \) раз:
\begin{align*}
    \lambda_{K\alpha} - \lambda_\emph{кг\,2} & = n(\lambda_{K\alpha} - 
    \lambda_\emph{кг\,1}); \\
    \frac{8\pi c}{3R(Z - 1)^2} - \frac{2\pi c\hbar}{eU_2} & = n\frac{8\pi c}
    {3R(Z - 1)^2} - n\frac{2\pi c\hbar}{eU_1}; \\
    \frac{4(n - 1)}{3R(Z - 1)^2} & = \frac{\hbar(nU_2 - U_1)}{eU_1U_2}; \\
    3R(Z - 1)^2 & = \frac{4eU_1(n - 1)}{\hbar\left(n - \frac{U_1}{U_2}\right)}.
\end{align*}

Окончательно:
\[
    Z = 2\sqrt{\frac{eU_1(n - 1)}{3R\hbar\left(n - \frac{U_1}{U_2}\right)}} + 1 = 29.
\]

\emph{Ответ:}
\[
    Z = 2\sqrt{\frac{eU_1(n - 1)}{3R\hbar\left(n - \frac{U_1}{U_2}\right)}} + 1 = 29.
\]
\end{document}
