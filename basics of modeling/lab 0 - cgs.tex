\documentclass[14pt,a4paper]{extarticle}
\usepackage[utf8]{inputenc}
\usepackage[russian]{babel}
\usepackage[margin = 1.5cm]{geometry}
\usepackage{array}
%\usepackage{amsmath}
\pagestyle{empty}

\makeatletter
  \renewcommand\section{\@startsection{section}{1}{\z@}{3ex}{3ex}
  {\normalfont\normalsize\bfseries}}
  \renewcommand\subsection{\@startsection{subsection}{2}{\z@}{2ex}{2ex}
  {\normalfont\normalsize\bfseries}}
\makeatother
\renewcommand{\thesection}{\arabic{section}.}
\renewcommand{\thesubsection}{\thesection\arabic{subsection}.}

\usepackage[root]{hedmaths}
\usepackage{pscyr}

\begin{document}

  \begin{center}
    \large\bfseries
    Системы единиц измерения
    \bigskip
  \end{center}

  \section{Система СГС}

  Международной системой единиц принята система СИ, или система
  <<метр-килограмм-секунда-ампер>>. В физике же более удобной является система
  СГС, или <<сантиметр-грамм-секунда>>, из-за отсутствия дополнительной единицы
  для тока.

  В системе СГС закон Кулона и сила Лоренца приобретают более простой вид:
  \begin{table}[h!]
    \center
    \begin{tabular}{r>{---\ \ \(}l<{\)}}
      Закон Кулона & \vec{F} = \cfrac{q_1q_2}{r^2} \cdot \cfrac{\vec{r}}{r}
        \\[1em]
      Сила Лоренца & \vec{F} = q\vec{E} + \cfrac{q}{c}
        \left[ \vec{v}, \vec{B} \right] \\
    \end{tabular}
  \end{table}

  Также видно, что в системе СГС электрическое и магнитное поля имеют одну
  размерность.

  Фундаментальные константы в системе СГС:
  \begin{table}[h!]
    \center
    \begin{tabular}{|r|l|} \hline
      Масса электрона & \( m_e = 9,\!109 \cdot 10^{-28} \)~г \\
      Заряд электрона & \( e = 4,\!804 \cdot 10^{-10} \)~ед.~СГС \\
      Постоянная Дирака & \( \hbar = 1,\!055 \cdot 10^{-27} \)~эрг\(\cdot\)с \\
      Скорость света & \( c = 2,\!99792458 \cdot 10^{10} \)~см/с \\
      Постоянная тонкой структуры &
        \( \alpha = 1 / 137 = 7,\!295 \cdot 10^{-3} \) \\ \hline
    \end{tabular}
  \end{table}

  \section{Атомная и релятивистская системы единиц}
  \subsection{Атомная система единиц}

  Помимо СГС, в физике часто используют естественные системы единиц -- атомную и
  релятивистскую.

  В атомной системе единиц фундаментальные константы \( m_e \), \( e \) и
  \( \hbar \) принимаются за единицы: \( m_e = 1 \), \( e = 1 \),
  \( \hbar = 1 \).

  Тогда из модели Бора \( mv^2 / r = e^2 / r^2 \) и
  \( mvr = n\hbar \) для первой электронной орбиты (\( n = 1, v = v_0,
  r = a_0 \)) получаем:
  \[
    a_0 = \frac{\hbar^2}{m_e e^2} = 1, \quad
      v_0 = \frac{e^2}{\hbar} = 1.
  \]

  Таким образом, единицей длины в атомной системе единиц является радиус первой
  боровской орбиты, а единицей скорости~-- скорость электрона на этой орбите.

  Тогда единицей времени является величина \( t_0 = a_0 / v_0 \), а единицей
  энергии \( E_0 = mv^2 \).

  Значения единиц атомной системы в единицах СГС:
  \begin{table}[h!]
    \center
    \begin{tabular}{c}
      \( a_0 = 5,\!296 \cdot 10^{-9} \)~см, \\
      \( v_0 = 2,\!187 \cdot 10^8 \)~см/с, \\
      \( t_0 = 2,\!421 \cdot 10^{-17} \)~с, \\
      \( E_0 = 4,\!357 \cdot 10^{-11} \)~эрг. \\
    \end{tabular}
  \end{table}

  Переведем некоторые величины, например, гравитационную постоянную, ускорение
  свободного падения и значение заряда Дирака, из системы СГС в атомную систему
  единиц.

  Единицы СГС в атомных единицах:
  \begin{table}[h!]
    \center
    \begin{tabular}{c}
      1~см \( = 1 / 5,\!296 \cdot 10^{-9}~a_0 \), \\
      1~с \( = 1 / 2,\!421 \cdot 10^{-17}~t_0 \), \\
      1~г \( = 1 / 9,\!109 \cdot 10^{-28}~m_e \), \\
      1~ед.~заряда~СГС \( = 1 / 4,\!804 \cdot 10^{-10}~e \). \\
    \end{tabular}
  \end{table}

  Значение гравитационной постоянной в системе СГС:
  \[
    G_\textsc{сгс} = 6,\!674 \cdot 10^{-8} \ \ \frac{\text{см}^3}{\text{г}
      \cdot \text{с}^2}.
  \]

  Тогда значение гравитационной постоянной в атомной системе единиц:
  \[
    G_\textsc{асе} = G_\textsc{сгс} \cdot \frac{9,\!109 \cdot 10^{-28} \cdot
      \left( 2,\!421 \cdot 10^{-17} \right)^2}{\left( 5,\!296 \cdot 10^{-9}
      \right)^3}\ \frac{a_0^3}{m_e t_0^2} = 1,\!973 \cdot 10^{-45}\
      \frac{a_0^3}{m_e t_0^2}.
  \]

  Аналогичным образом можно посчитать значение ускорения свободного падения в
  атомной системе единиц (\( g_\textsc{сгс} = 981~\text{см}/\text{с}^2 \)):
  \[
    g_\textsc{асе} = 981 \cdot \frac{\left( 2,\!421\cdot 10^{-17} \right)^2}
      {5,\!296 \cdot 10^{-9}}\ \frac{a_0}{t_0^2} = 1,\!086 \cdot 10^{-22}\
      \frac{a_0}{t_0^2}.
  \]

  Значение заряда Дирака (\( q_{d_\text{СГС}} = \sqrt{\hbar \cdot c} =
  5,\!624 \cdot 10^{-9} \)~ед.~СГС):
  \[
    q_{d_\text{АСЕ}} = \frac{5,\!624 \cdot 10^{-9}}{4,\!804 \cdot 10^{-10}}\ e
      = 11,\!710\ e.
  \]

  \subsection{Релятивистская система единиц}

  В релятивистской системе единиц за единицу принимаются \( m_e \), \( e \) и
  скорость света \( c \).

  За единицу расстояния в релятивистской системе единиц принимается классический
  радиус электрона:
  \[
    r_e = \frac{e^2}{m_e c^2} = 1.
  \]

\end{document}
