\clearpage
\renewcommand{\refname}{СПИСОК ИСПОЛЬЗОВАННЫХ ИСТОЧНИКОВ}
\addcontentsline{toc}{section}{Список использованных источников}
\begin{thebibliography}{10}
\bibitem{Salvat} Salvat, F. Analytical Dirac-Hartree-Fock-Slater screening
function for atom $(Z=1-92)$ / F. Salvat, J.D. Martiner, R. Mayol, J.
Parellada / Physical Review, - Volume 36, Number 2, - July 15, 1987. -
467-474 p.
\bibitem{Landau} Ландау,~Л.~Д. Теоретическая физика. В 10 т. Т. 3. Квантовая механика. Нерелятивистская теория. / Л.~Д.~Ландау, Е.~М.~Лифшиц.~--- Москва~: Наука, 1989.~--- 768 с.
\bibitem{mono} Смоляр,~В.~А. Прохождение электронов через вещество~: монография / В.~А.~Смоляр, И.~И.~Маглеванный.~--- Волгоград~: ИУНЛ ВолгГТУ, 2011.~--- 160~с.

\end{thebibliography}
\newpage
\addcontentsline{toc}{section}{Приложение А. Экранирующая функция и
    потенциальная энергия атома в приближении Томаса--Ферми}
\section*{Приложение А. Экранирующая функция и потенциальная энергия атома в
приближении Томаса--Ферми}
\lstinputlisting[language=c,
      caption=Расчёт решения уравнения Томаса-Ферми]{code/tf.c}

  \vspace{1.5em}
  \lstinputlisting[language=python,caption=Построение графиков]{code/lab2.py}
\newpage
\addcontentsline{toc}{section}{Приложение Б. Расчёт сечений упругого
рассеяния в борновском приближении}
\section*{Приложение Б. Расчёт сечений упругого рассеяния в борновском
    приближении}
  \label{sec:code}
  \lstinputlisting[language=python]{code/lab3.py}
\end{document}
