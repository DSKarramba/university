\documentclass[10pt,a4paper]{extarticle}
\usepackage[utf8]{inputenc}
\usepackage[russian]{babel}
\usepackage[margin = 1.5cm]{geometry}
\usepackage{array}
%\usepackage{amsmath}
\pagestyle{empty}

\makeatletter
  \renewcommand\section{\@startsection{section}{1}{\z@}{3ex}{3ex}
  {\normalfont\large\bfseries}}
  \renewcommand\subsection{\@startsection{subsection}{2}{\z@}{2ex}{2ex}
  {\normalfont\normalsize\bfseries}}
\makeatother
\renewcommand{\thesection}{\arabic{section}.}
\renewcommand{\thesubsection}{\thesection\arabic{subsection}.}

\usepackage{multirow}

\newcolumntype{C}[1]{>{\centering\arraybackslash}m{#1\textwidth}}
\renewcommand{\arraystretch}{1.2}

\usepackage[root]{hedmaths}
\usepackage{pscyr}

\begin{document}

  \begin{table}[h!]
    \center
    \small
    \begin{tabular}{|C{.4}|C{.2}|C{.2}|} \hline
      \multirow{4}{*}{Лабораторная работа №~1}
      & Студент, группа & Слоква~В.~И., Чечеткин~И.~А., Ф-469 \\ \cline{2-3}
      & Дата выполнения & 13.02.2013 \\ \cline{2-3}
      & Подпись & \\ \cline{2-3}
      \multirow{3}{*}{\parbox{.35\textwidth}{\centering Системы единиц
        измерения}} & Дата отчёта & \\ \cline{2-3}
      & Оценка & \\ \cline{2-3}
      & Подпись & \\ \hline
    \end{tabular}
  \end{table}

  \section{Система СГС}

  Международной системой единиц принята система СИ, или система
  <<метр-килограмм-секунда-ампер>>. В физике же более удобной является система
  СГС, или <<сантиметр-грамм-секунда>>, из-за отсутствия дополнительной единицы
  для тока.

  В системе СГС закон Кулона и сила Лоренца приобретают более простой вид:
  \begin{table}[h!]
    \center
    \begin{tabular}{r>{---\ \ \(}l<{\)}}
      Закон Кулона & \vec{F} = \cfrac{q_1q_2}{r^2} \cdot \cfrac{\vec{r}}{r}
        \\[1em]
      Сила Лоренца & \vec{F} = q\vec{E} + \cfrac{q}{c}
        \left[ \vec{v}, \vec{B} \right] \\
    \end{tabular}
  \end{table}

  Также видно, что в системе СГС электрическое и магнитное поля имеют одну
  размерность.

  Фундаментальные константы в системе СГС:
  \begin{table}[h!]
    \center
    \begin{tabular}{|r|l|} \hline
      Масса электрона & \( m_e = 9,\!109 \cdot 10^{-28} \)~г \\
      Заряд электрона & \( e = 4,\!804 \cdot 10^{-10} \)~ед.~СГС \\
      Постоянная Дирака & \( \hbar = 1,\!055 \cdot 10^{-27} \)~эрг\(\cdot\)с \\
      Скорость света & \( c = 2,\!99792458 \cdot 10^{10} \)~см/с \\
      Постоянная тонкой структуры\footnotemark[1] &
        \( \alpha = 1 / 137 = 7,\!295 \cdot 10^{-3} \) \\ \hline
    \end{tabular}
  \end{table}
  \footnotetext[1]{ Имеет смысл комптоновской длины электрона, измеренной радиусами
    Бора: \( \alpha = \Lambda / a_0 \), или квадрата заряда электрона,
    измеренного квадратом заряда Дирака: \( \alpha = e^2 / q_d^2 \)}

  \section{Атомная и релятивистская системы единиц}
  \subsection{Атомная система единиц}

  Помимо СГС, в физике часто используют естественные системы единиц -- атомную и
  релятивистскую. Все законы и уравнения в ней выглядят так же, как и в
  системе~СГС, но значительно упрощаются за счет принятия за единицу
  фундаментальных констант.

  В атомной системе единиц фундаментальные константы \( m_e \), \( e \) и
  \( \hbar \) принимаются за единицы: \( m_e = 1 \), \( e = 1 \),
  \( \hbar = 1 \).

  Тогда из модели Бора
  \begin{equation}
    \left\{
      \begin{array}{l}
        m_e v^2 / r = e^2 / r^2, \\
        m_e vr = n\hbar;
      \end{array}
    \right.
    \label{Bohr_model}
  \end{equation}
  для первой электронной орбиты (\( n = 1, v = v_0, r = a_0 \)) получаем:
  \[
    a_0 = \frac{\hbar^2}{m_e e^2} = 1, \quad
      v_0 = \frac{e^2}{\hbar} = 1.
  \]

  Таким образом, единицей длины в атомной системе единиц является радиус первой
  боровской орбиты, а единицей скорости~-- скорость электрона на этой орбите.

  Тогда единицей времени является величина
  \[
    t_0 = a_0 / v_0,
  \]
  а единицей энергии
  \[
    E_0 = mv^2.
  \]

  Значения единиц атомной системы в единицах СГС:
  \begin{equation}
    \begin{array}{c}
      a_0 = 5,\!296 \cdot 10^{-9}~\text{см},\\
      v_0 = 2,\!187 \cdot 10^8~\text{см}/\text{с}, \\
      t_0 = 2,\!421 \cdot 10^{-17}~\text{с}, \\
      E_0 = 4,\!357 \cdot 10^{-11}~\text{эрг}.
    \end{array}
    \label{atom_cgs}
  \end{equation}

  Переведем некоторые величины, например, гравитационную постоянную, ускорение
  свободного падения и значение заряда Дирака, из системы СГС в атомную систему
  единиц.

  Выразим из \eqref{atom_cgs} единицы СГС в атомных единицах:
  \begin{gather*}
    1~\text{см} / a_0 = 1 / 5,\!296 \cdot 10^{-9}, \\
    1~\text{с} / t_0 = 1 / 2,\!421 \cdot 10^{-17}, \\
    1~\text{г} / m_e = 1 / 9,\!109 \cdot 10^{-28}, \\
    1~\text{ед.~заряда~СГС} / e = 1 / 4,\!804 \cdot 10^{-10}.
  \end{gather*}

  Значение гравитационной постоянной в системе СГС:
  \[
    G_\textsc{сгс} = 6,\!674 \cdot 10^{-8} \ \frac{\text{см}^3}{\text{г}
      \cdot \text{с}^2}.
  \]

  Тогда значение гравитационной постоянной в атомной системе единиц:
  \[
    G_\textsc{асе} = G_\textsc{сгс} \cdot \frac{9,\!109 \cdot 10^{-28} \cdot
      \left( 2,\!421 \cdot 10^{-17} \right)^2}{\left( 5,\!296 \cdot 10^{-9}
      \right)^3} = 1,\!973 \cdot 10^{-45}~\text{ат.~ед.~гравитации}.
  \]

  Аналогичным образом можно посчитать значение ускорения свободного падения в
  атомной системе единиц (\( g_\textsc{сгс} = 981~\text{см}/\text{с}^2 \)):
  \[
    g_\textsc{асе} = g_\textsc{сгс} \cdot \frac{\left( 2,\!421\cdot 10^{-17} \right)^2}
      {5,\!296 \cdot 10^{-9}} = 1,\!086 \cdot 10^{-22}~\text{ат.~ед.~ускорения}.
  \]

  Значение заряда Дирака (\( q_{d_\text{СГС}} = \sqrt{\hbar \cdot c} =
  5,\!624 \cdot 10^{-9} \)~ед.~СГС):
  \[
    q_{d_\text{АСЕ}} = \frac{5,\!624 \cdot 10^{-9}}{4,\!804 \cdot 10^{-10}}\ e
      = 11,\!710~\text{ат.~ед.~заряда}.
  \]

  \subsection{Релятивистская система единиц}

  В релятивистской системе единиц за единицу принимаются \( m_e \), \( e \) и
  скорость света \( c \).

  Тогда, поскольку постоянную тонкой структуры можно выразить как
  \( \alpha = e^2 / (\hbar c) \), то за элементарный заряд выражается как
  \( e = \sqrt{\alpha} = \sqrt{1 / 137} \), а за единицу заряда принимается
  Дираков заряд.
  
  За единицу длины принимается комптоновская длина электрона:
  \[
    \Lambda = \hbar / (m_e c) = 1.
  \]
  
  Единицей энергии является энергия покоя электрона:
  \[
    E = m_e c^2 = 1,
  \]
  а единицей импульса:
  \[
    p = m_e c = 1.
  \]
  Единица времени:
  \[
    t = \Lambda \cdot m_e / p = 1.
  \]

  Таким образом, значения единиц релятивистской системы в единицах СГС:
  \begin{equation}
    \begin{array}{c}
      \Lambda = 3,\!862 \cdot 10^{-11}~\text{см}, \\
      p_r = 2,\!731 \cdot 10^{-17}~\text{см} \cdot \text{кг}/\text{с}, \\
      t_r = 1,\!288 \cdot 10^{-21}~\text{с}, \\
      E_r = 8,\!187 \cdot 10^{-7}~\text{эрг}, \\
      q_d = 5,\!623 \cdot 10^{-9}~\text{ед.~СГС}.
    \end{array}
    \label{relat_cgs}
  \end{equation}

  Переведем гравитационную постоянную и ускорение свободного падения из системы
  СГС в релятивистскую систему единиц.

  Необходимые для перевода единицы СГС, выраженные в релятивистских единицах
  (из \eqref{relat_cgs}):
  \begin{gather*}
    1~\text{см} / \Lambda = 1 / 3,\!862 \cdot 10^{-11}, \\
    1~\text{с} / t_r = 1 / 1,\!288 \cdot 10^{-21}, \\
    1~\text{г} / m_e = 1 / 9,\!109 \cdot 10^{-28}.
  \end{gather*}

  Значение гравитационной постоянной в релятивистской системе единиц:
  \[
    G_\textsc{рсе} = G_\textsc{сгс} \cdot \frac{9,\!109 \cdot 10^{-28} \cdot
      \left( 1,\!288 \cdot 10^{-21} \right)^2}{\left( 3,\!862 \cdot 10^{-11}
      \right)^3} = 2,\!623 \cdot 10^{-38}~\text{релят.~ед.~гравитации}.
  \]

  Значение ускорения свободного падения в релятивистской системе единиц:
  \[
    g_\textsc{рсе} = g_\textsc{сгс} \cdot \frac{\left( 1,\!288\cdot 10^{-21}
      \right)^2}{3,\!862 \cdot 10^{-11}} = 4,\!296 \cdot
      10^{-32}~\text{релят.~ед.~ускорения}.
  \]

\section{Таблица значений физических величин}
  Приведем полученные выше численные значения в виде таблицы:
  \begin{table}[h!]
    \center
    \begin{tabular}{|*{5}{C{.18}|}} \hline
        \textbf{Величина} & \textbf{СГС} & \textbf{CИ} & \textbf{АСЕ} &
          \textbf{РСЕ} \\ \hline
        Масса электрона & \( 9,\!109 \cdot 10^{-28} \)~г &
          \( 9,\!109 \cdot 10^{-31} \)~кг & \( 1~[M] \) & \( 1~[M] \) \\ \hline
        Постоянная Дирака & \( 1,\!055 \cdot 10^{-27} \)~эрг\( \cdot \)с &
          \( 1,\!055 \cdot 10^{-34} \)~Дж\( \cdot \)с & \( 1~[\hbar] \) &
          \( 1~[\hbar] \) \\ \hline
        Заряд электрона & \( 4,\!804 \cdot 10^{-10} \)~ед.~СГС &
          \( 1,\!602 \cdot 10^{-19} \)~Кл & \( 1~[q] \) &
          \( 8,\!544 \cdot 10^{-2}~[q] \) \\ \hline
        Заряд Дирака & \( 5,\!624 \cdot 10^{-9} \)~ед.~СГС &
          \( 1,\!875 \cdot 10^{-18} \)~Кл & \( 11,\!71~[q] \) &
          \( 1~[q] \) \\ \hline
        Скорость света & \( 3 \cdot 10^{10} \)~см/с & \( 3 \cdot 10^{8} \)~м/с &
          \( 137~[L]/[T] \) & \( 1~[L]/[T] \) \\ \hline
        Боровский радиус & \( 5,\!292 \cdot 10^{-9} \)~см &
          \( 5,\!292 \cdot 10^{-11} \)~м & \( 1~[L] \) &
          \( 137~[L] \) \\ \hline
        Комптоновская длина электрона & \( 3,\!862 \cdot 10^{-11} \)~см &
          \( 3,\!862 \cdot 10^{-13} \)~м & \( 7,\!295 \cdot 10^{-3}~[L] \) &
          \( 1~[L] \) \\ \hline
        Единица времени АСЕ & \( 2,\!421 \cdot 10^{-17} \)~с &
          \( 2,\!421 \cdot 10^{-17} \)~с & \( 1~[T] \) &
          \( 1,\!878 \cdot 10^5~[T] \) \\ \hline
        Единица времени РСЕ & \( 1,\!288 \cdot 10^{-21} \)~с &
          \( 1,\!288 \cdot 10^{-21} \)~с & \( 5,\!325 \cdot 10^{-5}~[T] \) &
          \( 1~[T] \) \\ \hline
        Энергия электрона в атоме водорода & \( -2,\!178 \cdot 10^{-11} \)~эрг &
          \( -2,\!178 \cdot 10^{-18} \)~Дж & \( -0,\!5~[E] \) &
          \( -2,\!662 \cdot 10^{-5}~[E] \) \\ \hline
        Энергия покоя электрона & \( 8,\!187 \cdot 10^{-7} \)~эрг &
          \( 8,\!187 \cdot 10^{-14} \)~Дж & \( 1,\!878 \cdot 10^5~[E] \) &
          \( 1~[E] \) \\ \hline
        Гравитационная постоянная & \( 6,\!674 \cdot 10^{-8}~\frac{\text{см}^3}
          {\text{г} \cdot \text{с}^2} \) &
          \( 6,\!674 \cdot 10^{-11}~\frac{\text{м}^3}
          {\text{кг} \cdot \text{с}^2} \) &
          \( 1,\!973 \cdot 10^{-45}~\frac{[L]^3}{[M] \cdot [T]^2} \) &
          \( 2,\!623 \cdot 10^{-38}~\frac{[L]^3}{[M] \cdot [T]^2} \) \\ \hline
        Ускорение свободного падения & \( 981 \)~см/с\(^2\) &
          \( 9,\!81 \)~м/с\(^2\) &
          \( 1,\!086 \cdot 10^{-22}~\frac{[L]}{[T]^2} \) &
          \( 4,\!296 \cdot 10^{-32}~\frac{[L]}{[T]^2} \) \\ \hline
    \end{tabular}
  \end{table}
\end{document}
