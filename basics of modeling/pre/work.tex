\documentclass[a4paper,14pt]{extarticle} %размер бумаги устанавливаем А4, шрифт 12пунктов
\usepackage[T2A]{fontenc}
\usepackage[utf8]{inputenc}%включаем свою кодировку: koi8-r или utf8 в UNIX, cp1251 в

\usepackage[english,russian]{babel}%используем русский и английский языки с переносами
\usepackage{amssymb,amsfonts,amsmath,mathtext,cite,enumerate,float} %подключаем
\usepackage[colorlinks=True,urlcolor=blue]{hyperref}
\usepackage{caption2, array, multirow, tabulary,tabularx, dcolumn} %Чтобы поменять двоеточие в названии таблицы на тире
\usepackage[pdftex]{graphicx}

\usepackage{geometry, multicol} % Меняем поля страницы
\geometry{left=1.5cm}% левое поле
\geometry{right=1.5cm}% правое поле
\geometry{top=1cm}% верхнее поле
\geometry{bottom=2cm}% нижнее поле

\renewcommand{\thetable}{\arabic{table}}
%\renewcommand{\thesubsection}{}
\renewcommand{\baselinestretch}{1.5}

\renewcommand{\tabularxcolumn}[1]{m{#1}}
\newcolumntype{C}[1]{>{\setlength{\hsize}{#1\hsize}\centering\arraybackslash}X}
\newcolumntype{T}[1]{>{\setlength{\hsize}{#1\hsize}}X}
\newcolumntype{Q}[1]{>{\centering\setlength{\hsize}{#1\hsize}\arraybackslash\vspace{1mm}$\displaystyle}X<{$\vspace{1mm}}}
\newcolumntype{,}{D{,}{,}{14}}
\renewcommand{\captionlabeldelim}{\textendash}
\graphicspath{{images/}}

%\usepackage[usenames,dvipsnames]{color}
%\usepackage[numbered,framed]{mcode}

\tolerance=10000

\begin{document}

\begin{titlepage}
\vskip 6cm
\vskip 4cm
\begin{center}
Министерство образования и науки РФ\\
Федеральное государственное бюджетное образовательное\\
учреждение высшего профессионального образования\\
ВОЛГОГРАДСКИЙ ГОСУДАРСТВЕННЫЙ ТЕХНИЧЕСКИЙ УНИВЕРСИТЕТ\\
(ВолгГТУ)\\
Факультет «Электроника и вычислительная техника»\\
Кафедра «Физика»\\
\vskip 2cm
Курсовая работа по дисциплине \\
«Основы моделирования сложных физических систем»\\
Тема: «Моделирование атомного потенциала и вычисление атомного форм-фактора и
сечения рассеяния электронов для атомов %s и %s»\\
\end{center}
\vskip 2cm
\begin{flushright}
\parbox{6cm}{
Выполнил:\\
студент группы Ф-469\\
%s\\
Проверил:\\
Профессор, д. ф.-м. н.\\
Смоляр В.А.\\}
\vskip 2cm
\end{flushright}
\begin{center}
\def\hrf#1{\hbox to#1{\hrulefill}}
Оценка работы \hrf{4em} баллов
\end{center}
\begin{center}
\vskip 3cm
Волгоград, 2014
\end{center}

\newpage
\begin{center}
Министерство образования и науки Российской Федерации\\
Федеральное государственное бюджетное образовательное учреждение высшего профессионального образования\\
 «Волгоградский государственный технический университет»
\end{center}
\def\hrf#1{\hbox to#1{\hrulefill}}
Факультет \hrulefill\\
Кафедра \hrulefill
\begin{center}
ПОЯСНИТЕЛЬНАЯ ЗАПИСКА\\
к курсовой работе (проекту)
\end{center}
по дисциплине \hrulefill\\
на тему \hrulefill\\
 \hrf{15em}\hrulefill\\
 \hrf{15em}\hrulefill\\
Студент\hrulefill\\
\raisebox{15pt}{\hspace{7cm} {\small (фамилия, имя, отчество)} }\\
\vskip -1.5cm
Группа \hrf{12em}\\
Руководитель работы (проекта)  \hrf{10em} \hspace{0.5cm} \hrulefill\\
\raisebox{15pt}{\hspace{7cm} {\small (подпись и дата подписания)} \hspace{0.3cm}{\small (инициалы и фамилия)} }
\vskip -0.5cm
Члены комиссии:\\
\hrf{15em} \hspace{1cm} \hrf{15em}\\
\raisebox{15pt}{\hspace{1cm} {\small (подпись и дата подписания)} \hspace{4cm}{\small (инициалы и фамилия)} }\\
\hrf{15em} \hspace{1cm} \hrf{15em}\\
\raisebox{15pt}{\hspace{1cm} {\small (подпись и дата подписания)} \hspace{4cm}{\small (инициалы и фамилия)} }\\
\hrf{15em} \hspace{1cm} \hrf{15em}\\
\raisebox{15pt}{\hspace{1cm} {\small (подпись и дата подписания)} \hspace{4cm}{\small (инициалы и фамилия)} }\\
\vskip -1cm
Нормоконтролер \hrf{10em} \hspace{1cm} \hrulefill\\
\raisebox{15pt}{\hspace{5cm} {\small (подпись, дата)} \hspace{4cm}{\small (инициалы и фамилия)} }\\
\begin{center}
\vskip -0.9cm
Волгоград, 2014
\end{center}
\end{titlepage}

	\tableofcontents
