\documentclass[a4paper,14pt]{extarticle} %размер бумаги устанавливаем А4, шрифт 12пунктов
\usepackage[T2A]{fontenc}
\usepackage[utf8]{inputenc}%включаем свою кодировку: koi8-r или utf8 в UNIX, cp1251 в 

\usepackage[english,russian]{babel}%используем русский и английский языки с переносами
\usepackage{amssymb,amsfonts,amsmath,mathtext,cite,enumerate,float} %подключаем 
\usepackage[colorlinks=True,urlcolor=blue]{hyperref}
\usepackage{caption2, array, multirow, tabulary,tabularx, dcolumn} %Чтобы поменять двоеточие в названии таблицы на тире
\usepackage[pdftex]{graphicx}

\usepackage{geometry, multicol} % Меняем поля страницы
\geometry{left=1.5cm}% левое поле
\geometry{right=1.5cm}% правое поле
\geometry{top=1cm}% верхнее поле
\geometry{bottom=2cm}% нижнее поле

\renewcommand{\thetable}{\arabic{table}}
%\renewcommand{\thesubsection}{}

\renewcommand{\tabularxcolumn}[1]{m{#1}}
\newcolumntype{C}[1]{>{\setlength{\hsize}{#1\hsize}\centering\arraybackslash}X}
\newcolumntype{T}[1]{>{\setlength{\hsize}{#1\hsize}}X}
\newcolumntype{Q}[1]{>{\centering\setlength{\hsize}{#1\hsize}\arraybackslash\vspace{1mm}$\displaystyle}X<{$\vspace{1mm}}}
\newcolumntype{,}{D{,}{,}{14}}
\renewcommand{\captionlabeldelim}{\textendash}
\graphicspath{{images/}}

%\usepackage[usenames,dvipsnames]{color}
%\usepackage[numbered,framed]{mcode}

\tolerance=10000

\begin{document}
	%\tableofcontents
\newpage
\begin{center}
\begin{tabularx}{\linewidth}{|C{1.8}|C{0.6}|C{0.6}|}
	\hline	
	\multirow{6}{1.8\linewidth}{\centering{\textbf{Лабораторная работа №1}} \\ Атомная и релятивистская системы единиц физических величин}& Студент, гр. & Аликов С. А., Ф-469 \\ \cline{2-3}
	& Дата выполнения & \\ \cline{2-3}
	& Преподаватель & \\ \cline{2-3}
	& Дата отчёта & \\ \cline{2-3}
	& Оценка & \\ \cline{2-3}
	& Преподаватель & \\ \hline
\end{tabularx}
\end{center}

\renewcommand{\tabularxcolumn}[1]{p{#1}}
\begin{center}
\begin{tabularx}{\linewidth}{T{0.5}|T{1.5}}
	\textit{\textbf{Цель работы:}}
	& Познакомиться с атомной и релятивистской системами единиц и выразить единицы различных физических величин (длины \(L\), времени \(T\), скорости \([v] = LT^{-1}\), энергии \([E] = ML^2T^{-2}\) и гравитационной постоянной \(\gamma\)) в этих системах в единицах СГС
\end{tabularx}
\end{center}

\section{Естественные системы единиц}
\subsection{Атомная система единиц}	
	
	В атомной системе единиц принимают, что фундаментальные постоянные~--- постоянная Планка~\(\hbar\), заряд электрона~\(e\), масса электрона~\(m_e\), равны единице.
	\begin{equation}\label{eq: eq1}
	\hbar, e, m_e = 1
	\end{equation}
	в этом случае в качестве единицы длины выступает радиус первой боровской орбиты. Согласно второму закону Ньютона и правилу квантования Бора:
	\begin{equation*}
	\left\{
	\begin{aligned}
		& \frac{m_e v_n^2}{r_n} = \frac{e^2}{r_n^2} \\
		& m_e v_n r_n = n\hbar
	\end{aligned}
	\right.
	\Rightarrow
	\left\{
	\begin{aligned}
		& m_e v_n^2 r_n = e^2 \\
		& m_e v_n r_n = n\hbar
	\end{aligned}
	\right.	
	\Rightarrow
	\left\{
	\begin{aligned}
		& n \hbar v_n = e^2 \\
		& m_e v_n r_n = n\hbar
	\end{aligned}
	\right.	
	\end{equation*}
	
	Из первого равенства находим скорость на \(n\)-й боровской орбите, подставляя её во второе равенство получаем соотношение для радиуса:
	\begin{equation*}
	\left\{
	\begin{aligned}
		& v_n = \frac{e^2}{n\hbar} \\
		& m_e v_n r_n = n\hbar
	\end{aligned}
	\right.	
	\Rightarrow
	\left\{
	\begin{aligned}
		& v_n = \frac{e^2}{n\hbar} \\
		& r_n = \frac{n\hbar}{m_e v_n} = \frac{n^2\hbar^2}{m_e e^2}
	\end{aligned}
	\right.	
	\end{equation*}
	
	При \(n = 1\) и учитывая (\ref{eq: eq1}), получаем, что скорость и радиус на первой боровской орбите являются единицами соответствующих физических величин в атомной системе единиц.
	\[
	L = r_1 = \frac{\hbar^2}{m_e e^2} \quad
	[v] = LT^{-1} = v_1 = \frac{e^2}{\hbar} = \frac{e^2}{\hbar c} c = \alpha c
	\]
	где \(\alpha\) --- постоянная тонкой структуры, \(\sqrt{\hbar c}\) носит название заряда Дирака. Найдём оставшиеся единицы:
	\[
	T = \frac{L}{[v]} = \frac{~\cfrac{\hbar^2}{m_e e^2}~}{\cfrac{e^2}{\hbar}} = \frac{\hbar^3}{m_e e^4}
	\]
	
	Единица энергии соответствует  потенциальной энергии на первой боровской орбите
	\[
	[E]=ML^2T^{-2} = \frac{e^2}{r_1} = \frac{m_e e^4}{\hbar^2}
	\]
	
	Гравитационная постоянная
	\[
	\gamma = 6,67\cdot10^{-8}\text{ см}^3/(\text{г}\cdot\text{c}^2)
	\]
	
	Используя выражения из таблицы (\ref{tab: tab1}) найдём сколько атомных единиц длины, времени и массы содержится в \(1\) сантиметре, грамме и секунде;
	{
	\begin{equation*}
	\begin{split}
	& 1\text{~см} = 1/\{r_1[\text{см}]\} \text{~а.~е.~длины} =
		1,890\cdot10^{8} \text{~а.~е.~длины} \\
	& 1\text{~г} = 1/\{m_e[\text{г}]\} \text{~а.~е.~массы} = 
		1,098\cdot10^{27} \text{~а.~е.~массы} \\
	& 1\text{~c} = 1/\{T[\text{с}]\} \text{~а.~е.~времени} = 
	4,134\cdot10^{16} \text{~а.~е.~времени}		
	\end{split}
	\end{equation*}
	}
	в результате 
	\[
	\gamma = 
	6,67\cdot10^{-8} \frac{1,890^3\cdot10^{24}}{4,134^2\cdot10^{32}\, 1,098\cdot10^{27}} = 2,399\cdot10^{-43}  \text{~а.~е.~}L^3M^{-1}T^{-2}
	\]
	
	\subsection{Релятивистская система единиц}
	В релятивистской системе единиц принимают равными единице фундаментальные постоянные~---  постоянную Планка~\(\hbar\), скорость света в вакууме~\(c\) и массу электрона~\(m_e\)
	\begin{equation}\label{eq: eq2}
		\hbar, c, m_e = 1
	\end{equation}
	
\begin{table}[h!]
\caption{Сводная таблица единиц физических величин в атомной и релятивистской системах}
\label{tab: tab1}
\renewcommand{\tabularxcolumn}[1]{m{#1}}
%\renewcommand{\arraystretch}{2}
\begin{tabularx}{\linewidth}{|C{1.0}|Q{1.0}|,|C{1.0}|Q{1.0}|,|}
	\hline
	\multicolumn{1}{|C{1.0}|}{Единица в атомной системе}& 
	\multicolumn{1}{C{1.0}|}{Выражение в атомной системе}&
	\multicolumn{1}{C{1.0}|}{Значение}&
	\multicolumn{1}{|C{1.0}|}{Единица в релятивистской системе}& 
	\multicolumn{1}{C{1.0}|}{Выражение в релятивистской системе}&
	\multicolumn{1}{C{1.0}|}{Значение} \\\hline
	1~а.~е.~длины
	& \frac{\hbar^2}{m_e e^2}
	& 5,2919 \cdot 10^{-9} \text{ см}
	& 1~р.~е.~длины
	& \frac{\hbar}{m_ec}
	& 3,8616 \cdot 10^{-11} \text{ см}
	\\ \hline
	1~а.~е.~времени
	& \frac{\hbar^3}{m_e e^4}
	& 2,4190 \cdot 10^{-17} \text{ с}
	& 1~р.~е.~времени
	& \frac{\hbar}{m_ec^2}
	& 1,2881 \cdot 10^{-21} \text{ с}
	\\ \hline
	1~а.~е.~скорости
	& \frac{e^2}{\hbar}
	& 2,1876 \cdot 10^{8} \text{ см/с}
	& 1~р.~е.~скорости
	& c
	& 2,9979 \cdot 10^{10} \text{ см/с}
	\\ \hline
	1~а.~е.~энергии
	& \frac{m_e e^4}{\hbar^2}
	& 4,3596 \cdot 10^{-11} \text{ эрг}
	& 1~р.~е.~энергии
	& m_e c^2
	& 8,1872 \cdot 10^{-7} \text{ эрг} 
	\\ \hline
	1~а.~е.~энергии
	& \frac{m_e e^4}{\hbar^2}
	& 27,21 \text{ эВ}
	& 1~р.~е.~энергии
	& m_e c^2
	& 0,511 \text{ МэВ} 
	\\ \hline \hline
	\(\gamma\)
	& \text{---}
	& 2,399\cdot10^{-43}\text{~а.~е.}
	& \(\gamma\)
	& \text{---}
	& 1,751\cdot10^{-45}\text{~р.~е.}
	\\ \hline
\end{tabularx}
\end{table}	
	
	Чтобы найти интересующие нас величины проведём анализ размерностей:
	\begin{equation*}
	\left\{
	\begin{split}
		& [\hbar] = [Flt] = [ml^2t/t^2] = ML^2T^{-1} \\
		& [c] = LT^{-1} \\
		& [m_e] = M
	\end{split}
	\right.
	\Rightarrow
	\left\{
	\begin{split}
		& [\hbar] = L[m_e][c] = T [m_e][c^2] \\
		& [c] = LT^{-1} \\
		& [m_e] = M
	\end{split}
	\right.	
	\end{equation*}
	Получаем
	\begin{equation*}
		L = \left[\frac{\hbar}{m_ec}\right] \quad
		T = \left[\frac{\hbar}{m_ec^2}\right] \quad
		[E] = [m_e c^2]
	\end{equation*}	
	В результате:
	\begin{equation*}
	{
	\begin{split}
		& L = \frac{\hbar}{m_ec}=\lambda_C \\
		& T = \frac{\hbar}{m_ec^2} \\
		& [v] = LT^{-1} = c \\
		& [E] = ML^2T^{-2} = m_e c^2
	\end{split}
	}
	\end{equation*}	
	где \(\lambda_C\)~--- комптоновская длина волны.
	Используя таблицу (\ref{tab: tab1}) найдём сколько релятивистских единиц длины, времени и массы содержится в \(1\) сантиметре, грамме и секунде;
	{
	\begin{equation*}
	\begin{split}
	& 1\text{~см} =
		2,590\cdot10^{10} \text{~р.~е.~длины} \\
	& 1\text{~г} =  
		1,098\cdot10^{27} \text{~р.~е.~массы} \\
	& 1\text{~c} =  
		7,763\cdot10^{20} \text{~р.~е.~времени}		
	\end{split}
	\end{equation*}
	}	
	в результате 
	\[
	\gamma = 
	6,67\cdot10^{-8} \frac{2,590^3\cdot10^{30}}{7,763^2\cdot10^{40}\, \cdot10^{27}} = 1,751\cdot10^{-45}  \text{~р.~е.~}L^3M^{-1}T^{-2}
	\]	
	
	Можно связать единицы в релятивистской системе и единицы в атомной системе с помощью постоянной тонкой структуры:
	\begin{equation*}
	{
	\begin{split}
	& \text{1~а.~е.~длины} = \frac{\hbar^2}{m_e e^2} = \frac{\hbar c}{e^2} \frac{\hbar}{m_e c} = \frac{1}{\alpha} \text{~р.~е.~длины} \\
	& \text{1~а.~е.~времени} = \frac{\hbar^3}{m_e e^4} = \frac{\hbar^2 c^2}{e^4}  \frac{\hbar}{m_ec^2} = \frac{1}{\alpha^2} \text{~р.~е.~времени} \\
	& \text{1~а.~е.~скорости} = LT^{-1} = \frac{1}{\alpha} \colon \frac{1}{\alpha^2} = \alpha \text{~р.~е.~скорости} \\
	& \text{1~а.~е.~энергии} = M[v]^2 = \alpha^2 \text{~р.~е.~энергии}
	\end{split}
	}
	\end{equation*}

Результаты расчётов приведены в таблице \label{tab: tab1}.

\newpage
\begin{center}
\begin{tabularx}{\linewidth}{|C{1.8}|C{0.6}|C{0.6}|}
	\hline	
	\multirow{6}{1.8\linewidth}{\centering{\textbf{Лабораторная работа №2}} \\ Атом Томаса-Ферми}& Студент, гр. & Аликов С. А., Ф-469 \\ \cline{2-3}
	& Дата выполнения & \\ \cline{2-3}
	& Преподаватель & \\ \cline{2-3}
	& Дата отчёта & \\ \cline{2-3}
	& Оценка & \\ \cline{2-3}
	& Преподаватель & \\ \hline
\end{tabularx}
\end{center}

\renewcommand{\tabularxcolumn}[1]{p{#1}}
\begin{center}
\begin{tabularx}{\linewidth}{T{0.5}|T{1.5}}
	\textit{\textbf{Цель работы:}}
	& Познакомиться с моделью атома Томаса-Ферми, построить потенциальную энергию от расстояния до ядра.
\end{tabularx}
\end{center}

\section{Атом Томаса-Ферми}
\subsection{Вывод уравнения Томаса-Ферми}

	Поставим перед собой задачу найти потенциал атома. Он будет определяться потенциалом ядра и потенциалом окружающих ядро электронов. В модели атома Томаса-Ферми полагают, что для электронов атома применимо распределение Ферми-Дирака, то есть мы рассматриваем ядро атома окружённое газом из электронов. Атом полагаем сферически симметричным. В этом случае потенциал атома зависит только от расстояния до ядра. Потенциальная энергия отдельного электрона в таком потенциале:
	\[
		U(r) = -e \varphi(r)
	\]
	
	Распределение Ферми-Дирака имеет вид:
	\begin{equation}
	\label{eq: Fermi}
	f = \cfrac{1}{1+\exp \bigg(\cfrac{E - \zeta}{kT}\bigg)}
	\end{equation}
	где \(\zeta \) --- химический потенциал. Он показывает насколько увеличится энергия системы~(в~данном случае атома) при добавлении ещё одного электрона. Для атома он равен:
	\begin{equation}
	\label{eq: zeta}
	\zeta = U(r) + \cfrac{p_F^2(r)}{2m_e}
	\end{equation}
	где \(p_F(r)\) --- импульс электрона на уровне Ферми или максимальный импульс электрона. Химический потенциал является постоянной величиной для данной системы, а так как электрический потенциал и потенциальная энергия определены с точностью до константы, можно ввести вспомогательную величину, совпадающую по физическому смыслу с потенциальной энергией:
	\begin{equation}\label{eq: u1}
	U_1 = U - \zeta = -  \cfrac{p_F^2(r)}{2m_e}
	\end{equation}
	
	При этом для \(U_1\) останется справедливым уравнение Пуассона:
	\begin{equation}
		\Delta \varphi = - 4\pi \rho
		\quad \Rightarrow \quad
		\Delta U = 4 \pi e \rho
		\quad \Rightarrow \quad
		\Delta U_1 = 4 \pi e \rho = - 4\pi e^2 n
	\end{equation}
	где \(n\) --- концентрация электронов. В силу предположения о сферической симметрии она является функцией только расстояния до центра атома.
	Покажем, что концентрацию электронов и импульс Ферми можно связать друг с другом. Для этого найдём количество квантовых состояний импульсы которых попадают в интервал \((p, p+dp)\). 
	Из соотношения неопределённостей следует что в ячейке фазового пространства \(\Delta \Gamma = \Delta x\,\Delta y\,\Delta z\,\Delta p_x\Delta p_y\Delta p_z = (2\pi\hbar)^3\) может находиться только два электрона с противоположными спинами. В результате число квантовых состояний в ячейке фазового пространства \(d\Gamma\):
	\[
	dN = 2 \cfrac{d\Gamma}{\Delta \Gamma}
	\]
	Для сферического слоя \((p, p+dp)\):
	\[
	d\Gamma = 4\pi p^2 dp dV
	\]
	В результате
	\[
	dN = 2 \cfrac{4\pi p^2 dp dV}{(2\pi\hbar)^3}
	\quad \Rightarrow \quad
	n = \cfrac{dN}{dV} = \cfrac{p^2 dp}{\pi^2\hbar^3}
	\]
	Интегрируя до максимального импульса \(p_F(r)\):
	\begin{equation*}
	n(r) = \cfrac{p_F^3(r)}{3\pi^2\hbar^3}
	\end{equation*}
	или
	\begin{equation}\label{eq: n}
	p_F^3(r) = 3\pi^2\hbar^3 n(r)
	\end{equation}	
	Из (\ref{eq: u1}) и (\ref{eq: n})
	\begin{equation*}
	n(r) =  \cfrac{(2m_e)^{3/2}}{3\pi^2\hbar^3} [-U_1]^{3/2}
	\end{equation*}
	Подставляем полученное выражение в уравнение Пуассона, в котором оставляем только радиальную часть оператора Лапласа. Получаем уравнение для потенциальной энергии \(U_1\):
	\begin{equation}\label{eq: tomas}
	\cfrac{1}{r} \cfrac{d^2}{dr^2}(rU_1) =
	- \cfrac{4 e^2(2m_e)^{3/2}}{3\pi\hbar^3} (-U_1)^{3/2}
	\end{equation}
	Введём обозначения:
	\begin{equation}\label{eq: norma}
	\begin{split}
		& \Phi = \cfrac{U_1(r)}{-Ze^2/r} \\
		& x = \cfrac{1}{ba_0Z^{-1/3}} r \\
		& b = \cfrac{(3\pi)^{2/3}}{2^{7/3}}
	\end{split}
	\end{equation}
	При этих обозначения уравнение (\ref{eq: tomas}) примет вид:
	\begin{equation}\label{eq: tomas2}
	\cfrac{d^2\Phi}{dx^2} = \cfrac{\Phi^{3/2}}{x^{1/2}}
	\end{equation}
	
	Действительно из (\ref{eq: tomas2}) следует
	\begin{gather*}
	-\cfrac{b^2 a_0^2 Z^{-2/3}}{Ze^2}
	\cfrac{d^2 (r U_1)}{dr^2} = 
	\cfrac{b^{1/2}a_0^{1/2}Z^{-1/6}}{Z^{3/2} e^3}
	\cfrac{r^{3/2}(-U_1)^{3/2}}{r^{1/2}} \\
	\Downarrow \\
	\cfrac{d^2 (r U_1)}{dr^2} =
	\cfrac{b^{-3/2}a_0^{-3/2}}{e} r (-U_1)^{3/2}
	\end{gather*}
	Учитывая, что \(a_0 = \hbar^2/m_e e^2\)  и подставляя \(b\)
	\[
	\cfrac{d^2 (r U_1)}{dr^2} =
	\cfrac{2^{7/2}m_e^{3/2}e^2}{(3\pi)\hbar^3}r (-U_1)^{3/2}
	\]
	что совпадает с (\ref{eq: tomas}).	
	
	Функция \(\Phi\) носит название функции экранирования. Она показывает как сильно отличается потенциал атома от потенциала ядра, то есть характеризует вклад электронов в атомный потенциал.	
	
	Граничные условия для уравнения (\ref{eq: tomas2}) найдём из условия, что вблизи 0 потенциал ведёт себя как потенциал голого ядра \(-Ze^2/r\), а вдали должен убывать до нуля.
	\begin{equation}
	\begin{split}
	& \Phi(0) = 1 \\
	& \Phi(\infty) = 0 
	\end{split}
	\end{equation}
	
	\subsection{Решение уравнения Томаса-Ферми}
	
	Уравнение (\ref{eq: tomas2}) решается численно после приведение к виду:
	\begin{equation}
	\begin{split}
	& \cfrac{d\xi}{dx} = \cfrac{\Phi^{3/2}}{x^{1/2}} \\
	& \cfrac{d\Phi}{dx} = \xi
	\end{split}
	\end{equation}
	Начальное условие для \(\xi\) подбирается, так чтобы удовлетворялось второе условие для \(\Phi\). Задача имеет особую точку $x = 0$ и никакой заменой устранить эту особую точку невозможно. В этом смысле задача является жёстской и не может быть решена численно, пока точка не исключена. Как это сделать нам подсказывает физическая действительность. В самом деле внутри атома кроме электронов существует ядро, внутри которого потенциальная энергия зависит от радиуса не по закону $-Ze^2/r$. Даже простейшая модель ядра как равномерно заряженной жидкости даёт иной результат. Поэтому задачу следует решать не на промежутке $[0, \infty]$, а на промежутке $[x_0, \infty]$, где
\[
	x_0 = \frac{1}{bZ^{-1/3}}\, r_a	
\]

$r_a = r_0/a_0$ -- радиус ядра в атомных единицах. В качестве радиуса ядра будем использовать радиус ядра атома углерода:
\[
	r_0 = 				
1{.}2\cdot10^{-13}\cdot12^{-1/3}
\]

%	\begin{figure}[b]
%	\begin{multicols}{2}
%	\centering
%	\caption{Потенциал Томаса-Ферми для атома углерода}
%	\label{ris: 2.1}
%	\includegraphics[width = 0.5\textwidth]{21.pdf}
%	
%	\caption{Потенциал Томаса-Ферми для атома кобальта}
%	\label{ris: 2.2}
%	\includegraphics[width = 0.5\textwidth]{22.pdf}
%	\end{multicols}
%	\end{figure}

Дальнейшее решение задачи требует определение начального условия для $\xi(x_0)$. Для этой цели используем таблицу значений функции $\Phi$ в зависимости от $x$, приведённую у Ландау. Выбираем $\xi(x_0)$ таким образом, чтобы численное решение уравнения прошло через точки, заданные в этой таблице. После этого переходим к переменным $U_1$ и $r$ и строим соответствующие зависимости.
	
	На рисунке (\ref{ris: 2.3}) приведена зависимость потенциальной энергии $U_1$ электрона в поле атома от расстояния до его центра для атомов углерода и кобальта.
	
	\begin{figure}[ht]
	\centering
	\caption{Потенциал для атомов углерода и кобальта}
	\label{ris: 2.3}
	\includegraphics[width = 0.8\textwidth]{23.pdf}
	\end{figure}
	
	Код решающий задачу на SciLAB приведён в приложении \ref{pril1}
	
\newpage
\begin{center}
\begin{tabularx}{\linewidth}{|C{1.8}|C{0.6}|C{0.6}|}
	\hline	
	\multirow{6}{1.8\linewidth}{\centering{\textbf{Лабораторная работа №2}} \\ Сечение упругого рассеяния в борновском приближение.}& Студент, гр. & Аликов С. А., Ф-469 \\ \cline{2-3}
	& Дата выполнения & \\ \cline{2-3}
	& Преподаватель & \\ \cline{2-3}
	& Дата отчёта & \\ \cline{2-3}
	& Оценка & \\ \cline{2-3}
	& Преподаватель & \\ \hline
\end{tabularx}
\end{center}

\renewcommand{\tabularxcolumn}[1]{p{#1}}
\begin{center}
\begin{tabularx}{\linewidth}{T{0.5}|T{1.5}}
	\textit{\textbf{Цель работы:}}
	& Рассчитать форм-фактор, дифференциальное сечение упругого рассеяния и полное сечение для выбранных атомов.
\end{tabularx}
\end{center}

\section{Аналитические функции экранирования}

Большинство из предложенных приблизительных аналитических функций 
экранирования основаны на статистической модели атома Томаса - Ферми 
(TФ); есть только несколько исключений, основанных на самосогласованных 
вычислениях Хартри-Фока (ХФ) или Хартри-Фока-Слейтера (ХФС). Удобны для дальнейшего применения функции экранирования в форме Мольер и Сальвата.

\subsection{Приближение Мольер}

Уравнение (\ref{eq: tomas}) не имеет аналитического 
решения, но хорошо аппроксимируется формулой Мольер
\begin{equation}
\Phi(r)=\sum_{i=1}^3B_i\exp(-\beta_ir/b)
\label{eq:a8}
\end{equation}
где
$$B_1=0.1,\quad B_2=0.55,\quad B_3=0.35, $$
$$\beta_1=6.0,\quad \beta_2=1.2,\quad \beta_3=0.3$$
 
Функция (8) отличается от точного решения уравнения (7) меньше чем на 
0.002 в диапазоне $0<x<6$.

\subsection{Приближение Сальвата}

Использование уравнения (\ref{eq: tomas}) неоправдано, так как оно является приближённым. Правильнее было бы использовать уравнение Дирака. После решения уравнения Дирака в приближении Хартри-Фока-Слейтера (то есть в приближении самосогласованного поля) можно найти потенциал и функцию экранирования. Приближение Сальвата предполагает использование функции экранирования вида:
\begin{equation}
{\Phi}(r) = \sum\limits_{i = 1}^3 {{A_i}\exp ( - {a_i}r)};
\label{eq:Salvat11}
\end{equation}
при этом параметры функции экранирования $A_i,\, a_i$ для каждого атома индивидуальны. 

Параметры функции экранирования (\ref{eq:Salvat11}) находились из требования совпадения значений радиальных моментов $R_n$, полученных из нее, с вычисленными методом ДХФС, для $n = - 1, 0, 1, 2, 3, 4 $. Это приводит к следующим соотношениям
\begin{equation}
\begin{gathered}
  A_1 a_1 + A_2 a_2 + A_3 a_3 = R_{-1}, \hfill \\
  A_1 + A_2 + A_3 = 1, \hfill \\
  \frac{A_1}{a_1^n} + \frac{A_2}{a_2^n} + \frac{A_3}{a_3^n} = R_n\quad (n = 1,2,3,4). \hfill \\
\end{gathered}
\label{eq:Salvat15}
\end{equation}
Здесь
\[
R_n \equiv \frac{1}{{(n + 1)!Z}}\int {{r^n}\rho (r){d^3}r}
\label{eq:Salvat13}
\]
и, в соответствии с уравнением Пуассона,
\begin{equation}
\rho (r) = \frac{Z}{4\pi r}\frac{d^2 \Phi(r)}{d r^2}
= \frac{Z}{4\pi r}\sum\limits_{i = 1}^3 {{A_i}a_i^2\exp ( - {a_i}r)} .
\label{eq:rhoPoisson}
\end{equation}

Легко видеть, что
\[
\begin{gathered}
  {R_{ - 1}} = \Phi '(0), \hfill \\
  {R_0} = \Phi (0), \hfill \\
  {R_n} = \frac{1}
{{(n - 1)!}}\int\limits_0^\infty  {{r^{n - 1}}\Phi (r)dr\quad (n \geqslant 1).}  \hfill \\
\end{gathered}
\]

Использование радиальных моментов для определения коэффициентов делает сечения рассеяния Борна практически 
совпадающими с вычислеными с помощью экранирующей функции ДХФС. 


\section{Сечение упругого рассеяния в борновском приближении}

В борновском приближении для быстрой частицы дифференциальное сечение определяется по формуле:
\begin{equation*}
	\frac{d\sigma}{d\Omega} =
	\frac{4Z^2}{q^4}
	\Big[
		1 - \frac{F(q)}{Z}
	\Big]^2
\end{equation*}

где q - переданный импульс в столкновении и 
\begin{equation}
F(q)=\int\limits_0^{\infty}\frac{\sin(qr)}{qr}\rho(r)4\pi r^2dr
\label{eq:b2}
\end{equation}
-- атомный форм-фактор. Для плотности, связанной аналитическими экранирующими функциями в форме Мольер и Сальвата, форм-фактор 
принимает простое выражение 
\begin{equation}
\frac{F(q)}{Z}=\sum_{i=1}^3\frac{A_ia_i^2}{a_i^2+q^2}.
\label{eq:b3}
\end{equation}

Полное сечение рассеяния найдём после интегрирования 
\begin{equation}
\sigma =
\int\limits_0^{2\pi}
d\varphi
\int\limits_0^\pi
\frac{4Z^2}{q^4}
	\Big[
		1 - \frac{F(q)}{Z}
	\Big]^2
\sin \theta
d\theta
\end{equation}
 
$q$ -- переданный импульс. При упругом столкновении энергия частицы $E$ не меняется. Это значит, что импульс по модулю $p = \sqrt{2E/m}$(в атомной системе $p = k = \sqrt{2E}$) тоже не меняется. $\theta$ -- угол между начальным и конечным направлениями импульса. В результате $q = 2 k \sin \theta/2$. Выполняем замену в интеграле:
\begin{equation}
\frac{q^2}{2k^2} = 1-\cos \theta
\end{equation}

Получаем полное сечение в виде:
\begin{equation}
\sigma =
\pi \frac{4Z^2}{E}
\int\limits_0^{2k}
\frac{1}{q^3}
	\Big[
		1 - \frac{F(q)}{Z}
	\Big]^2
dq 
\label{eq: 21}
\end{equation}

Подставим в последнюю формулу приближённое выражение $F(q)$ и заменим единицу суммой коэффициентов $A_i$, получим
\begin{equation*}
\sigma =
\pi \frac{4Z^2}{E}
\int\limits_0^{2k}
\frac{1}{q^3}
	\Big[
		\sum_{i=1}^3 \left( A_i - \frac{A_ia_i^2}{a_i^2+q^2} \right)
	\Big]^2
dq
\end{equation*}

Получаем
\begin{equation}
\sigma =
\pi \frac{4Z^2}{E}
\int\limits_0^{2k}
	\Big[
		\sum_{i=1}^3 \frac{A_i}{a_i^2+q^2}
	\Big]^2 q
dq \label{eq: sect}
\end{equation}

	\begin{figure}[h!]
	\centering
	\caption{Экранирующие функции для атомов углерода и кобальта в приближениях Мольер и Сальвата}
	\label{ris: 3.1}
	\includegraphics[width = 0.7\textwidth]{31.pdf}
	\end{figure}
	
	\begin{figure}[h!]
		\centering
		\caption{Плотность зарядов (электронов) для атомов углерода и кобальта}
		\label{ris: 3.2}
		\includegraphics[width = 0.7\textwidth]{32.pdf}
	\end{figure}
	\newpage

Это выражение имеет то преимущество перед (\ref{eq: 21}, что не обладает особенностью в 0. Его удобно интегрировать численно. Для представления результатов был написан код в среде Scilab. Он приведён в приложении \ref{pril2}. В его основу легли таблицы из работы Сальвата \cite{Salvat} и соотношения [\ref{eq:a8}-\ref{eq: sect}]. Результаты численного моделирования для атомов кобальта и углерода приведены на рисунках \ref{ris: 3.1}-\ref{ris: 3.6}.

	\begin{figure}[h!]
		\centering
		\caption{Радиальная плотность зарядов (электронов) для атомов углерода и кобальта}
		\label{ris: 3.3}
		\includegraphics[width = 0.7\textwidth]{33.pdf}
	\end{figure}
	
	\begin{figure}[h!]
		\centering
		\caption{Форм-фактор для атомов углерода и кобальта}
		\label{ris: 3.4}
		\includegraphics[width = 0.7\textwidth]{34.pdf}
	\end{figure}
	\newpage
		
	\begin{figure}[h!]
		\centering
		\caption{Дифференциальное эффективное сечение упругого рассеяния электронов в борновском приближении для атомов углерода и кобальта}
		\label{ris: 3.5}
		\includegraphics[width = 0.7\textwidth]{35.pdf}
	\end{figure}
	
	\begin{figure}[h!]
		\centering
		\caption{Полное сечение упругого рассеяния электронов в борновском приближении для атомов углерода и кобальта в приближениях Мольер и Сальвата}
		\label{ris: 3.6}
		\includegraphics[width = 0.7\textwidth]{36.pdf}
	\end{figure}
	
\clearpage
\begin{thebibliography}{10}
\bibitem{Salvat} Salvat, F. Analytical Dirac-Hartree-Fock-Slater screening 
function for atom $(Z=1-92)$ / F. Salvat, J.D. Martiner, R. Mayol, J. 
Parellada / Physical Review, - Volume 36, Number 2, - July 15, 1987. - 
467-474 p.
\bibitem{Landau} Ландау,~Л.~Д. Теоретическая физика. В 10 т. Т. 3. Квантовая механика. Нерелятивистская теория. / Л.~Д.~Ландау, Е.~М.~Лифшиц.~--- Москва~: Наука, 1989.~--- 768 с.
\end{thebibliography}
\newpage 
\appendix

\section{Листинг к вопросу: <<Экранирующая функция и потенциальная энергия атома в приближении Томаса-Ферми>>}
	\label{pril1}
	\begin{verbatim}
	//Правая часть системы ДУ
	function ydot = screening(x,y)
	ydot =[y(2) y(1)^1.5/sqrt(x)];
	endfunction
	//Таблица значений из Ландау
	function dat = Landau()
	dat = [
	0.0     1.0
	0.02	0.972
	0.04	0.947
	0.06	0.924
	0.08	0.902
	0.10	0.882
	0.2	0.793
	0.3	0.721
	0.4	0.660
	0.5	0.607
	0.6	0.561
	0.7	0.521
	0.8	0.485
	0.9	0.453
	1.0	0.424
	1.2	0.374
	1.4	0.333
	1.6	0.298
	1.8	0.268
	2.0	0.243
	2.2	0.221
	2.4	0.202
	2.6	0.185
	2.8	0.170
	3.0	0.157
	3.2	0.145
	3.4	0.134
	3.6	0.125
	3.8	0.116
	4.0	0.108
	4.5	0.0919
	5.0	0.0788
	6	0.0594
	7	0.0461
	8	0.0366
	9	0.0296
	10	0.0243 
	11 0.0202
	12 0.0171
	13 0.0145
	14 0.0125
	15 0.0108
	20 0.0058
	25 0.0035
	30 0.0023
	];
	endfunction
	
	//Константы
	N = 30;
	a0 = 5.28e-9; 
	b = 0.88534;
	e = 4.8032e-10;
	ee = 1.6e-12
	e2 = e^2;
	r0 = 1.2e-13*12^(1/3);
	
	//Очищаем массивы
	clear x; clear r; clear Y; clear khi; clear U;
	
	//Задаём заряд атома
	Z = 6;
	//Определяем нулевое значение для данного атома
	x0 = r0/(b*a0*Z^(-1/3));
	//Формируем массивы x и r
	x = [x0:0.01:N];
	r =  x*(b*Z^(-1/3));
	//Задаём начальные условия второе 	из них подбираем 
	//по совпадению зависимости с Ландау
	y0 = [1; -1.5677562];
	//Решаем систему
	Y = ode("stiff",y0,x0,x,screening);
	//Получаем значения $\Phi$
	khi = Y(1,:);
	
	dat = Landau();
	xL = dat(:,1);
	rL =  xL*(b*Z^(-1/3)); 
	khiL = dat(:,2);
	
	//Строим зависимости
	set("current_figure",1)
	clf();
	plot(x, khi, "r-", xL,khiL, ".b" );
	xlabel('x, AU'), ylabel('$\Phi$');
	
	U = - Z*e2/a0/e*khi./r;
	set("current_figure",2)
	clf();
	plot(r, U, ".r-");
	xlabel('r, AU'), ylabel('$U, эВ$');
	
	//Повторяем код для других элементов
	clear x; clear r; clear Y; clear khi; clear U;
	N = 30;
	Z = 27;
	r0 = 1.2e-13*12^(1/3);
	x0 = r0/(b*a0*Z^(-1/3));
	x = [x0:0.01:N];
	r =  x*(b*Z^(-1/3));
	y0 = [1; -1.562094];
	Y = ode("stiff",y0,x0,x,screening);
	khi = Y(1,:);
	
	dat = Landau();
	xL = dat(:,1);
	rL =  xL*(b*Z^(-1/3));
	khiL = dat(:,2);
	
	set("current_figure",3)
	clf();
	plot(x, khi, "r-", xL,khiL, ".b" );
	xlabel('x, AU'), ylabel('$\Phi$');
	
	U = - Z*e2/a0/e*khi./r;
	set("current_figure",4)
	clf();
	plot(r, U, ".r-");
	xlabel('r, AU'), ylabel('$U, эВ$');
	\end{verbatim}
	
\section{Листинг к вопросу: <<Сечение рассеяния в борновском приближении для различных атомов>>}
	\label{pril2}
\begin{verbatim}
	//Приближение и Мольер и Сальвата для функции экранирования
	function result = screen(A1, A2, A3, a1, a2, a3, r)
	    result = A1*exp(-a1*r) + A2*exp(-a2*r) + A3*exp(-a3*r);
	endfunction
	//Плотность заряда в зависимости от расстояния
	function result = density(A1, A2, A3, a1, a2, a3, r, Z)
	    result = Z*r.*(A1*a1^2*exp(-a1*r) + 
	    A2*a2^2*exp(-a2*r) + A3*a3^2*exp(-a3*r));
	endfunction
	//Форм-фактор с точностью до константы Z
	function result = formfactor(A1, A2, A3, a1, a2, a3, q)
	    result = A1*a1^2 ./(a1^2 + q.^2) + 
	    A2*a2^2 ./(a2^2 + q.^2) + A3*a3^2 ./(a3^2 + q.^2);
	endfunction
	//Эффективное дифференциальное сечение рассеяния
	function result = diffsect(A1, A2, A3, a1, a2, a3, q, Z)
	    form = formfactor(A1, A2, A3, a1, a2, a3, q);
	    result = (4*Z^2 ./q.^4).*(1 - form).^2;
	endfunction
	//Подынтегральное выражение для полного сечения рассеяния
	function result = inint( q, A1, A2, A3, a1, a2, a3, Z)
	    result = q.*(A1./(a1^2 + q.^2) + 
	    A2./(a2^2 + q.^2) + A3./(a3^2 + q.^2)).^2;
	endfunction
	//Полное сечение рассеяния
	function result = sect(A1, A2, A3, a1, a2, a3, E, Z)
	    uplimit = sqrt(8*E);
	    downlimit = 0.;
	    result = (%pi*4*Z^2 ./E).*
	    integrate('inint( q, A1, A2, A3, a1, a2, a3, Z)', 
	    'q', downlimit, uplimit);
	endfunction
	//Здесь находится весь код
	function result = all(Z, str)
		 // Таблица из статьи Сальвата
	    table_Z_A1_A2_a1_a2_a3 = [
	        1 -184.39 185.39 2.0027 1.9973 0;
	        2 -0.2259 1.2259 5.5272 2.3992 0;
	        3 0.6045 0.3955 2.8174 0.6625 0;
	        4 0.3278 0.6722 4.5430 0.9852 0;
	        5 0.2327 0.7673 5.9900 1.2135 0;
	        6 0.1537 0.8463 8.0404 1.4913 0;
	        7 0.0996 0.9004 10.812 1.7687 0;
	        8 0.0625 0.9375 14.823 2.0403 0;
	        9 0.0368 0.9632 21.400 2.3060 0;
	        10 0.0188 0.9812 34.999 2.5662 0;
	        11 0.7444 0.2556 4.1205 0.8718 0;
	        12 0.6423 0.3577 4.7266 1.0025 0;
	        13 0.6002 0.3998 5.1405 1.0153 0;
	        14 0.5160 0.4840 5.8492 1.1732 0;
	        15 0.4387 0.5613 6.6707 1.3410 0;
	        16 0.5459 -0.5333 6.3703 2.5517 1.6753;
	        17 0.7249 -0.7548 6.2118 3.3883 1.8596;
	        18 2.1912 -2.2852 5.5470 4.5687 2.0446;
	        19 0.0486 0.7759 30.260 3.1243 0.7326;
	        20 0.5800 0.4200 6.3218 1.0094 0;
	        21 0.5543 0.4457 6.6328 1.1023 0;
	        22 0.0112 0.6832 99.757 4.1286 1.0090;
	        23 0.0318 0.6753 42.533 3.9404 1.0533;
	        24 0.1075 0.7162 18.959 3.0638 1.0014;
	        25 0.0498 0.6866 31.864 3.7811 1.1279;
	        26 0.0512 0.6995 31.825 3.7716 1.1606;
	        27 0.0500 0.7142 32.915 3.7908 1.1915;
	        28 0.0474 0.7294 34.758 3.8299 1.2209;
	        29 0.0771 0.7951 25.326 3.3928 1.1426;
	        30 0.0400 0.7590 40.343 3.9465 1.2759;
	        31 0.1083 0.7489 20.192 3.4733 1.0064;
	        32 0.0610 0.7157 29.200 4.1252 1.1845;
	        33 0.0212 0.6709 62.487 4.9502 1.3582;
	        34 0.4836 0.5164 8.7824 1.6967 0;
	        35 0.4504 0.5496 9.3348 1.7900 0;
	        36 0.4190 0.5810 9.9142 1.8835 0;
	        37 0.1734 0.7253 17.166 3.1103 0.7177;
	        38 0.0336 0.7816 55.208 4.2842 0.8578;
	        39 0.0689 0.7202 31.366 4.2412 0.9472;
	        40 0.1176 0.6581 22.054 4.0325 1.0181;
	        41 0.2257 0.5821 14.240 2.9702 1.0170;
	        42 0.2693 0.5763 14.044 2.8611 1.0591;
	        43 0.2201 0.5618 15.918 3.3672 1.1548;
	        44 0.2751 0.5943 14.314 2.7370 1.1092;
	        45 0.2711 0.6119 14.654 2.7183 1.1234;
	        46 0.2784 0.6067 14.645 2.6155 1.4318;
	        47 0.2562 0.6505 15.588 2.7412 1.1408;
	        48 0.2271 0.6155 16.914 3.0841 1.2619;
	        49 0.2492 0.6440 16.155 2.8819 0.9942;
	        50 0.2153 0.6115 17.793 3.2937 1.1478;
	        51 0.1806 0.5767 19.875 3.8092 1.2829;
	        52 0.1308 0.5504 24.154 4.6119 1.4195;
	        53 0.0588 0.5482 39.996 5.9132 1.5471;
	        54 0.4451 0.5549 11.805 1.7967 0;
	        55 0.2708 0.6524 16.591 2.6964 0.6814;
	        56 0.1728 0.6845 22.397 3.4595 0.8073;
	        57 0.1947 0.6384 20.764 3.4657 0.8911;
	        58 0.1913 0.6467 21.235 3.4819 0.9011;
	        59 0.1868 0.6558 21.803 3.5098 0.9106;
	        60 0.1665 0.7057 23.949 3.5199 0.8486;
	        61 0.1624 0.7133 24.598 3.5560 0.8569;
	        62 0.1580 0.7210 25.297 3.5963 0.8650;
	        63 0.1538 0.7284 26.017 3.6383 0.8731;
	        64 0.1587 0.7024 25.497 3.7364 0.9550;
	        65 0.1453 0.7426 27.547 3.7288 0.8890;
	        66 0.1413 0.7494 28.346 3.7763 0.8969;
	        67 0.1374 0.7558 29.160 3.8244 0.9048;
	        68 0.1336 0.7619 29.990 3.8734 0.9128;
	        69 0.1299 0.7680 30.835 3.9233 0.9203;
	        70 0.1267 0.7734 31.681 3.9727 0.9288;
	        71 0.1288 0.7528 31.353 4.0904 1.0072;
	        72 0.1303 0.7324 31.217 4.2049 1.0946;
	        73 0.1384 0.7096 30.077 4.2492 1.1697;
	        74 0.1500 0.6871 28.630 4.2426 1.2340;
	        75 0.1608 0.6659 27.568 4.2341 1.2970;
	        76 0.1722 0.6468 26.586 4.1999 1.3535;
	        77 0.1834 0.6306 25.734 4.1462 1.4037;
	        78 0.2230 0.6176 22.994 3.7346 1.4428;
	        79 0.2289 0.6114 22.864 3.6914 1.4886;
	        80 0.2098 0.6004 24.408 3.9643 1.5343;
	        81 0.2708 0.6428 20.941 3.2456 1.1121;
	        82 0.2380 0.6308 22.987 3.6217 1.2373;
	        83 0.2288 0.6220 23.792 3.7796 1.2534;
	        84 0.1941 0.6105 26.695 4.2582 1.3577;
	        85 0.1500 0.6031 31.840 4.9285 1.4683;
	        86 0.0955 0.6060 43.489 5.8520 1.5736;
	        87 0.3192 0.6233 20.015 2.9091 0.7207;
	        88 0.2404 0.6567 24.501 3.5524 0.8376;
	        89 0.2266 0.6422 25.684 3.7922 0.9335;
	        90 0.2176 0.6240 26.554 4.0044 1.0238;
	        91 0.2413 0.6304 25.193 3.6780 0.9699;
	        92 0.2448 0.6298 25.252 3.6397 0.9825
	    ];
	    // Коэффициенты для для приближения Сальвата
	    A1 = table_Z_A1_A2_a1_a2_a3(Z,2);
	    A2 = table_Z_A1_A2_a1_a2_a3(Z,3);
	    A3 = 1-A1-A2;
	    a1 = table_Z_A1_A2_a1_a2_a3(Z,4);
	    a2 = table_Z_A1_A2_a1_a2_a3(Z,5);
	    a3 = table_Z_A1_A2_a1_a2_a3(Z,6);
	    // Коэффициенты для приближения Мольер
	    b = 0.88534*Z^(-1/3);
	    B1 = 0.1;
	    B2 = 0.55;
	    B3 = 0.35;
	    beta1 = 6.0 ./b;
	    beta2 = 1.2 ./b;
	    beta3 = 0.3 ./b;
	   
	    N = 300; // Размер массивов
	    // Массив координат в а. е.
	    x = linspace(0.001, 10, N);
	    // Массив энергий в а. е.
	    E = linspace(3e3, 20e3, N);
	    // Массив импульсов в а. е.
	    q = linspace(0.001, 10, N);
	    
	    //Последовательно:
	    //функция экранирования
	    //плотность заряда
	    //радиальная плотность заряда
	    //форм-фактор
	    //эффективное дифференциальное сечение
	    //полное сечение
	    //Первый блок по Мольер, второй по Сальвату
	    molier_screen = screen(B1, B2, B3, beta1, beta2, beta3, x);
	    molier_density = density(B1, B2, B3, beta1, beta2, beta3, x, Z);
	    molier_radial_density = 4*%pi*(x.^2).*molier_density;
	    molier_formfactor = formfactor(B1, B2, B3, beta1, beta2, beta3, q);
	    molier_diffsect = diffsect(B1, B2, B3, beta1, beta2, beta3, q, Z);
	    molier_sect = sect(B1, B2, B3, beta1, beta2, beta3, E, Z);
	    
	    salvat_screen = screen(A1, A2, A3, a1, a2, a3, x);
	    salvat_density = density(A1, A2, A3, a1, a2, a3, x, Z);
	    salvat_radial_density = 4*%pi*(x.^2).*salvat_density;
	    salvat_formfactor = formfactor(A1, A2, A3, a1, a2, a3, q);
	    salvat_diffsect = diffsect(A1, A2, A3, a1, a2, a3, q, Z);   
	    salvat_sect = sect(A1, A2, A3, a1, a2, a3, E, Z);
	    
	    //Строим графики
	    set("current_figure",1);
	    plot(x, molier_screen, strcat([str, "--"]));
	    plot(x, salvat_screen, strcat([str, "-"]));
	    xlabel('$r, \text{а. е.}$'), 
	    ylabel('$\text{Функция экранирования}$');
	    
	    set("current_figure",2);
	    plot(x, molier_density, strcat([str, "--"]));
	    plot(x, salvat_density, strcat([str, "-"]));
	    xlabel('$r, \text{а. е.}$'), 
	    ylabel('$\text{Плотность заряда, а. е.}$');
	    
	    set("current_figure",3);
	    plot(x, molier_radial_density, strcat([str, "--"]));
	    plot(x, salvat_radial_density, strcat([str, "-"]));
	    xlabel('$r, \text{а. е.}$'), ylabel('$\text{Радиальная
	     плотность заряда, а. е.}$');
	    
	    set("current_figure",4);
	    plot(q, molier_formfactor, strcat([str, "--"]));
	    plot(q, salvat_formfactor, strcat([str, "-"]));
	    xlabel('$\text{Переданный импульс }q\text{, а. е.}$'), 
	    ylabel('$\text{Форм-фактор}$');
	    
	    set("current_figure",5);
	    plot(q, molier_diffsect, strcat([str, "--"]));
	    plot(q, salvat_diffsect, strcat([str, "-"]));
	    xlabel('$\text{Переданный импульс }q\text{, а. е.}$'), 
	    ylabel('$\text{Дифференциальное сечение рассеяния}$');
	    
	    set("current_figure",6);
	    plot(x, molier_sect, strcat([str, "--"]));
	    plot(x, salvat_sect, strcat([str, "-"]));
	    xlabel('$\text{Энергия, налетающих электронов }
	    q\text{, а. е.}$'), ylabel('$\text{Полное сечение рассеяния}$');
	
	    result = 1;
	endfunction
	
	//Очищаем графики
	for i = 1:6
	    set("current_figure",i);
	    clf();
	end
	//Применяем функцию all для всех зарядов с параметром
	//цветом линии
	true = all(6, "b");
	true = all(27, "r");
	//Добавляем легенду
	for i = 1:6
	    set("current_figure",i);
	    xgrid(17)
	    legend([strcat(["Z = ", string(6), " по Мольер"]);
	    strcat(["Z = ", string(6), " по Сальвату"]); 
	    strcat(["Z = ", string(27), " по Мольер"]);
	    strcat(["Z = ", string(27), " по Сальвату"])]);
	end
\end{verbatim}
\end{document}