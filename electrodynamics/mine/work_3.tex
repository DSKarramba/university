\documentclass[a4paper, 14pt]{extarticle}
\usepackage[utf8]{inputenc}
\usepackage[paper=a4paper, top=1cm, right=1cm, bottom=1.5cm, left=2cm]{geometry}
\usepackage{setspace}
\onehalfspacing

\usepackage{graphicx}
\graphicspath{{plots/}, {images/}}

\parindent=1.25cm

\usepackage{titlesec}

\titleformat{\section}
    {\normalsize\bfseries}
    {\thesection}
    {1em}{}

\titleformat{\subsection}
    {\normalsize\bfseries}
    {\thesubsection}
    {1em}{}

% Настройка вертикальных и горизонтальных отступов
\titlespacing*{\chapter}{0pt}{-30pt}{8pt}
\titlespacing*{\section}{\parindent}{*4}{*4}
\titlespacing*{\subsection}{\parindent}{*4}{*4}

\usepackage[square, numbers, sort&compress]{natbib}
\makeatletter
\bibliographystyle{unsrt}
\renewcommand{\@biblabel}[1]{#1.} 
\makeatother


\newcommand{\maketitlepage}[6]{
    \begin{titlepage}
        \singlespacing
        \newpage
        \begin{center}
            Министерство образования и науки Российской Федерации \\
            Федеральное государственное бюджетное образовательное \\
            учреждение высшего профессионального образования \\
            <<Волгоградский государственный технический университет>> \\
            #1 \\
            Кафедра #2
        \end{center}


        \vspace{14em}

        \begin{center}
            \large Семестровая работа #6 по дисциплине
            \\ <<#3>>
        \end{center}

        \vspace{5em}

        \begin{flushright}
            \begin{minipage}{.35\textwidth}
                Выполнила:\\#4
                \vspace{1em}\\
                Проверил:\\#5
                \\
                \\ Оценка \underline{\ \ \ \ \ \ \ \ \ \ \ \ \ \ \ \ }
            \end{minipage}
        \end{flushright}

        \vspace{\fill}

        \begin{center}
            Волгоград, \the\year
        \end{center}

    \end{titlepage}
    \setcounter{page}{2}
}

\newcommand{\maketitlepagewithvariant}[7]{
    \begin{titlepage}
        \singlespacing
        \newpage

        \begin{center}
            Министерство образования и науки Российской Федерации \\
            Федеральное государственное бюджетное образовательное \\
            учреждение высшего профессионального образования \\
            <<Волгоградский государственный технический университет>> \\
            #1 \\
            Кафедра #2
        \end{center}


        \vspace{8em}

        \begin{center}
            \large Семестровая работа #6 по дисциплине
            \\ <<#3>>
        \end{center}

        \vspace{1em}
        \begin{center}
            Вариант №#7
        \end{center}
        \vspace{4em}

        \begin{flushright}
            \begin{minipage}{.35\textwidth}
                Выполнила:\\#4
                \vspace{1em}\\
                Проверил:\\#5
                \\
                \\ Оценка \underline{\ \ \ \ \ \ \ \ \ \ \ \ \ \ \ \ }
            \end{minipage}
        \end{flushright}

        \vspace{\fill}

        \begin{center}
            Волгоград, \the\year
        \end{center}

    \end{titlepage}
    \setcounter{page}{2}
}

\input{../../.preambles/10-russian}
\input{../../.preambles/20-math}
\input{../../.preambles/22-vectors}
\input{../../.preambles/30-physics}

\newcommand{\ds}{\displaystyle}

\begin{document}
\maketitlepage{Факультет электроники и вычислительной техники}{физики}
{Электродинамика}{студент группы Ф-369\\Чечеткин~И.~А.}
{доцент Грецов~М.~В.}{№3}

Задача 1.15: \emph{Определить поле, создаваемое заряженным проводящим шаром
радиуса \( a \). Заряд его \( Q \). Диэлектрическая проницаемость окружающей
среды \( \eps = \eps(r) \), где \( r \) -- расстояние от центра шара.}

\vspace*{2em}
\emph{Решение:}
Учитывая, что задача имеет сферическую симметрию, воспользуемся для решения
сферической системой координат. Тогда можно записать следующие соотношения:
\[
	D_r = D(r), \quad D_\theta = D_\psi = 0.
\]

Запишем уравнение Максвелла для индукции электрического поля:
\[
	\div{\vec{D}} = \rho.
\]

Воспользовавшись теоремой Остроградского-Гаусса, имеем:
\[
	\int \div\vec{D}\,dV = \oint \vec{D}\d\vec{S} = D \oint dS = 4\pi r^2 D; \quad
	\int \rho\,dV = q.
\]

Поле внутри шара, то есть при \( r < a \) и \( q = 0 \):
\[
	D = E = 0.
\]

Поле вне шара, то есть при \( r > a \) и \( q = Q \):
\[
	D = \frac{Q}{4\pi r^2}; \quad
	E = \eps\Ezero D = \frac{Q}{4\pi\Ezero\eps(r) \cdot r^2}.
\]

\vspace*{2em}   
\emph{Ответ:} \( \ds D = \frac{Q}{4\pi r^2}, \quad
E = \frac{Q}{4\pi\Ezero\eps(r) \cdot r^2} \).

\newpage
%-------------------------------------------------------------------------------
Задача 1.59: \emph{Вычислить емкость плоского конденсатора. Поверхность
обкладок \( S \), между ними два плоскопараллельных слоя однородных
диэлектриков. Толщина первого слоя \( d_1 \), проницаемость \( \eps_1 \),
второго -- соответственно \( d_2 \) и \( \eps_2 \). Краевым эффектом
пренебречь.}

\vspace*{2em}
\emph{Решение:}

По определению емкости плоского конденсатора, с учетом того, что поле
\( \vec{D} \) порождается только свободным поверхностным зарядом, т. е.
\( D = \sigma = \eps\Ezero E\):
\[
    C = \frac{q}{\Delta\phi} = \frac{q}{Ed} = \frac{\sigma S}{Ed} =
    \frac{\eps\Ezero ES}{Ed} = \frac{\eps\Ezero S}{d}.
\]

Для вычисления емкости двухслойного конденсатора представим его в виде
последовательного соединения двух однослойных, емкости которых:
\[
    C_1 = \frac{\eps_1\Ezero S}{d_1}, \quad C_2 = \frac{\eps_2\Ezero S}{d_2}.
\]

Тогда:
\[
    C = \frac{C_1C_2}{C_1 + C_2} = \frac{\eps_1\eps_2\Ezero^2 S^2}
    {S\Ezero d_1d_2\left(\frac{\eps_1}{d_1} + \frac{\eps_2}{d_2}\right)} =
    \frac{\Ezero\eps_1\eps_2 S}{\eps_1d_2 + \eps_2d_1}.
\]

\vspace*{2em}
\emph{Ответ:} \( C = \cfrac{\Ezero\eps_1\eps_2 S}{\eps_1d_2 + \eps_2d_1} \).

\newpage
%-------------------------------------------------------------------------------
Задача 1.116: \emph{Найти распределение зарядов, индуцированных на поверхности
цилиндра предыдущей задачи, и суммарный заряд, индуцированный на единице длины
цилиндра.}
\vspace*{-1em}

\singlespacing
{\footnotesize\emph{Предыдущая задача: В вакууме имеется бесконечно длинный заземленный
проводящий круглый цилиндр радиуса \( a \). Параллельно его оси протянута нить на
расстоянии \( l > a \) от нее. Нить равномерно заряжена с линейной плотностью
\( \chi \). Определить создаваемое ею поле и силу, действующую на единицу длины
нити.}}

\vspace*{1em}
\onehalfspacing
\emph{Решение:}

Из решения задачи 1.115:
\[
    E_R = -\pder{\phi}{R} = 2\chi \Biggl(\frac{R - l\cos\theta}{R_1^2} -
    \frac{R - l'\cos\theta}{R_2^2}\Biggr),
\]
где \( R_1 = \sqrt{l^2 + R^2 - 2lR\cos\theta} \), \( R_2 = \sqrt{l'^2 + R^2 -
2l'R\cos\theta} \), \( l' = a^2/l \).

Распределение заряда на поверхности определяется формулой:
\begin{gather*}
    \sigma = -\Ezero\pder{\phi}{R}\biggr|_{R = a} = \Ezero E_R\bigr|_{R = a} =
    2\chi\Ezero\Biggl(\frac{R - l\cos\theta}{l^2 + R^2 - 2lR\cos\theta} - \\ -
    \frac{R - a^2/l\cos\theta}{a^4/l^2 + R^2 - 2a^2R/l\cos\theta}\Biggr)_{R = a}
    = 2\chi\Ezero\Biggl(\frac{a - l\cos\theta}{l^2 + a^2 - 2al\cos\theta} - \\ -
    \frac{a - a^2/l\cos\theta}{a^4/l^2 + a^2 - 2a^3/l\cos\theta}\Biggr) =
    2\chi\Ezero\Biggl(\frac{a^2-al\cos\theta}{a(a^2 + l^2 - 2al\cos\theta)} - \\
    - \frac{l^2 - al\cos\theta}{a(a^2 + l^2 - 2al\cos\theta)}\Biggr) =
    2\chi\Ezero\frac{a^2 - l^2}{a(a^2 + l^2 - 2al\cos\theta)}.
\end{gather*}

Суммарный заряд, индуцированный на единице длины цилиндра, можно выразить через
поверхностную плотность зарядов следующим образом:
\[
    q_\emph{и} = \int \sigma\,dS = \int\limits_0^{2\pi} a\sigma\d\theta =
    2\chi\Ezero \int\limits_0^{2\pi} \frac{a^2 - l^2}{a^2 + l^2 -
    2al\cos\theta}\d\theta = 2\chi\Ezero\cdot (-2\pi) = -4\pi\Ezero\chi.
\]

\vspace*{2em}
\emph{Ответ:} \( \sigma = \cfrac{2\chi\Ezero(a^2 - l^2)}{a(a^2 + l^2 -
2al\cos\theta)} \), \( q_\emph{и} = -4\pi\Ezero\chi \).

\newpage
%-------------------------------------------------------------------------------
Задача 2.35: \emph{Полупространство заполнено однородным магнетиком с
проницаемостью \( \mu_1 \), а второе полупространство -- однородным магнетиком
с проницаемостью \( \mu_2 \). В первой среде имеется плоский контур \( L \) с
током \( I \), расположенный параллельно плоскости раздела обеих сред на
расстоянии \( h \) от нее. Определить создаваемое током магнитное поле.}

\vspace*{2em}
\emph{Решение:}

\emph{Ответ:}
\end{document}
