\documentclass[a4paper, 14pt]{extarticle}
\usepackage[utf8]{inputenc}
\usepackage[paper=a4paper, top=1cm, right=1cm, bottom=1.5cm, left=2cm]{geometry}
\usepackage{setspace}
\onehalfspacing

\usepackage{graphicx}
\graphicspath{{plots/}, {images/}}

\parindent=1.25cm

\usepackage{titlesec}

\titleformat{\section}
    {\normalsize\bfseries}
    {\thesection}
    {1em}{}

\titleformat{\subsection}
    {\normalsize\bfseries}
    {\thesubsection}
    {1em}{}

% Настройка вертикальных и горизонтальных отступов
\titlespacing*{\chapter}{0pt}{-30pt}{8pt}
\titlespacing*{\section}{\parindent}{*4}{*4}
\titlespacing*{\subsection}{\parindent}{*4}{*4}

\usepackage[square, numbers, sort&compress]{natbib}
\makeatletter
\bibliographystyle{unsrt}
\renewcommand{\@biblabel}[1]{#1.} 
\makeatother


\newcommand{\maketitlepage}[6]{
    \begin{titlepage}
        \singlespacing
        \newpage
        \begin{center}
            Министерство образования и науки Российской Федерации \\
            Федеральное государственное бюджетное образовательное \\
            учреждение высшего профессионального образования \\
            <<Волгоградский государственный технический университет>> \\
            #1 \\
            Кафедра #2
        \end{center}


        \vspace{14em}

        \begin{center}
            \large Семестровая работа #6 по дисциплине
            \\ <<#3>>
        \end{center}

        \vspace{5em}

        \begin{flushright}
            \begin{minipage}{.35\textwidth}
                Выполнила:\\#4
                \vspace{1em}\\
                Проверил:\\#5
                \\
                \\ Оценка \underline{\ \ \ \ \ \ \ \ \ \ \ \ \ \ \ \ }
            \end{minipage}
        \end{flushright}

        \vspace{\fill}

        \begin{center}
            Волгоград, \the\year
        \end{center}

    \end{titlepage}
    \setcounter{page}{2}
}

\newcommand{\maketitlepagewithvariant}[7]{
    \begin{titlepage}
        \singlespacing
        \newpage

        \begin{center}
            Министерство образования и науки Российской Федерации \\
            Федеральное государственное бюджетное образовательное \\
            учреждение высшего профессионального образования \\
            <<Волгоградский государственный технический университет>> \\
            #1 \\
            Кафедра #2
        \end{center}


        \vspace{8em}

        \begin{center}
            \large Семестровая работа #6 по дисциплине
            \\ <<#3>>
        \end{center}

        \vspace{1em}
        \begin{center}
            Вариант №#7
        \end{center}
        \vspace{4em}

        \begin{flushright}
            \begin{minipage}{.35\textwidth}
                Выполнила:\\#4
                \vspace{1em}\\
                Проверил:\\#5
                \\
                \\ Оценка \underline{\ \ \ \ \ \ \ \ \ \ \ \ \ \ \ \ }
            \end{minipage}
        \end{flushright}

        \vspace{\fill}

        \begin{center}
            Волгоград, \the\year
        \end{center}

    \end{titlepage}
    \setcounter{page}{2}
}

\input{../../.preambles/10-russian}
\input{../../.preambles/20-math}
\input{../../.preambles/30-physics}

\newcommand{\rot}{\mathrm{rot\,}}
\renewcommand{\vec}[1]{\mathbf{#1}}
\renewcommand{\mid}[1]{\bar{#1}}

\begin{document}
\maketitlepage{Факультет электроники и вычислительной техники}{физики}
{Электродинамика}{студент группы Ф-369\\Чечеткин~И.~А.}
{доцент Грецов~М.~В.}{№2}
\emph{698.} Определить закон движения частицы во взаимно перпендикулярных
однородных постоянных электрическом \( \vec{E} \) и магнитном \( \vec{H} \)
полях. Сделать это двумя способами:
\vspace*{-1em}
\begin{enumerate}\itemsep-.5em
    \item используя преобразование Лоренца и считая известным движение частицы в
    чисто электрическом или чисто магнитном поле;
    \item интегрируя следующие уравнения:
    \begin{align*}
        \der{p_i}{\tau} = eF^i, \\
        m\der{u_i}{\tau} = \frac{e}{c}F_{ik}u_k.
    \end{align*}
\end{enumerate}

\vspace*{2em}
\emph{Решение:}
\begin{enumerate}
    \item
    \item
\end{enumerate}
\vspace*{2em}        
\emph{Ответ:}
\newpage

%-------------------------------------------------------------------------------
\emph{732.} Найти электромагнитное поле \( \vec{H} \), \( \vec{E} \) заряда
\( e \), движущегося равномерно по окружности радиуса \( a \). Движение
нерелятивистское, угловая скорость \( \omega \). Расстояние до точки наблюдения
\( r \gg a \). Найти средние по времени угловое распределение \( \mid{\der{I}
{\Omega}} \) и полную интенсивность \( \mid{I} \) излучения, а также исследовать
его поляризацию.

\vspace*{2em}
\emph{Решение:}

\vspace*{2em}
\emph{Ответ:}
\newpage

%-------------------------------------------------------------------------------
\emph{761.} Заряд \( e \) движется с малой скоростью \( \vec{v} \) и ускорением
\( \vec{\dot{v}} \) в ограниченной области.  Определить угловое распределение
\( \der{I}{\Omega} \) и полное излучение \( I \).

\vspace*{2em}
\emph{Решение:}

\vspace*{2em}
\emph{Ответ:}
\end{document}
