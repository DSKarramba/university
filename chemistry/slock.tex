\input{../../.preambles/01-semester_work}
\input{../../.preambles/10-russian}
\input{../../.preambles/20-math}
\input{../../.preambles/30-physics}

\newcommand{\bi}[1]{\( _{#1} \)}
\renewcommand{\.}{\;\!}

\begin{document}

\maketitlepage{Химико-технологический факультет}{общей и неорганической химии}
{Общая и неорганическая химия}{}{14}{студентка группы Ф-369\\Слоква~В.~И.}{f}
{старший преподаватель\\Гаджиева~Н.~Х.}{f}

%-------------------------------------------------------------------------------

\emph{1. Масса 1 л кислорода равна 1,4 г. Какой объем кислорода расходуется при
сгорании 21 г магния, эквивалент которого равен 1\!/2?}

\vspace*{2em}
\emph{Решение:}

\vspace*{2em}
\emph{Ответ: }

\newpage %----------------------------------------------------------------------

\emph{2. Определить плотность по водороду газовой смеси, состоящей из аргона
объемом 56 л и азота объемом 28 л. Объемы газов приведены к нормальным
условиям.}

\vspace*{2em}
\emph{Решение:}

\vspace*{2em}
\emph{Ответ: }

\newpage %----------------------------------------------------------------------

\emph{3. Определите эквивалент и эквивалентную массу в соединениях:\\
Na\bi{2}\.H\.As\.O\bi{4}, Na\.H\bi{2}\.As\.O\bi{4}, Na\bi{3}\.As\.O\bi{4}.}

\vspace*{2em}
\emph{Решение:}

\vspace*{2em}
\emph{Ответ: }

\newpage %----------------------------------------------------------------------

\emph{4. К раствору, содержащему хлорид кальция Ca\.Cl\bi{2}, массой 4,5~г,
прилили раствор, содержащий фосфат натрия Na\bi{3}\.PO\bi{4}, массой 4,1~г.
Определите массу полученного осадка, если выход продукта составляет 88\%.}

\vspace*{2em}
\emph{Решение:}

\vspace*{2em}
\emph{Ответ: }

\newpage %----------------------------------------------------------------------

\emph{5. В воде массой 400 г растворили сероводород объемом 12~мл (н.у.).
Определите массовую долю сероводорода в растворе.}

\vspace*{2em}
\emph{Решение:}

\vspace*{2em}
\emph{Ответ: }

\newpage %----------------------------------------------------------------------

\emph{6. Какую массу раствора с массовой долей хлорида натрия 20\% необходимо
добавить к воде объемом 40~мл для получения раствора с массовой долей соли 6\%?}

\vspace*{2em}
\emph{Решение:}

\vspace*{2em}
\emph{Ответ: }

\newpage %----------------------------------------------------------------------

\emph{7. Напишите электронную формулу элемента тантала Ta. Укажите валентность
в нормальном и возбужденном состояниях. Рассчитайте значение суммарного спина.}

\vspace*{2em}
\emph{Решение:}

\vspace*{2em}
\emph{Ответ: }

\newpage %----------------------------------------------------------------------

\emph{8. Определите молярную, нормальную концентрации, титр, массовую долю,
мольные доли растворенного вещества и растворителя в 10~н~растворе
H\bi{2}\.SO\bi{4} (\( \rho \) = 1,29~г/мл).}

\vspace*{2em}
\emph{Решение:}

\vspace*{2em}
\emph{Ответ: }

\newpage %----------------------------------------------------------------------

\emph{9. Дипольные моменты гидридов C\.H\bi{4}, N\.H\bi{3}, H\bi{2}\.O, H\.Cl
равны соответственно 0; 1,44; 1,84; 1,061~Д. Какая из связей Э-Н более полярна?}

\vspace*{2em}
\emph{Решение:}

\vspace*{2em}
\emph{Ответ: }

\newpage %----------------------------------------------------------------------

\emph{10. По знаку \( \Delta G_{298} \) определите, какие из приведенных
оксидов можно восстановить водородом при стандартных условиях:
Al\bi{2}\.O\bi{3}, Zn\.O, Pb\.O?}

\vspace*{2em}
\emph{Решение:}

\vspace*{2em}
\emph{Ответ: }

\newpage %----------------------------------------------------------------------

\emph{11. Определите, как изменится скорость прямой и обратной реакции при
увеличении давления в системах в 3 раза:\\
а) H\bi{2\text{(г)}} + I\bi{2\text{(г)}} = 2H\.I\bi{\text{(г)}}, \qquad
б) H\bi{2}\.S\bi{\text{(г)}} = H\bi{2\text{(г)}} + S\bi{\text{(к)}}.\\
В какую сторону сместится равновесие при данном увеличении давления?}

\vspace*{2em}
\emph{Решение:}

\vspace*{2em}
\emph{Ответ: }

\newpage %----------------------------------------------------------------------

\emph{12. Вычислите давление пара раствора, содержащего при 20\( ^\circ C \)
0,62~моля сахара в 450~г воды. Давление водяного пара при этой температуре
равно 2332,75~Па.}

\vspace*{2em}
\emph{Решение:}

\vspace*{2em}
\emph{Ответ: }

\newpage %----------------------------------------------------------------------

\emph{13. При растворении 0,029~г неэлектролита в 100~г ацетона
(С\.Н\bi{3})\bi{2}\.С\.О, температура кипения последнего повысилась на
0,43\( ^\circ C \). Вычислите эбуллиоскопическую константу ацетона.}

\vspace*{2em}
\emph{Решение:}

\vspace*{2em}
\emph{Ответ: }

\newpage %----------------------------------------------------------------------

\emph{14. Вычислите концентрацию ионов водорода \( \bigr[H^+ \bigl] \) в 0,02~М
растворе муравьиной кислоты, если \( \alpha \) = 3,24\%.}

\vspace*{2em}
\emph{Решение:}

\vspace*{2em}
\emph{Ответ: }

\newpage %----------------------------------------------------------------------

\emph{15. Вычислите концентрацию ионов гидроксида в растворе, рН которого
10,25.}

\vspace*{2em}
\emph{Решение:}

\vspace*{2em}
\emph{Ответ: }

\newpage %----------------------------------------------------------------------

\emph{16. Используя метод электронного баланса, расставьте коэффициенты в
уравнении реакции, укажите окислитель и восстановитель:} \vspace*{-1em}
\[
    \textit{Hg\.S + H\.NO\bi{3} + H\.Cl = Hg\.Cl\bi{2} + H\bi{2}\.SO\bi{4}
    + N\.O + H\bi{2}\.O.}
\]

\vspace*{1em}
\emph{Решение:}

\vspace*{2em}
\emph{Ответ: }

\newpage %----------------------------------------------------------------------

\emph{17. При электролизе раствора некоторого металла выделилось 0,16~г его.
Процесс происходил при силе тока 1,8~А в течение 4,2~мин. Определите
эквивалентную массу металла.}

\vspace*{2em}
\emph{Решение:}

\vspace*{2em}
\emph{Ответ: }

\newpage %----------------------------------------------------------------------

\emph{18. Какие процессы будут протекать в гальваническом элементе\\
Al\;/\;Al\.Cl\bi{3}\;//\;Mn\.SO\bi{4}\;/\;Mn? Вычислите ЭДС этого элемента.}

\vspace*{2em}
\emph{Решение:}

\vspace*{2em}
\emph{Ответ: }

\end{document}
