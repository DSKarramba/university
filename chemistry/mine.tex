\documentclass[a4paper, 14pt]{extarticle}
\usepackage[utf8]{inputenc}
\usepackage[paper=a4paper, top=1cm, right=1cm, bottom=1.5cm, left=2cm]{geometry}
\usepackage{setspace}
\onehalfspacing

\usepackage{graphicx}
\graphicspath{{plots/}, {images/}}

\parindent=1.25cm

\usepackage{titlesec}

\titleformat{\section}
    {\normalsize\bfseries}
    {\thesection}
    {1em}{}

\titleformat{\subsection}
    {\normalsize\bfseries}
    {\thesubsection}
    {1em}{}

% Настройка вертикальных и горизонтальных отступов
\titlespacing*{\chapter}{0pt}{-30pt}{8pt}
\titlespacing*{\section}{\parindent}{*4}{*4}
\titlespacing*{\subsection}{\parindent}{*4}{*4}

\usepackage[square, numbers, sort&compress]{natbib}
\makeatletter
\bibliographystyle{unsrt}
\renewcommand{\@biblabel}[1]{#1.} 
\makeatother


\newcommand{\maketitlepage}[6]{
    \begin{titlepage}
        \singlespacing
        \newpage
        \begin{center}
            Министерство образования и науки Российской Федерации \\
            Федеральное государственное бюджетное образовательное \\
            учреждение высшего профессионального образования \\
            <<Волгоградский государственный технический университет>> \\
            #1 \\
            Кафедра #2
        \end{center}


        \vspace{14em}

        \begin{center}
            \large Семестровая работа #6 по дисциплине
            \\ <<#3>>
        \end{center}

        \vspace{5em}

        \begin{flushright}
            \begin{minipage}{.35\textwidth}
                Выполнила:\\#4
                \vspace{1em}\\
                Проверил:\\#5
                \\
                \\ Оценка \underline{\ \ \ \ \ \ \ \ \ \ \ \ \ \ \ \ }
            \end{minipage}
        \end{flushright}

        \vspace{\fill}

        \begin{center}
            Волгоград, \the\year
        \end{center}

    \end{titlepage}
    \setcounter{page}{2}
}

\newcommand{\maketitlepagewithvariant}[7]{
    \begin{titlepage}
        \singlespacing
        \newpage

        \begin{center}
            Министерство образования и науки Российской Федерации \\
            Федеральное государственное бюджетное образовательное \\
            учреждение высшего профессионального образования \\
            <<Волгоградский государственный технический университет>> \\
            #1 \\
            Кафедра #2
        \end{center}


        \vspace{8em}

        \begin{center}
            \large Семестровая работа #6 по дисциплине
            \\ <<#3>>
        \end{center}

        \vspace{1em}
        \begin{center}
            Вариант №#7
        \end{center}
        \vspace{4em}

        \begin{flushright}
            \begin{minipage}{.35\textwidth}
                Выполнила:\\#4
                \vspace{1em}\\
                Проверил:\\#5
                \\
                \\ Оценка \underline{\ \ \ \ \ \ \ \ \ \ \ \ \ \ \ \ }
            \end{minipage}
        \end{flushright}

        \vspace{\fill}

        \begin{center}
            Волгоград, \the\year
        \end{center}

    \end{titlepage}
    \setcounter{page}{2}
}

\input{../../.preambles/10-russian}
\input{../../.preambles/20-math}
\input{../../.preambles/30-physics}

\newcommand{\bi}[1]{\( _{#1} \)}
\renewcommand{\.}{\;\!}

\begin{document}

\maketitlepage{Химико-технологический факультет}{общей и неорганической химии}
{Общая и неорганическая химия}{}{15}{студент группы Ф-369\\Чечеткин~И.~А.}{m}
{старший преподаватель\\Гаджиева~Н.~Х.}{f}

%-------------------------------------------------------------------------------

\emph{1. Образец смеси оксида кальция и карбоната кальция массой 0,8~г
обрабатывали избытком раствора соляной кислоты, при этом выделился газ объемом
112~л (н.у.). Определите массовую долю оксида кальция в смеси.}

\vspace*{2em}
\emph{Решение:}

\vspace*{2em}
\emph{Ответ: }

\newpage %----------------------------------------------------------------------

\emph{2. Некоторое количество металла, эквивалентная масса которого
27,9~г/моль, вытесняет из кислоты 600~мл водорода, измеренного при н.у.
Определите массу металла.}

\vspace*{2em}
\emph{Решение:}

\vspace*{2em}
\emph{Ответ: }

\newpage %----------------------------------------------------------------------

\emph{3. Определите эквивалент и эквивалентную массу в соединениях:
K\bi{3}\.PO\bi{4}, K\.H\bi{2}\.PO\bi{4}.}

\vspace*{2em}
\emph{Решение:}

\vspace*{2em}
\emph{Ответ: }

\newpage %----------------------------------------------------------------------

\emph{4. При обработке 40~г смеси порошков алюминия и меди раствором едкого
натра получено 7,6~л водорода при нормальных условиях. Вычислите массовую долю
меди в смеси.}

\vspace*{2em}
\emph{Решение:}

\vspace*{2em}
\emph{Ответ: }

\newpage %----------------------------------------------------------------------

\emph{5. Какой объем 20\% (по массе) серной кислоты (\( \rho \) = 1,14 г/мл),
при взаимодействии с цинком, потребуется для получения 200~мл водорода (н.у.)?}

\vspace*{2em}
\emph{Решение:}

\vspace*{2em}
\emph{Ответ: }

\newpage %----------------------------------------------------------------------

\emph{6. Определите нормальную, молярную концентрации, титр, мольные доли
растворенного вещества и растворителя 18\% (по массе) раствора серной кислоты
(\( \rho \) = 1,08 г/мл).}

\vspace*{2em}
\emph{Решение:}

\vspace*{2em}
\emph{Ответ: }

\newpage %----------------------------------------------------------------------

\emph{7. Вычислите осмотическое давление раствора, содержащего при
\( 0^\circ C \) в 0,25~л раствора 2,8~г глицерина C\bi{3}\.H\bi{8}\.O\bi{3}}.

\vspace*{2em}
\emph{Решение:}

\vspace*{2em}
\emph{Ответ: }

\newpage %----------------------------------------------------------------------

\emph{8. Какая масса гидроксида натрия была взята, если при нейтрализации
гидроксида натрия хлороводородом по реакции:} \vspace*{-1em}
\[
    \textit{Na\.OH\bi{\text{(к)}} + H\.Cl\bi{\text{(г)}} = Na\.Cl\bi{\text{(к)}} +
    H\bi{2}\.O\bi{\text{(г)}}}
\]

\vspace*{-1em}
\emph{выделилось 352,9~кДж теплоты?}

\vspace*{2em}
\emph{Решение:}

\vspace*{2em}
\emph{Ответ: }

\newpage %----------------------------------------------------------------------

\emph{9. Реакция C\.O\bi{\text{(г)}} + Cl\bi{2\text{(г)}} =
C\.O\.Cl\bi{2\text{(г)}} протекает в объеме 20~л. Состав равновесной смеси:
0,28~г C\.O, 0,355~г Cl\bi{2}, 0,495~г C\.O\.Cl\bi{2}. Вычислите константу
равновесия реакции.}

\vspace*{2em}
\emph{Решение:}

\vspace*{2em}
\emph{Ответ: }

\newpage %----------------------------------------------------------------------

\emph{10. Определите возможность самопроизвольного протекания следующих
реакций: а) C\bi{\text{(к)}} + O\bi{2\text{(г)}} = C\.O\bi{2\text{(г)}}; \qquad
б) Na\bi{\text{(к)}} + 1\!/2\.Cl\bi{2\text{(г)}} = Na\.Cl\bi{\text{(к)}}.}

\vspace*{2em}
\emph{Решение:}

\vspace*{2em}
\emph{Ответ: }

\newpage %----------------------------------------------------------------------

\emph{11. Определите концентрацию \( \bigr[OH^-\bigl] \), если концентрация
\( \bigr[H^+\bigl] \) равна:\\
а) \( 10^{-3} \)~моль/л, б) \( 2\cdot 10^{-4} \)~моль/л.}

\vspace*{2em}
\emph{Решение:}

\vspace*{2em}
\emph{Ответ: }

\newpage %----------------------------------------------------------------------

\emph{12. Электролиз раствора соли привел к увеличению pH в катодном
пространстве. Какая из солей -- Ba\.Cl\bi{2}, Cu\.SO\bi{4} или Zn\.Cl\bi{2}
подверглась электролизу? Напишите электролиз выбранной соли.}

\vspace*{2em}
\emph{Решение:}

\vspace*{2em}
\emph{Ответ: }

\newpage %----------------------------------------------------------------------

\emph{13. При электролизе раствора Cu\.SO\bi{4} с медными электродами масса
катода увеличилась на 0,005~кг. Какое количество электричества было пропущено
через электролизер?}

\vspace*{2em}
\emph{Решение:}

\vspace*{2em}
\emph{Ответ: }

\newpage %----------------------------------------------------------------------

\emph{14. Напишите уравнения гидролиза в молекулярном и ионно-молекулярном виде
следующих солей: (NH\bi{4})\bi{2}\.S, (NH\bi{4})\bi{2}\.SO\bi{4},
(NH\bi{4})\bi{3}\.PO\bi{4}.}

\vspace*{2em}
\emph{Решение:}

\vspace*{2em}
\emph{Ответ: }

\newpage %----------------------------------------------------------------------

\emph{15. Какие процессы будут протекать при работе гальванического элемента
Ni\;/\;Ni\.SO\bi{4}\ (0,02~M)\;//\;Au\.Cl\bi{3}\;/\;Au? Вычислите ЭДС гальванического
элемента.}

\vspace*{2em}
\emph{Решение:}

\vspace*{2em}
\emph{Ответ: }

\newpage %----------------------------------------------------------------------

\emph{16. Используя метод электронного баланса, расставьте коэффициенты в
уравнении реакции, укажите окислитель и восстановитель:} \vspace*{-1em}
\[
    \textit{N\.O + H\bi{2}\.SO\bi{4} + Cr\.O\bi{2} = H\.NO\bi{3} +
    Cr\bi{2}\.(SO\bi{4})\bi{3} + H\bi{2}\.O}.
\]

\vspace*{1em}
\emph{Решение:}

\vspace*{2em}
\emph{Ответ: }

\newpage %----------------------------------------------------------------------

\emph{17. Напишите электронную формулу элемента лантана La. Укажите валентность
в нормальном и возбужденном состоянии. Вычислите значение суммарного спина.}

\vspace*{2em}
\emph{Решение:}

\vspace*{2em}
\emph{Ответ: }

\newpage %----------------------------------------------------------------------

\emph{18. Раствор, содержащий 33,2~г Ba\.(NO\bi{3})\bi{2} в 300~г воды, кипит
при \( 100,\!466^\circ C\). Вычислите степень диссоциации соли в растворе.}

\vspace*{2em}
\emph{Решение:}

\vspace*{2em}
\emph{Ответ: }

\end{document}
