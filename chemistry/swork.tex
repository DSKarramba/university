\documentclass[a4paper, 14pt]{extarticle}
\usepackage[utf8]{inputenc}
\usepackage[paper=a4paper, top=1cm, right=1cm, bottom=1.5cm, left=2cm]{geometry}
\usepackage{setspace}
\onehalfspacing

\usepackage{graphicx}
\graphicspath{{plots/}, {images/}}

\parindent=1.25cm

\usepackage{titlesec}

\titleformat{\section}
    {\normalsize\bfseries}
    {\thesection}
    {1em}{}

\titleformat{\subsection}
    {\normalsize\bfseries}
    {\thesubsection}
    {1em}{}

% Настройка вертикальных и горизонтальных отступов
\titlespacing*{\chapter}{0pt}{-30pt}{8pt}
\titlespacing*{\section}{\parindent}{*4}{*4}
\titlespacing*{\subsection}{\parindent}{*4}{*4}

\usepackage[square, numbers, sort&compress]{natbib}
\makeatletter
\bibliographystyle{unsrt}
\renewcommand{\@biblabel}[1]{#1.} 
\makeatother


\newcommand{\maketitlepage}[6]{
    \begin{titlepage}
        \singlespacing
        \newpage
        \begin{center}
            Министерство образования и науки Российской Федерации \\
            Федеральное государственное бюджетное образовательное \\
            учреждение высшего профессионального образования \\
            <<Волгоградский государственный технический университет>> \\
            #1 \\
            Кафедра #2
        \end{center}


        \vspace{14em}

        \begin{center}
            \large Семестровая работа #6 по дисциплине
            \\ <<#3>>
        \end{center}

        \vspace{5em}

        \begin{flushright}
            \begin{minipage}{.35\textwidth}
                Выполнила:\\#4
                \vspace{1em}\\
                Проверил:\\#5
                \\
                \\ Оценка \underline{\ \ \ \ \ \ \ \ \ \ \ \ \ \ \ \ }
            \end{minipage}
        \end{flushright}

        \vspace{\fill}

        \begin{center}
            Волгоград, \the\year
        \end{center}

    \end{titlepage}
    \setcounter{page}{2}
}

\newcommand{\maketitlepagewithvariant}[7]{
    \begin{titlepage}
        \singlespacing
        \newpage

        \begin{center}
            Министерство образования и науки Российской Федерации \\
            Федеральное государственное бюджетное образовательное \\
            учреждение высшего профессионального образования \\
            <<Волгоградский государственный технический университет>> \\
            #1 \\
            Кафедра #2
        \end{center}


        \vspace{8em}

        \begin{center}
            \large Семестровая работа #6 по дисциплине
            \\ <<#3>>
        \end{center}

        \vspace{1em}
        \begin{center}
            Вариант №#7
        \end{center}
        \vspace{4em}

        \begin{flushright}
            \begin{minipage}{.35\textwidth}
                Выполнила:\\#4
                \vspace{1em}\\
                Проверил:\\#5
                \\
                \\ Оценка \underline{\ \ \ \ \ \ \ \ \ \ \ \ \ \ \ \ }
            \end{minipage}
        \end{flushright}

        \vspace{\fill}

        \begin{center}
            Волгоград, \the\year
        \end{center}

    \end{titlepage}
    \setcounter{page}{2}
}

\input{../../.preambles/10-russian}
\input{../../.preambles/20-math}
\input{../../.preambles/30-physics}

\begin{document}

\maketitlepagewithvariant{Химико-технологический факультет}
{общая и неорганическая химия}{Общая и неорганическая химия}
{студент группы Ф-369\\Чечеткин~И.~А.}{старший преподаватель Гаджиева~Н.~Х.}
{\!\!}{15}

%-------------------------------------------------------------------------------

\emph{1. Образец смеси оксида кальция и карбоната кальция массой 0,8~г
обрабатывали избытком раствора соляной кислоты, при этом выделился газ объемом
112~л (н.у.). Определите массовую долю оксида кальция в смеси.}

\vspace*{2em}
\emph{Решение:}

\vspace*{2em}
\emph{Ответ: }

\newpage %----------------------------------------------------------------------

\emph{2. Некоторое количество металла, эквивалентная масса которого
27,9~г/моль, вытесняет из кислоты 600~мл водорода, измеренного при н.у.
Определите массу металла.}

\vspace*{2em}
\emph{Решение:}

\vspace*{2em}
\emph{Ответ: }

\newpage %----------------------------------------------------------------------

\emph{3. Определите эквивалент и эквивалентную массу в соединениях:
\( \mathrm{K_3PO_4} \), \( \mathrm{KH_2PO_4} \).}

\vspace*{2em}
\emph{Решение:}

\vspace*{2em}
\emph{Ответ: }

\newpage %----------------------------------------------------------------------

\emph{4. При обработке 40~г смеси порошков алюминия и меди раствором едкого
натра получено 7,6~л водорода при нормальных условиях. Вычислите массовую долю
меди в смеси.}

\vspace*{2em}
\emph{Решение:}

\vspace*{2em}
\emph{Ответ: }

\newpage %----------------------------------------------------------------------

\emph{5. Какой объем 20\% (по массе) серной кислоты (\( \rho = 1,14 \)~г/мл),
при взаимодействии с цинком, потребуется для получения 200~мл водорода (н.у.)?}

\vspace*{2em}
\emph{Решение:}

\vspace*{2em}
\emph{Ответ: }

\newpage %----------------------------------------------------------------------

\emph{6. Определите нормальную, молярную концентрации, титр, мольные доли
растворенного вещества и растворителя 18\% (по массе) раствора серной кислоты
(\( \rho = 1,08 \)~г/мл).}

\vspace*{2em}
\emph{Решение:}

\vspace*{2em}
\emph{Ответ: }

\newpage %----------------------------------------------------------------------

\emph{7. Вычислите осмотическое давление раствора, содержащего при
\( 0^\circ \)C в 0,25~л раствора 2,8~г глицерина \( \mathrm{C_3H_8O_3} \).}

\vspace*{2em}
\emph{Решение:}

\vspace*{2em}
\emph{Ответ: }

\newpage %----------------------------------------------------------------------

\emph{8. Какая масса гидроксида натрия была взята, если при нейтрализации
гидроксида натрия хлороводородом по реакции: \( \mathrm{NaOH_\emph{(к)} +
HCl_\emph{(г)} = NaCl_\emph{к} + H_2O_\emph{(г)}} \) выделилось 352,9~кДж
теплоты?}

\vspace*{2em}
\emph{Решение:}

\vspace*{2em}
\emph{Ответ: }

\newpage %----------------------------------------------------------------------

\emph{9. Реакция \( \mathrm{CO_\emph{(г)} + Cl_{2\emph{(г)}} =
COCl_{2\emph{(г)}}} \) протекает в объеме 20~л. Состав равновесной смеси:
0,28~г \( \mathrm{CO} \), 0,355~г \( \mathrm{Cl}_2 \), 0,495~г
\( \mathrm{COCl}_2 \). Вычислите константу равновесия реакции.}

\vspace*{2em}
\emph{Решение:}

\vspace*{2em}
\emph{Ответ: }

\newpage %----------------------------------------------------------------------

\emph{10. Определите возможность самопроизвольного протекания следующих
реакций: а) \( \mathrm{C_\emph{(к)} + O_{2\emph{(г)}} = CO_{2\emph{(г)}}} \);
б) \( \mathrm{Na_\emph{(к)} + 1/2Cl_{2\emph{(г)}} = NaCl_\emph{(к)}} \).}

\vspace*{2em}
\emph{Решение:}

\vspace*{2em}
\emph{Ответ: }

\newpage %----------------------------------------------------------------------

\emph{11. Определите концентрацию \( \left[\mathrm{OH}^-\right] \), если
концентрация \( \left[\mathrm{H}^+\right] \) равна:\\
а) \( 10^{-3} \)~моль/л, б) \( 2\cdot 10^{-4} \)~моль/л.}

\vspace*{2em}
\emph{Решение:}

\vspace*{2em}
\emph{Ответ: }

\newpage %----------------------------------------------------------------------

\emph{12. Электролиз раствора соли привел к увеличению pH в катодном
пространстве. Какая из солей -- \( \mathrm{BaCl}_2 \), \( \mathrm{CuSO}_4 \)
или \( \mathrm{ZnCl}_2 \) подверглась электролизу? Напишите электролиз
выбранной соли.}

\vspace*{2em}
\emph{Решение:}

\vspace*{2em}
\emph{Ответ: }

\newpage %----------------------------------------------------------------------

\emph{13. При электролизе раствора \( \mathrm{CuSO}_4 \) с медными электродами
масса катода увеличилась на 0,005~кг. Какое количество электричества было
пропущено через электролизер?}

\vspace*{2em}
\emph{Решение:}

\vspace*{2em}
\emph{Ответ: }

\newpage %----------------------------------------------------------------------

\emph{14. Напишите уравнения гидролиза в молекулярном и ионно-молекулярном виде
следующих солей: \( \mathrm{(NH_4)_2S} \), \( \mathrm{(NH_4)_2SO}_4 \),
\( \mathrm{(NH_4)_3PO}_4 \).}

\vspace*{2em}
\emph{Решение:}

\vspace*{2em}
\emph{Ответ: }

\newpage %----------------------------------------------------------------------

\emph{15. Какие процессы будут протекать при работе гальванического элемента
\( \mathrm{Ni/\ NiSO_4\ (0,02 M)//AuCl_3/Au} \)? Вычислите ЭДС гальванического
элемента.}

\vspace*{2em}
\emph{Решение:}

\vspace*{2em}
\emph{Ответ: }

\newpage %----------------------------------------------------------------------

\emph{16. Используя метод электронного баланса, расставьте коэффициенты в
уравнении реакции, укажите окислитель и восстановитель:}
\[
    \mathrm{NO + H_2SO_4  + CrO_2 = HNO_3 + Cr_2(SO_4)_3 + H_2O}.
\]

\vspace*{2em}
\emph{Решение:}

\vspace*{2em}
\emph{Ответ: }

\newpage %----------------------------------------------------------------------

\emph{17. Напишите электронную формулу элемента лантана \( \mathrm{La} \). Укажите
валентность в нормальном и возбужденном состоянии. Вычислите значение
суммарного спина.}

\vspace*{2em}
\emph{Решение:}

\vspace*{2em}
\emph{Ответ: }

\newpage %----------------------------------------------------------------------

\emph{18. Раствор, содержащий 33,2~г \( \mathrm{Ba(NO_3)_2} \) в 300 г воды,
кипит при \( 100,466^\circ \)С. Вычислите степень диссоциации соли в растворе.}

\vspace*{2em}
\emph{Решение:}

\vspace*{2em}
\emph{Ответ: }

\end{document}
