\part{Заключение}
В рамках магистерской диссертации был разработан метод кластеризации предпочтений жителей по перемещению, выраженных в форме пары геораспределенных объектов <<пункт отправления--пункт назначения>>. Для этого были использованы алгоритм кластеризации k-средних (подробно описан в разделе~\ref{sec:kmeans}) и метрика \emph{Route}, рассчитывающая расстояния между геораспределенными объектами по графу дорог, для чего использующая сервис маршрутизации Open Source Routing Machine (подробно описана в разделе~\ref{sec:distance}). Было сделано четыре реализации метода: последовательная, последовательная с использованием OSRM, параллельная и параллельная с использованием OSRM. В последовательной и параллельной реализациях в качестве метрики используется метрика \emph{Surface}, представляющая собой решение обратной геодезической задачи.

Для оценки эффективности работы предложенного метода было сгенерировано четыре тестовые выборки, представленные на рисунках~\ref{pic:full}, \ref{pic:railway-river}, \ref{pic:common}. Одна выборка используется для тестирования работы алгоритма на псевдореалистичных данных, две выборки используются для проверки метрикой учета препятствий, и последняя~--- для тестов алгоритма на скорость.

Полученные результаты, приведенные в разделе~\ref{sec:workresults} показали, что метрика с использованием OSRM позволяет учитывать наличие препятствий между объектами, в то время как вторая метрика этого делать не может. Однако, из таблицы~\ref{tab:results} видно, что время расчета расстояния метрикой \emph{Route} существенно (в 5-6 раз) больше времени, затрачиваемого метрикой \emph{Surface}, и в десятки раз больше времени, затрачиваемого евклидовой метрикой. Отсюда был сделан вывод, что предложенный метод может быть использован для кластеризации геораспределенных данных, если на рассматриваемой местности находятся какие-либо естественные или искуственные препятствия, или если время кластеризации не играет большой роли, а важен именно результат работы алгоритма. В разделах~\ref{sec:preconclusions} и~\ref{sec:conclusions} приведен более подробный анализ результатов работы.

Результаты данной работы используются для построения сети остановочных пунктов общественного транспорта, по которым будет проводится построение маршрутов общественного транспорта.
