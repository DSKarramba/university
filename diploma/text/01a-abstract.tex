% переработать тексты
\tocless\part{Аннотация}
Развитие городов сопряжено с модернизацией существующей сети общественного транспорта. Современные коммуникационные технологии позволяют собирать данные о перемещениях людей в городской среде, на основе которых можно сделать выводы о предпочтениях людей по перемещению внутри городской среды. Фактически, подобные предпочтения можно рассматривать как требования к структуре сети общественного транспорта. Имея данные о ежедневных передвижениях людей, мы можем понять реальные предпочтения и потребности людей в системе городского транспорта. Следовательно, может быть предложена модификация транспортной сети или даже набор возможных альтернатив. Эта модификация отражает реальные потребности людей и сокращает время передвижения и повышает общий уровень удовлетворенности. Чтобы произвести такие модификации необходимо сначала решить задачу кластеризации геораспределенных данных, учитывая при этом препятствия, лежащие между объектами данных, для получения сети предпочтительных остановочных пунктов общественного транспорта. В данной работе предлагаются методы, позволяющие на основе данных о предпочтениях жителей формировать схему остановочных пунктов общественного транспорта.

Список ключевых слов: геораспределенные данные, транспорт, общественный транспорт, остановочные пункты, кластеризация, кластеризация с учетом препятствий, кластеризация пространственных данных, k-means, анализ больших данных, поддержка принятия решений.

\tocless\part{Abstract}
Urban development is connected with the existing public transportation network modernization. Modern communication technologies make it possible to collect data on the movements of people in the urban environment, based on which is possible to draw conclusions about people's preferences on the movement inside the urban space. In fact, these preferences may be considered as requirements for the public transportation network. Having data about people's everyday movements, we can understand the real people preferences and needs in the urban transport system. Hence, as the results of analysis the modified transport network (or even a bunch of alternatives) can be suggested. This new solutions reflects real people needs and reduce transfer time and increase satisfaction level. To design the modification for transport network, we must first locate the preferable transport terminals. Actually, it's necessary to solve the geospatial clustering problem -- clustering data with condition of existence of obstacles between data samples.
The proposed methods allows to create a public transportation terminals network based on data about the transportation users preferences.

List of keywords: geospatial data, transport, public transport, transport terminals, clustering, clustering with obstructed distance, clustering geospatial data, big data analytics, decision support.