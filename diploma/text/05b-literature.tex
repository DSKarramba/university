\renewcommand{\bibname}{%
    \vspace{-1em}\begin{center}
        Список используемой литературы
    \end{center}\vspace{-2em}
}

\pagestyle{empty}

\begin{thebibliography}{99}
    \bibitem{bib:1} N. Sadovnikova, D. Parygin, E. Gnedkova, B. Sanzhapov, and N. Gidkova. 
        Evaluating the sustainability of Volgograd. In The Sustainable City VIII. WIT Press, 2013.
    \bibitem{bib:2} Нестерова А. Новая маршрутная сеть г. Томска представлена общественности 
        [Электронный ресурс] // Сетевое издание Центр дорожной информации. 2015. URL: 
        \url{http://road.perm.ru/index.php?id=1475} (дата обращения: 01.12.2015).
    \bibitem{bib:3} Nielsen G., Lange T. Network Design for Public Transport Success -- theory and 
        examples // Norwegian Ministry of Transport and Communications, Oslo. -- 2008.
    \bibitem{voron} Воронцов К.В. Машинное обучение. Курс лекций [Электронный ресурс].2011. URL: \url{http://www.machinelearning.ru/wiki/images/6/6d/Voron-ML-1.pdf} (дата обращения: 25.03.2016).
    \bibitem{fraley} Fraley C., Raftery A. E. Model-based clustering, discriminant analysis, and density estimation~// Journal of the American statistical Association.~--- 2002.~--- Т.~97.~--- №~458.~--- С.~611--631.
    \bibitem{elcan} Elkan C. Using the triangle inequality to accelerate k-means~// ICML.~--- 2003.~--- Т.~3.~--- С.~147--153.
    \bibitem{kanungo} Kanungo T. et al. An efficient k-means clustering algorithm: Analysis and implementation~// Pattern Analysis and Machine Intelligence, IEEE Transactions on.~--- 2002.~--- Т.~24.~--- №~7.~--- С.~881--892.
    \bibitem{likas} Likas A., Vlassis N., Verbeek J. J. The global k-means clustering algorithm~// Pattern recognition.~--- 2003.~--- Т.~36.~--- №~2.~--- С.~451--461.
    \bibitem{hybrid} Ткаченко О. М. и др. Метод кластеризации на основе последовательного запуска k-средних с усовершенствованным выбором кандидата на новую позицию вставки //Научные труды Винницкого национального технического университета.~--- 2012.~--- №~2.
    \bibitem{lenka} Лисин А. В., Файзуллин Р. Т. Применение метаэвристических алгоритмов к решению задач кластеризации методом k-средних~// Компьютерная оптика.~--- 2015.~--- Т.~39.~--- №~3.~--- С.~406--412.
    \bibitem{approxkm} Wang J. et al. Fast approximate k-means via cluster closures~//Multimedia Data Mining and Analytics.~--- Springer International Publishing, 2015.~--- С.~373--395.
    \bibitem{krishna} Krishna K., Murty M. N. Genetic K-means algorithm //Systems, Man, and Cybernetics, Part B: Cybernetics, IEEE Transactions on.~--- 1999.~--- Т.~29.~--- №~3.~--- С.~433--439.
    \bibitem{fastkm} Lai J. Z. C., Huang T. J., Liaw Y. C. A fast k-means clustering algorithm using cluster center displacement~// Pattern Recognition.~--- 2009.~--- Т.~42.~--- №~11.~--- С.~2551--2556.
    \bibitem{mrkd} Pelleg D., Moore A. Accelerating exact k-means algorithms with geometric reasoning~// Proceedings of the fifth ACM SIGKDD international conference on Knowledge discovery and data mining.~--- ACM, 1999.~--- С.~277--281.
    \bibitem{hussein} Hussein N. A Fast Greedy K-Means Algorithm. Master's Thesis. // University of Amsterdam, Amsterdam.~--- 2002.
    \bibitem{kpp} Arthur D., Vassilvitskii S. k-means++: The advantages of careful seeding~// Proceedings of the eighteenth annual ACM-SIAM symposium on Discrete algorithms.~--- Society for Industrial and Applied Mathematics, 2007.~--- С.~1027--1035.
    \bibitem{lenka8} Duman S., G\"{u}ven\c{c} U., Y\"{o}r\"{u}keren N. Gravitational Search Algorithm for Economic Dispatch with Valve-Point Effects.~// International Review of Electrical Engineering 2010.~--- Т.~5.~--- С.~2890--2895.
    \bibitem{nister} Nister D., Stewenius H. Scalable recognition with a vocabulary tree~// Computer vision and pattern recognition, 2006 IEEE computer society conference on.~--- IEEE, 2006.~--- Т.~2.~--- С.~2161--2168.
    \bibitem{philbin} philbin2007.pdf
    \bibitem{ukraine} Петров С. А. и др. Гібридний алгоритм кластер-аналізу для формування апріорного розбиття простору ознак на класи знань в системах дистанційного навчання.~// Вісник Вінницького політехнічного інституту.~--- №~4.~--- 2015.~--- С.~80--87.
    \bibitem{xmeans} Pelleg D., Moore A. W. X-means: Extending K-means with Efficient Estimation of the Number of Clusters~// Proceeding ICML '00 Proceedings of the Seventeenth International Conference on Machine Learning.~--- 2000.~--- Т.~1.~--- С.~727--734.
    \bibitem{numbers} Sugar C. A., James G. M. Finding the number of clusters in a dataset~// Journal of the American Statistical Association.~--- Т.~98.~--- №~463.~--- 2003.~--- С.~750--763.
    \bibitem{silhouettes} Rousseeuw P. J. Silhouettes: a graphical aid to the interpretation and validation of cluster analysis~// Journal of computational and applied mathematics.~--- 1987.~--- Т.~20.~--- С.~53--65.
    \bibitem{rescale} de Amorim R.~C., Hennig C. Recovering the number of clusters in data sets with noise features using feature rescaling factors~// Information Sciences.~--- 2015.~--- Т.~324.~--- С.~126--145.
    \bibitem{genetic} Llet\'{\i} R. et al. Selecting variables for k-means cluster analysis by using a genetic algorithm that optimises the silhouettes //Analytica Chimica Acta. – 2004. – Т. 515. – №. 1. – С. 87--100.
    \bibitem{FDBSCAN} Тханг В. В., Пантюхин Д. В., Галушкин А. И. Гибридный алгоритм кластеризации FastDBSCAN.~// Труды Московского Физико-Технического Института.~--- 2015.~--- Т.~7.~--- №~3.~--- С.~77--81.
    \bibitem{ms} Comaniciu D., Meer P. Mean shift: A robust approach toward feature space analysis~// Pattern Analysis and Machine Intelligence, IEEE Transactions on.~--- 2002.~--- Т.~24.~--- №~5.~--- С.~603--619.
    \bibitem{inc} Charikar M. et al. Incremental clustering and dynamic information retrieval~// SIAM Journal on Computing.~--- 2004.~--- Т.~33.~--- №~6.~--- С.~1417--1440.
    \bibitem{huang} Huang Z. A Fast Clustering Algorithm to Cluster Very Large Categorical Data Sets in Data Mining~// DMKD.~--- 1997.~--- 8~с.
    \bibitem{birch} Zhang T., Ramakrishnan R., Livny M. BIRCH: an efficient data clustering method for very large databases~// ACM Sigmod Record.~--- ACM, 1996.~--- Т.~25.~--- №~2.~--- С.~103--114.
    \bibitem{presence} Tung A. K. H., Hou J., Han J. Spatial clustering in the presence of obstacles~// Data Engineering, 2001. Proceedings 17th International Conference on.~--- IEEE, 2001.~--- С.~359--367.
    \bibitem{cod} Han J., Kamber, M., Tung, A. K. H. Spatial Clustering Methods in Data Mining: A Survey.~// Geographic Data Mining and Knowledge Discovery, Research Monographs in GIS.~--- 2001.~--- С.~201--231.
    \bibitem{koperski} Koperski K., Han J., Adhikary J. Mining knowledge in geographical data~// Communications of ACM.~--- 1998.~--- Т.~26.
    \bibitem{estivill} Estivill-Castro V., Lee I. Argument free clustering for large spatial point-data sets via boundary extraction from Delaunay Diagram~// Computers, Environment and urban systems.~--- 2002.~--- Т.~26.~--- №~4.~--- С.~315--334.
    \bibitem{obstacles} Estivill-Castro V., Lee I. Clustering with obstacles for geographical data mining~// ISPRS Journal of Photogrammetry and Remote Sensing.~--- 2004.~--- Т.~59.~--- №~1.~--- С.~21--34.
    \bibitem{geodesic} Karney C. F. F. Algorithms for geodesics~// Journal of Geodesy.~--- 2013.~--- Т.~87.~--- №~1.~--- С.~43--55.
    \bibitem{meanshift} Cheng Y. Mean shift, mode seeking, and clustering~// Pattern Analysis and Machine Intelligence, IEEE Transactions on.~--- 1995.~--- Т.~17.~--- №~8.~--- С.~790--799.
    \bibitem{RFFI} Golubev A. et al. Strategway: web solutions for building public transportation routes using big geodata analysis~// Proceedings of the 17th International Conference on Information Integration and Web-based Applications \& Services.~--- ACM, 2015.~--- С.~91.
    \bibitem{algms} Angelov P. Autonomous Learning Systems: From Data Streams to Knowledge in Real-time.~--- John Wiley \& Sons, 2013.~--- 273~с.
    \bibitem{dbscan-pos} Автоматическая обработка текстов на естественном языке и компьютерная лингвистика: учеб. пособие~/ Большакова Е.И., Клышинский Э.С., Ландэ Д.В.,
Носков А.А., Пескова О.В., Ягунова Е.В.~--- М.: МИЭМ, 2011.~--- 272~с.
    \bibitem{neiskiy} Нейский И. М. Классификация и сравнение методов кластеризации~// Интеллектуальные технологии и системы. Сборник учебно-методических работ и статей аспирантов и студентов.~--- М.: НОК <<CLAIM>>, 2006.~--- Выпуск 8.~--- С.~130--142.
    \bibitem{macqueen} MacQueen J. et al. Some methods for classification and analysis of multivariate observations~// Proceedings of the fifth Berkeley symposium on mathematical statistics and probability.~--- 1967.~--- Т.~1.~--- №~14.~--- С.~281--297.
    \bibitem{geographiclib} GeographicLib~--- a small set of C++ classes for converting between geographic, UTM, UPS, MGRS, and geocentric coordinates [Электронный ресурс]. 2015. URL: \url{http://geographiclib.sourceforge.net/} (дата обращения: 14.11.2015).
    \bibitem{OSRM} Open Source Routing Machine [Электронный ресурс]. 2014. URL: \url{http://project-osrm.org/} (дата обращения: 25.11.2014).
    \bibitem{OSM} The Free Wiki World Map~--- An openly licensed map of the world being created by volunteers using local knowledge, GPS tracks and donated sources [Электронный ресурс]. 2014. URL:\url{https://www.openstreetmap.org} (Дата обращения: 25.11.2014).
    \bibitem{scatter} Project "Scatter"~--- generate geospatial data. 2015. URL: \url{https://vstu-cad-stuff.github.io/scatter/} (Дата обращения: 15.03.2015).
        
%    \bibitem{bib:4} Holsapple, C. Decisions and Knowledge. Handbook on Decision Support Systems 1, 
%        (Cosgrove). [Электронный ресурс]. -- 2008. -- Режим доступа: 
%        \url{http://www.springerlink.com/index/g182q711470w2510.pdf}.
%    \bibitem{bib:5} Tennenhouse D. Proactive computing //
%        Communications of the ACM. -- 2000. -- Т. 43. -- №. 5. -- С. 43-50.
%    \bibitem{bib:6} PTV Visum [Электронный ресурс] // PTV Group. 2014. URL: 
%        \url{http://vision-traffic.ptvgroup.com/en-uk/products/ptv-visum/} 
%        (дата обращения: 16.11.2014).
%    \bibitem{bib:7} INRO [Электронный ресурс] // Emme. 2014. URL: 
%        \url{https://www.inrosoftware.com/en/products/emme/} (дата обращения: 11.11.2014).
%    \bibitem{bib:8} Cube [Электронный ресурс] // Citilabs. 2014. URL: 
%        \url{http://www.citilabs.com/software/cube/} (дата обращения: 23.11.2014).
%    \bibitem{aimsun} Transport Simulation Systems [Электронный ресурс] // Aimsun. URL: 
%        \url{https://www.aimsun.com/wp/} (дата обращения: 01.03.2016).
%    \bibitem{transims} Transims [Электронный ресурс] // NASA. URL: 
%        \url{https://code.google.com/archive/p/transims/} (дата обращения: 01.03.2016).
%    \bibitem{osrm} Open Source Routing Machine [Электронный ресурс]. 2015. URL: http://project-osrm.org/ 
%        (дата обращения: 25.11.2015).
%    \bibitem{bib:9} Ceder A. Designing public transport network and routes //
%        Advanced Modeling for Transit Operations and Service Planning. -- 2003. -- Т. 3. -- С. 59-91.
%    \bibitem{bib:17} Sadovnikova N. et al. Models and Methods for the Urban Transit System Research //
%        Creativity in Intelligent Technologies and Data Science. -- Springer International Publishing, 
%        2015. -- С. 488-499. -- (Ser. Communications in Computer and Information Science. Vol. 535)
%    \bibitem{bib:19} Гладков Л.А., Курейчик В.В., Курейчик В.М. Генетические алгоритмы / 
%        Под ред. В.М. Курейчика. -- 2-е изд., испр. и доп. -- М.: ФИЗМАТЛИТ, 2006. -- 320 с.
%    \bibitem{bib:20} Golubev A, Chechetkin I, Solnushkin K.S., Sadovnikova N., Parygin D., Shcherbakov M., 
%        Brebels A., Strategway: web solutions for building public transportation routes using big geodata 
%        analysis // Proceedings of The 17th International Conference on Information Integration and 
%        Web-based Applications \& Services (iiWAS2015) (December 11 - 13, 2015 Brussels, Belgium) 
%        ACM New York, New York pp. 665 - 668
%    \bibitem{bib:20.2} Садовникова Н.П., Щербаков М.В., Парыгин Д.С., Солнушкин К.С., Голубев А.В., 
%        Чечёткин И.А. Комплекс инструментов интеллектуального анализа данных strategway для поддержки 
%        принятия решений по управлению развитием инфраструктуры города / В сборнике: Развитие средних 
%        городов: замысел, модели, практика Материалы III Международной научно-практической конференции. 
%        Волгоград, 2015. С. 147-150
%    \bibitem{bib:21} Bast H. et al. Route planning in transportation networks //
%        arXiv preprint arXiv:1504.05140. -- 2015.
%    % полезная информация | дипломная работа
%    \bibitem{bib:22} Чалой Е. В., Шамрай Н. Б. Построение матрицы корреспонденций для транспортной 
%        сети г. Владивостока.
%    % почитать
%    \bibitem{bib:23} Гасников А. и др. (ред.). Введение в математическое моделирование транспортных 
%        потоков. -- Litres, 2015.
%    \bibitem{bib:24} Гасников А. В., Гасникова Е. В. О возможной динамике в модели расчета матрицы 
%        корреспонденций (А. Дж. Вильсона) //ТРУДЫ МФТИ. -- 2010. -- Т. 2. -- №. 4. -- С. 45.
%    \bibitem{bib:25} Werneck R. F. Public Transit Labeling //Experimental Algorithms: 
%        14th International Symposium, SEA 2015, Paris, France, June 29–July 1, 2015, 
%        Proceedings. -- Springer, 2015. -- Т. 9125. -- С. 273.
%    \bibitem{ceder2007} Ceder A. Public Transit Planning and Operation: Theory, Modeling and Practice. 2007.
%    \bibitem{rodeheffer2013symmetric} Rodeheffer T. L. The Symmetric Shortest-Path Table Routing 
%        Conjecture. -– 2013.
%    \bibitem{delling2014round} Delling D., Pajor T., Werneck R. F. Round-based public transit routing //
%        Transportation Science. -- 2014. -- Т. 49. -- №. 3. -- С. 591-604.
%    \bibitem{delling2015customizable} Delling D. et al. Customizable route planning in road networks //
%        Transportation Science. -- 2015.
%    \bibitem{delling2015public} Delling D. et al. Public transit labeling //Experimental Algorithms. -- 
%        Springer International Publishing, 2015. -- С. 273-285.
%    \bibitem{abraham2013alternative} Abraham I. et al. Alternative routes in road networks //Journal of 
%        Experimental Algorithmics (JEA). -- 2013. -- Т. 18. -- С. 1.3.
%    \bibitem{wei2012constructing} Wei L. Y., Zheng Y., Peng W. C. Constructing popular routes from 
%        uncertain trajectories //Proceedings of the 18th ACM SIGKDD international conference on 
%        Knowledge discovery and data mining. -- ACM, 2012. -- С. 195-203.
%    \bibitem{dwyer2009fast} Dwyer T., Nachmanson L. Fast edge-routing for large graphs //
%        Graph Drawing. -- Springer Berlin Heidelberg, 2009. -- С. 147-158.
%    \bibitem{bib:27} Wang Y., Zheng Y., Xue Y. Travel time estimation of a path using sparse 
%        trajectories //Proceedings of the 20th ACM SIGKDD international conference on Knowledge 
%        discovery and data mining. -- ACM, 2014. -- С. 25-34.
%    \bibitem{bib:28} Berlingerio M. et al. AllAboard: a System for Exploring Urban Mobility and 
%        Optimizing Public Transport Using Cellphone Data //Mobile Phone Data for Development, Analysis 
%        of mobile phone datasets for the development of Ivory Coast, 
%        viewed. -- 2014. -- Т. 9. -- С. 397-411.
%    \bibitem{bib:29} Krushel E. G. et al. An Experience of Optimization Approach Application to Improve 
%        the Urban Passenger Transport Structure //Knowledge-Based Software Engineering. -- Springer 
%        International Publishing, 2014. -- С. 27-39.
%    \bibitem{bib:30} Mees P. et al. Public transport network planning: a guide to best practice in NZ 
%        cities. -- 2010. -- №. 396.
%    \bibitem{bib:31} Гузенко А. В. Развитие городского пассажирского транспорта мегаполиса: проблемы 
%        и перспективы //Вестник Томского государственного университета. -- 2009. -- №. 321.
%    \bibitem{bib:32} Кузьмич С. И., Федина Т. О. Транспортные проблемы современных городов и 
%        моделирование загрузки улично-дорожной сети //Известия Тульского государственного 
%        университета. Технические науки. -- 2008. -- №. 3.
%    \bibitem{bib:34} Шуравина Е. Н. Проблемы современной транспортной системы россии //
%        Вестник Самарского государственного университета. -- 2011. -- №. 90.
%    \bibitem{bib:39} Хегай Ю. А. Проблемы автомобильного транспорта в россии //Теория и практика 
%        общественного развития. -- 2014. -- №. 8.
%    \bibitem{bib:40} Синицына Е. Б., Лазарев Ю. Г. Современное состояние проблемы совершенствования 
%        транспортной инфраструктуры //Технико-технологические проблемы сервиса. -- 2013. -- №. 4 (26).
%    \bibitem{bib:33} Андрианов В. Ю. Геоинформационные системы для транспорта и коммуникаций //
%        T-Comm-Телекоммуникации и Транспорт. -- 2010. -- №. S2.
%    \bibitem{bib:35} Корягин М. Е. Теоретические аспекты оптимизации управления движением городского 
%        транспорта //Вестник Кузбасского государственного технического университета. -- 2012. -- №. 1 (89).
%    \bibitem{bib:36} Кочегурова Е. А., Мартынова Ю. А. Оптимизация составления маршрутов общественного 
%        транспорта при создании автоматизированной системы поддержки принятия решений //
%        Известия Томского политехнического университета. -- 2013. -- Т. 323. -- №. 5.
%    \bibitem{bib:37} Агуреев И. Е., Митюгин В. А., Пышный В. А. Подготовка и обработка исходных данных 
%        для математического моделирования автомобильных транспортных систем //Известия Тульского 
%        государственного университета. Технические науки. -- 2014. -- №. 6.
%    \bibitem{bib:38} Ефимова Е. А. Сравнительный анализ создания имитационной модели пропускной 
%        способности городской транспортной сети //Известия высших учебных заведений. Поволжский регион. 
%        Технические науки. -- 2009. -- №. 1.
%    \bibitem{bib:41} Денисов М. В., Агуреев И. Е. Математическое описание динамики пассажирских 
%        транспортных систем //Известия Тульского государственного университета. 
%        Технические науки. -- 2010. -- №. 4-2.
%    \bibitem{bib:42} Палант А. Ю. Обзор Методов Обследования Пассажиропотоков //
%        Бизнес Информ. -- 2014. -- №. 11.
%    \bibitem{bib:43} Чернов В. П., Кабалина Т. В. Исследование оценки качества в системе критериев 
%        эффективности перевозок пассажиров //Актуальные проблемы экономики и права. -- 2010. -- №. 4 (16).
%    \bibitem{bib:44} Лойко В. И., Параскевов А. В. Меры по обеспечению эффективной организации городского 
%        дорожного движения //Политематический сетевой электронный научный журнал Кубанского 
%        государственного аграрного университета. -- 2010. -- №. 64.
%    \bibitem{bib:45} Кочетов Ю. А. Методы локального поиска для дискретных задач размещения //
%        Специальность 05.13. 18 математическое моделирование, численные методы и 
%        комплексы программ. -- 2011.
%    \bibitem{bib:46} Блох И. И., Дураков А. В. Алгоритмы построения маршрута на карте по параметрам.
%    \bibitem{bib:47} Дасгупта С., Пападимитриу Х., Вазирани У. Алгоритмы //М.: МНЦМО. -- 2014.
%    \bibitem{bib:48} Шербина О. А. Метаэвристические алгоритмы для задач комбинаторной оптимизации (ОБЗОР).
%        [Электронный ресурс]. -- Режим доступа: \url{http://tvim.info/files/56\_72\_Shcherbina.pdf} 
%        (дата обращения: 01.08.2015).
%    \bibitem{bib:50} Ипатов А. В. Модифицированный метод имитации отжига в задаче маршрутизации 
%        транспорта //Труды Института математики и механики УрО РАН. -- 2011. -- Т. 17. -- №. 4. -- С. 121-125.
%    \bibitem{bib:51} Ипатов А. В. Решение задачи маршрутизации транспорта методом имитации отжига //
%        Проблемы теорет. и прикл. математики: тр. -- С. 290-294.
%    \bibitem{bib:52} Карпенко А. П. Современные алгоритмы поисковой оптимизации //Алгоритмы, 
%        вдохновленные природой: учеб. пособие. М.: Изд-во МГТУ им. НЭ Баумана. -- 2014.
%    \bibitem{bib:53} Ковалев М. Я. Теория алгоритмов. Курс лекций: в 2 ч //Минск: БГУ. -- 2003.
%    \bibitem{bib:54} Кочетов Ю. А. Вероятностные методы локального поиска для задач дискретной 
%        оптимизации //Дискретная математика и ее приложения. Сборник лекций молодежных и научных школ 
%        по дискретной математике и ее приложениям. М: МГУ. -- 2001. -- С. 87-117.
%    \bibitem{bib:55} Кулаков Ю. А., Воротников В. В. Формирование оптимальных маршрутов в мобильных сетях 
%        на основе модифицированного алгоритма Дейкстры //Вестник НТУУ <<КПИ>>: Информатика, управление 
%        и вычислительная техника. -- 2012. -- Т. 2012. -- №. 56.
%    \bibitem{bib:56} Романовский И. В. Алгоритмы решения экстремальных задач. -- Наука, 1977.
%    \bibitem{bib:57} Кочетов Ю. А., Младенович Н., Хансен П. Локальный поиск с чередующимися окрестностями //
%        Дискретный анализ и исследование операций. -- 2003. -- Т. 10. -- №. 1. -- С. 11-43.

    % прочие источники
    % https://mipt.ru/education/chair/computational_mathematics/upload/22b/Book-arpglktefbb.pdf
    % http://zoneos.com/traffic/
    % http://www.mou.mipt.ru/gasnikov1129.pdf
    % Швецов http://www.isa.ru/transnet/TrafficReview.pdf
    % http://spkurdyumov.ru/uploads/2013/08/Semenov.pdf
\end{thebibliography}

\pagestyle{plain}