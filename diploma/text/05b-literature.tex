\renewcommand{\bibname}{%
    \vspace{-.9em}\begin{center}
        Список используемой литературы
    \end{center}\vspace{-2em}
}
\clearpage
\phantomsection
\addcontentsline{toc}{chapter}{Список использованных источников}

\pagestyle{empty}

\begin{thebibliography}{99}
    \bibitem{bib:1} Evaluating the sustainability of Volgograd~/ N. Sadovnikova, D. Parygin, E. Gnedkova, B. Sanzhapov, and N. Gidkova.~// In The Sustainable City VIII. WIT Press, 2013.
    \bibitem{bib:2} Нестерова А. Новая маршрутная сеть г. Томска представлена общественности 
        [Электронный ресурс] // Сетевое издание Центр дорожной информации.~--- 2015.~--- Режим доступа: 
        \url{http://road.perm.ru/index.php?id=1475} (дата обращения: 01.12.2015).
    \bibitem{bib:3} Nielsen G., Lange T. Network Design for Public Transport Success -- theory and 
        examples // Norwegian Ministry of Transport and Communications, Oslo. -- 2008.
    \bibitem{voron} Воронцов К.В. Машинное обучение. Курс лекций [Электронный ресурс].~--- 2011.~--- Режим доступа: \url{http://www.machinelearning.ru/wiki/images/6/6d/Voron-ML-1.pdf} (дата обращения: 25.03.2016).
    \bibitem{fraley} Fraley C., Raftery A. E. Model-based clustering, discriminant analysis, and density estimation~// Journal of the American statistical Association.~--- 2002.~--- V.~97.~--- N.~458.~--- P.~611--631.
    \bibitem{elcan} Elkan C. Using the triangle inequality to accelerate k-means~// ICML.~--- 2003.~--- V.~3.~--- P.~147--153.
    \bibitem{kanungo} Kanungo T. et al. An efficient k-means clustering algorithm: Analysis and implementation~// Pattern Analysis and Machine Intelligence, IEEE Transactions on.~--- 2002.~--- V.~24.~--- N.~7.~--- P.~881--892.
    \bibitem{likas} Likas A., Vlassis N., Verbeek J. J. The global k-means clustering algorithm~// Pattern recognition.~--- 2003.~--- V.~36.~--- N.~2.~--- P.~451--461.
    \bibitem{hybrid} Ткаченко О. М. и др. Метод кластеризации на основе последовательного запуска k-средних с усовершенствованным выбором кандидата на новую позицию вставки //Научные труды Винницкого национального технического университета.~--- 2012.~--- №~2.
    \bibitem{lenka} Лисин А. В., Файзуллин Р. Т. Применение метаэвристических алгоритмов к решению задач кластеризации методом k-средних~// Компьютерная оптика.~--- 2015.~--- Т.~39.~--- №~3.~--- С.~406--412.
    \bibitem{approxkm} Wang J. et al. Fast approximate k-means via cluster closures~//Multimedia Data Mining and Analytics.~--- Springer International Publishing, 2015.~--- P.~373--395.
    \bibitem{krishna} Krishna K., Murty M. N. Genetic K-means algorithm //Systems, Man, and Cybernetics, Part B: Cybernetics, IEEE Transactions on.~--- 1999.~--- V.~29.~--- N.~3.~--- P.~433--439.
    \bibitem{fastkm} Lai J. Z. C., Huang T. J., Liaw Y. C. A fast k-means clustering algorithm using cluster center displacement~// Pattern Recognition.~--- 2009.~--- V.~42.~--- N.~11.~--- P.~2551--2556.
    \bibitem{mrkd} Pelleg D., Moore A. Accelerating exact k-means algorithms with geometric reasoning~// Proceedings of the fifth ACM SIGKDD international conference on Knowledge discovery and data mining.~--- ACM, 1999.~--- P.~277--281.
    \bibitem{hussein} Hussein N. A Fast Greedy K-Means Algorithm. Master's Thesis. // University of Amsterdam, Amsterdam.~--- 2002.
    \bibitem{kpp} Arthur D., Vassilvitskii S. k-means++: The advantages of careful seeding~// Proceedings of the eighteenth annual ACM-SIAM symposium on Discrete algorithms.~--- Society for Industrial and Applied Mathematics, 2007.~--- P.~1027--1035.
    \bibitem{lenka8} Duman S., G\"{u}ven\c{c} U., Y\"{o}r\"{u}keren N. Gravitational Search Algorithm for Economic Dispatch with Valve-Point Effects.~// International Review of Electrical Engineering 2010.~--- V.~5.~--- P.~2890--2895.
    \bibitem{nister} Nister D., Stewenius H. Scalable recognition with a vocabulary tree~// Computer vision and pattern recognition, 2006 IEEE computer society conference on.~--- IEEE, 2006.~--- V.~2.~--- P.~2161--2168.
    \bibitem{philbin} Philbin J. et al. Object retrieval with large vocabularies and fast spatial matching~// Computer Vision and Pattern Recognition, 2007. CVPR'07. IEEE Conference on.~--- IEEE, 2007.~--- P. 1-8.
    \bibitem{ukraine} Петров С. А. и др. Гібридний алгоритм кластер-аналізу для формування апріорного розбиття простору ознак на класи знань в системах дистанційного навчання.~// Вісник Вінницького політехнічного інституту.~--- №~4.~--- 2015.~--- С.~80--87.
    \bibitem{xmeans} Pelleg D., Moore A. W. X-means: Extending K-means with Efficient Estimation of the Number of Clusters~// Proceeding ICML '00 Proceedings of the Seventeenth International Conference on Machine Learning.~--- 2000.~--- V.~1.~--- P.~727--734.
    \bibitem{numbers} Sugar C. A., James G. M. Finding the number of clusters in a dataset~// Journal of the American Statistical Association.~--- V.~98.~--- N.~463.~--- 2003.~--- P.~750--763.
    \bibitem{silhouettes} Rousseeuw P. J. Silhouettes: a graphical aid to the interpretation and validation of cluster analysis~// Journal of computational and applied mathematics.~--- 1987.~--- V.~20.~--- P.~53--65.
    \bibitem{rescale} de Amorim R.~C., Hennig C. Recovering the number of clusters in data sets with noise features using feature rescaling factors~// Information Sciences.~--- 2015.~--- V.~324.~--- P.~126--145.
    \bibitem{genetic} Llet\'{\i} R. et al. Selecting variables for k-means cluster analysis by using a genetic algorithm that optimises the silhouettes~// Analytica Chimica Acta.~--- 2004.~--- V.~515.~--- N.~1.~--- P.~87--100.
    \bibitem{FDBSCAN} Тханг В. В., Пантюхин Д. В., Галушкин А. И. Гибридный алгоритм кластеризации FastDBSCAN.~// Труды Московского Физико-Технического Института.~--- 2015.~--- Т.~7.~--- №~3.~--- С.~77--81.
    \bibitem{ms} Comaniciu D., Meer P. Mean shift: A robust approach toward feature space analysis~// Pattern Analysis and Machine Intelligence, IEEE Transactions on.~--- 2002.~--- V.~24.~--- N.~5.~--- P.~603--619.
    \bibitem{inc} Charikar M. et al. Incremental clustering and dynamic information retrieval~// SIAM Journal on Computing.~--- 2004.~--- V.~33.~--- N.~6.~--- P.~1417--1440.
    \bibitem{huang} Huang Z. A Fast Clustering Algorithm to Cluster Very Large Categorical Data Sets in Data Mining~// DMKD.~--- 1997.~--- 8~p.
    \bibitem{birch} Zhang T., Ramakrishnan R., Livny M. BIRCH: an efficient data clustering method for very large databases~// ACM Sigmod Record.~--- ACM, 1996.~--- V.~25.~--- N.~2.~--- P.~103--114.
    \bibitem{presence} Tung A. K. H., Hou J., Han J. Spatial clustering in the presence of obstacles~// Data Engineering, 2001. Proceedings 17th International Conference on.~--- IEEE, 2001.~--- P.~359--367.
    \bibitem{cod} Han J., Kamber, M., Tung, A. K. H. Spatial Clustering Methods in Data Mining: A Survey.~// Geographic Data Mining and Knowledge Discovery, Research Monographs in GIS.~--- 2001.~--- P.~201--231.
    \bibitem{koperski} Koperski K., Han J., Adhikary J. Mining knowledge in geographical data~// Communications of ACM.~--- 1998.~--- V.~26.
    \bibitem{estivill} Estivill-Castro V., Lee I. Argument free clustering for large spatial point-data sets via boundary extraction from Delaunay Diagram~// Computers, Environment and urban systems.~--- 2002.~--- V.~26.~--- N.~4.~--- P.~315--334.
    \bibitem{obstacles} Estivill-Castro V., Lee I. Clustering with obstacles for geographical data mining~// ISPRS Journal of Photogrammetry and Remote Sensing.~--- 2004.~--- V.~59.~--- N.~1.~--- P.~21--34.
    \bibitem{geodesic} Karney C. F. F. Algorithms for geodesics~// Journal of Geodesy.~--- 2013.~--- V.~87.~--- N.~1.~--- P.~43--55.
    \bibitem{meanshift} Cheng Y. Mean shift, mode seeking, and clustering~// Pattern Analysis and Machine Intelligence, IEEE Transactions on.~--- 1995.~--- V.~17.~--- N.~8.~--- P.~790--799.
    \bibitem{RFFI} Golubev A. et al. Strategway: web solutions for building public transportation routes using big geodata analysis~// Proceedings of the 17th International Conference on Information Integration and Web-based Applications \& Services.~--- ACM, 2015.~--- P.~91.
    \bibitem{algms} Angelov P. Autonomous Learning Systems: From Data Streams to Knowledge in Real-time.~--- John Wiley \& Sons, 2013.~--- 273~p.
    \bibitem{dbscan-pos} Автоматическая обработка текстов на естественном языке и компьютерная лингвистика: учеб. пособие~/ Большакова Е.И., Клышинский Э.С., Ландэ Д.В.,
Носков А.А., Пескова О.В., Ягунова Е.В.~--- М.: МИЭМ, 2011.~--- 272~с.
    \bibitem{neiskiy} Нейский И. М. Классификация и сравнение методов кластеризации~// Интеллектуальные технологии и системы. Сборник учебно-методических работ и статей аспирантов и студентов.~--- М.: НОК <<CLAIM>>, 2006.~--- Выпуск 8.~--- С.~130--142.
    \bibitem{macqueen} MacQueen J. et al. Some methods for classification and analysis of multivariate observations~// Proceedings of the fifth Berkeley symposium on mathematical statistics and probability.~--- 1967.~--- V.~1.~--- N.~14.~--- P.~281--297.
    \bibitem{geographiclib} GeographicLib~--- a small set of C++ classes for converting between geographic, UTM, UPS, MGRS, and geocentric coordinates [Электронный ресурс].~--- 2015.~--- Режим доступа: \url{http://geographiclib.sourceforge.net/} (дата обращения: 14.11.2015).
    \bibitem{OSRM} Open Source Routing Machine [Электронный ресурс].~--- 2014.~--- Режим доступа: \url{http://project-osrm.org/} (дата обращения: 25.11.2014).
    \bibitem{OSM} The Free Wiki World Map~--- An openly licensed map of the world being created by volunteers using local knowledge, GPS tracks and donated sources [Электронный ресурс].~--- 2014.~--- Режим доступа:\url{https://www.openstreetmap.org} (Дата обращения: 25.11.2014).
    \bibitem{sklearn} scikit-learn. Machine Learning in Python. Simple and efficient tools for data mining and data analysis [Электронный ресурс].~--- 2015.~--- Режим доступа: \url{http://scikit-learn.org/} (дата обращения: 17.03.2015).
    \bibitem{mirkes} E.M. Mirkes, K-means and K-medoids applet [Электронный ресурс].~--- University of Leicester, 2011.~--- Режим доступа: \url{http://www.math.le.ac.uk/people/ag153/homepage/KmeansKmedoids/Kmeans_Kmedoids.html} (дата обращения: 17.03.2015).
    \bibitem{wiki} k-means clustering~--- Wikipedia, the free encyclopedia [Электронный ресурс].~--- 2014.~--- Режим доступа: \url{https://en.wikipedia.org/wiki/K-means_clustering} (дата обращения: 25.11.2014).
    \bibitem{scatter} Project "Scatter"~--- generate geospatial data [Электронный ресурс].~--- 2015.~--- Режим доступа: \url{https://vstu-cad-stuff.github.io/scatter/} (Дата обращения: 15.03.2015).
\end{thebibliography}

\pagestyle{plain}