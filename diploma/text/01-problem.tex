\chapter{Введение в проблему кластеризации геораспределенных данных}

Развитие транспортной инфраструктуры города основывается на устаревших нормативах, не учитывающих стремительное увеличение личного\linebreak транспорта и изменений функционального назначения городских пространств \cite{bib:1}. Это ведет к ухудшению транспортной ситуации, увеличению пробок, снижению качества транспортного обслуживания и, как следствие, к усилению неудовлетворенности жителей. Критичным фактором является игнорирование фактических предпочтений жителей по перемещению в городе при проектировании маршрутов общественного транспорта.

Для оптимизации маршрутной сети общественного транспорта необходимо проанализировать большой объем данных, характеризующих численность и мобильность населения, среднее время перемещения, расположение мест приложения труда и жилых массивов. Источниками этих данных выступают статистические сборники, выписки о численности сотрудников крупных предприятий, собираемые муниципальными предприятиями общественного транспорта, информация о количестве проданных билетах на маршрутах общественного транспорта. Для сбора данных о перемещениях жителей организуется целый комплекс мероприятий по натурному подсчету пассажиропотока в подвижном составе общественного транспорта и на остановочных пунктах существующих маршрутов, а также анкетированию жителей \cite{bib:2, bib:3}. Такие традиционные методы являются достаточно трудоемкими, а полученные данные не в полной мере отражают динамично меняющуюся ситуацию. В связи с этим, необходимо использовать современные технологии и новые ресурсы для получения актуальных данных о предпочтениях жителей по перемещениям в городе и интенсивности пассажиропотоков. Основываясь на современных подходах к анализу данных можно получить ценную информацию для поддержки принятия решений в процессе планирования развития транспортной системы города.

\section{Анализ предметной области}
Развитие городов приводит к модернизации существующей сети общественного транспорта. Управление городской инфраструктурой или городское планирование представляет собой набор сложных процедур, где решения могут оказывать сильное влияние на удовлетворенность жителей, в том числе~--- не удовлетворить всех. Новой проблемой для быстрорастущих городов становится создание надежной системы общественного транспорта. В идеале, оптимальная сеть городского общественного транспорта уменьшает время передвижения настолько, как если бы жители использовали личный транспорт.

Часто текущие системы планирования транспортной сети не принимают во внимание реальные предпочтения большинства жителей города. Это приводит к основным проблемам: общественным транспортом становится неудобно пользоваться, жители чаще начинают использовать личный транспорт для перемещения, из-за большего количества машин на дорогах начинают образовываться пробки, что приводит потере личного времени и, как следствие, к повышению недовольства жителей.

В настоящее время повсеместный сбор данных и технологии обработки данных, которые становятся все более доступными, открывают новые возможности для проектировщиков систем общественного транспорта. Большая часть населения, живущего в городах, имеют мобильные телефоны или смартфоны. Операторы мобильной связи могут получать огромное количество данных о перемещении владельцев телефонов по городу. Исходя из предположения, что каждый человек имеет свои определенные <<шаблоны перемещения>>~--- набор часто используемых пунктов отправления–назначения, которые можно рассматривать как личные предпочтения по перемещению в городе, эти данные можно использовать для оценки текущей транспортной сети и ее модификации на основе предпочтений жителей.

Для решения данных проблем необходимо выполнить следующие процедуры: сбор данных, обработка данных, создание набора маршрутных сетей и выбор сети из набора в соответствии с критериями качества.

Сбор данных требует методов для получения и предварительной обработки для дальнейшего эффективного хранения. Основным моментом для сбора данных является обеспечение качества данных, устранения пробелов в них.

Поскольку получение данных происходит от каждого отдельного человека, то для выявления остановочных пунктов общественного транспорта необходимо выявить центры скоплений пунктов отправления и назначения. Эта задача является задачей кластеризации равномерно распределенных данных. Поскольку элементы данных являются географическими объектами, то при кластеризации необходимо учитывать естественные и искусственные препятствия~--- реки или железные дороги.

Следующим шагом является объединение центров кластеров для создания маршрутов сети общественного транспорта. Эта проблема лежит на пересечении проблемы поиска кратчайшего пути и задачи выбора маршрута транспортного средства со входным параметром пассажиропотока. В отличие от задач поиска кратчайшего пути с функцией затрат, на данном этапе рассматривается функция стоимости, включающая в себя время пешего хода, время, проведенное в пути, и другие изменения. Конечные точки маршрутов и остановки, через которые проложен маршрут, являются центрами кластеров, определенных на предыдущем шаге.

На последнем шаге происходит отбор самой оптимальной сети маршрутов по заданному набору критериев качества.

Формально, данную задачу можно разбить на две подзадачи: сбор данных и их кластеризация и построение оптимальной сети маршрутов по полученным центрам кластеров.

Для кластеризации геораспределенных данных было проанализировано множество алгоритмов, некоторые из которых подробно описаны во второй главе: алгоритмы среднего сдвига (Mean Shift), k-средних (k-means), иерархический метод BIRCH, алгоритмы DBSCAN и OPTICS. В качестве метрик расстояний между точками были рассмотрены евклидова метрика, решение обратной геодезической задачи и расчет расстояний по городским дорогам с при помощи сервиса построения маршрутов Open Source Routing Machine (OSRM)~\cite{OSRM}.

\section{Состояние современных исследований} \label{sec:methods_}
Задача кластеризации заключается в разбиении множество данных объектов на непересекающиеся подмножества~--- кластеры~--- таким образом, чтобы каждое подмножество состояло из схожих объектов, а объекты разных кластеров существенно отличались, при этом кластеры изначально неизвестны. Для определения схожести объектов необходимо задавать функцию расстояния на множестве объектов.

Решение задачи кластеризации является неоднозначным. Во-первых, не существует всеми принятого критерия качества кластеризации. Существует множество логически или эмпирически выведенных критериев качества, также существует множество алгоритмов, которые не имеют четко выраженного критерия качества, но осуществляющих достаточно разумную кластеризацию. Во-вторых, число кластеров, как правило, является неизвестным заранее и устанавливается определенным критерием, отличающегося у различных конкретных алгоритмов. В-третьих, результат кластеризации существенно зависит от выбора метрики \( \rho \), выбор которой также субъективен и определяется экспертом \cite{voron, fraley}.

Поскольку выбор метрики зачастую является абсолютно субъективным, то для решения задачи кластеризации создаются новые, всё более сложные, алгоритмы и улучшаются старые. Одним из широко применяемых алгоритмов для кластеризации является алгоритм k-средних, или k-means \cite{elcan, kanungo, likas, hybrid, lenka, approxkm, krishna}.

Алгоритм k-means производит разбиение входной выборки на \( k \) кластеров. Действие алгоритма таково, что он стремится минимизировать среднеквадратичное отклонение точек кластера от его центра. Основная мысль алгоритма заключается в том, что на каждом шаге перевычисляется центр масс каждого кластера, полученного на предыдущем шаге, а затем выборка переразбивается на кластеры вновь в соответствии с тем, какой из новый центров кластеров оказался ближе по выбранной метрике. Алгоритм завершается, когда на каком-либо шаге не происходит изменения кластеров.

Достоинствами метода являются его простота и быстрота использования (вычислительная сложность равна~\( O(nki) \), где \( n \)~--- число объектов выборки, \( k \)~--- число кластеров, \( i \)~--- число итераций алгоритма), а также его понятность и прозрачность. Недостатками алгоритма являются: (i) чувствительность к выбросам, (ii) необходимость задавать количество кластеров, (iii) необходимость сканировать всю выборку для определения положения каждого кластера.

Из-за медленной работы алгоритма на больших выборках данных, а также простоты идеи и неплохих результатов для большинства выборок, предлагается множество способов ускорить или улучшить алгоритм k-средних.

Так, в работе \cite{elcan} для ускорения работы алгоритма k-means, предложено использование неравенства треугольника. Ускоренный алгоритм избегает излишних вычислений расстояния с помощью сохранения <<верхних>> и <<нижних>> границ для дистанций между точками и центрами, для расчета которых двумя различными методами применяется неравенство треугольника. Ускоренный алгоритм выдает те же результаты, что и обычный, но при этом количество вычислений дистанций сокращается в десятки раз.

Еще один вариант ускорения алгоритма путем сокращения лишних вычислений предлагают авторы статьи \cite{fastkm}: после каждой итерации алгоритма необходимо разделять кластеры на те, которые изменили своё положение, так называемые <<активные>>, и <<статические>>~--- оставшиеся на месте; и продолжать дальнейшие расчеты для активных кластеров. При этом для корректировки списков сохраняется минимальное значение расстояний от центров кластера до точек, при достижении которых кластер снова становится <<активным>>.

В работе \cite{mrkd} алгоритм ускоряется при помощи хранения результатов в так называемых <<мульти-размерных kd-деревьях>>. В узлах дерева хранятся центры кластеров, точки ищутся в <<листьях>>~--- особых гиперпрямоугольниках, что позволяет сократить количество расчетов.

В этой же статье рассматривается еще один алгоритм~--- так называемый алгоритм <<Черного списка>>. Идея заключается в идентификации тех центров, которым точно не будут принадлежать точки из листьев, и исключить их из расчетов.

В статье \cite{kanungo} предлагается <<алгоритм фильтрации>>. Он также основан на хранении многомерных данных в структуре kd-деревьев, а отличается тем, что выбор кластера, которому должен принадлежать лист, осуществляется построением гиперплоскости между претендентами.

Авторы работы \cite{approxkm} оценивают отношение расстояний между точкой и двумя центрами кластеров, и приходят к выводу, что большая часть <<активных>>, то есть сдвигающих центр, точек находится в определенном диапазоне значений этого отношения. Используя этот факт на шаге присвоения точек, они создают ускоренную версию алгоритма k-means, однако результат в данном случае оказывается не точным, а приближенным.

В диссертации \cite{hussein} был описан <<быстрый жадный алгоритм k-сред\-них>>. Предположение, используемое в алгоритме, такое же, как в обычном жадном (глобальном) алгоритме \cite{likas}; оно заключается в том, что глобальный оптимум может быть достигнут при запуске алгоритма с (\( k-1 \)) центрами, и \( k \)-тым центром, достраиваемым автоматически на подходящую позицию. Автор объединил варианты ускорения, предложенные в \cite{mrkd} и \cite{kanungo}, с наивным алгоритмом.

В работе \cite{hybrid} предложен усовершенствованный метод k-means, который путем последовательного запуска алгоритма для центров от \( 1 \) до \( k \) приближает решение к глобальному минимуму искажения. Это достигается за счет процедуры определения векторов-кандидатов на выбор позиции для вставки нового центра, при этом общее время работы изменяется незначительно.

Кришна и Мёрти в своей работе \cite{krishna} представляют <<генетический k-means>>~--- генетический алгоритм, в котором в качестве операции кроссовера выступает операция k-means. Сам генетический алгоритм был изменен для специфической работы~--- кластеризации.

Результат работы алгоритма k-средних очень сильно зависит от начального распределения кластеров. В работе \cite{kpp} рассматривается метод k-means++, в котором начальное распределение центров кластеров задается особым образом, опираясь на вероятности, рассчитанные по кратчайшим расстояниям от точек до выбранных центров.

В работе \cite{lenka} представлен алгоритм квантования цветов, основанный на методе k-средних и дополненный алгоритмом гравитационного поиска \cite{lenka8}. После стандартного расчета расстояний по метрике, рассчитываются <<массы>> кластеров и <<силы>>, действующие на них. Таким образом, в работе достигается улучшение значений заданной целевой функции.

Работы \cite{nister} и \cite{philbin} описывают построение словарного дерева с использованием двух различных вариаций алгоритма k-means: иерархического и приближенного соответственно. Иерархический алгоритм строит дерево результатов, на каждом следующем уровне разбивая имеющиеся кластеры на количество, называемое <<коэффициентом разветвления>>, при этом каждый из полученных кластеров рассчитывается независимо от других. В приближенном алгоритме изначально каждое дерево развернуто до листьев, затем итерационно создается очередь приоритетных узлов до достижения конкретного числа открытых путей по дереву. При этом увеличение количества деревьев не приводит к значительному увеличению затрачиваемого времени.

В работе \cite{ukraine} алгоритм k-средних контролируется введением и комбинированием эвристических критериев: критерия оценки ошибки кластеризации и информационного критерия функциональной эффективности.

В работе \cite{xmeans} алгоритм k-means дополнен оценкой количества кластеров с помощью алгоритма X-means, запускаемого после каждой итерации основного алгоритма. X-means основан на двух идеях для определения оптимального количества кластеров: создание нового кластера рядом с одним из существующих и разделение некоторого количества кластеров на равноотстоящие друг от друга кластеры. Решение о принятии созданной структуры принимается на основе Байесовского информационного критерия.

Определение оптимального количества кластеров во входной выборке является одной из сложнейших проблем в кластерном анализе \cite{kpp, xmeans, numbers, silhouettes, rescale}.

Авторы работы \cite{numbers} предлагают метод определения количества кластеров, основанный на искажениях: после очередной кластеризации рассчитываются соответствующие искажения, затем искажения трансформируются, рассчитываются значения <<скачков>> и среди них выбирается значение количества кластеров.

В работах \cite{silhouettes} и \cite{rescale} для определения количества кластеров используется метод <<силуэтов>>. Каждый кластер представляется силуэтом, основанном на его плотности и разреженности. В \cite{silhouettes} автор предлагает метод построения и применения силуэтов для оценки результатов кластерного анализа, а в \cite{rescale} описывается применение различных методов (индекс Данна, индекс Калинского-Харабаза и др.) для определения количества кластеров в зашумленной выборке.

В статье \cite{genetic} для определения кластерной структуры предварительно используется генетический алгоритм и метод силуэтов, которые в совокупности определяют оптимальные параметры для кластеризации методом k-средних.

Тханг, Пантюхин и Галушкин в работе \cite{FDBSCAN} представляют гибридный алгоритм кластеризации на основе алгоритма k-means и DBSCAN – плотностного алгоритма для кластеризации зашумленных пространственных данных. При существенном сокращении времени кластеризации точность при некоторых параметрах достигает точности алгоритма DBSCAN.

Также существует множество других алгоритмов кластеризации, основанных на различных идеях о поиске внутренней структуры данных. Можно выделить несколько больших групп алгоритмов по способу кластеризации~\cite{cod}: разделяющие методы, иерархические методы, плотностные методы и сеточные методы.

Для кластеризации при обработке изображений и данных высокой размерности зачастую применяют алгоритм среднего сдвига, или Mean Shift. В статье \cite{ms} автор использует этот алгоритм для определения границ регионов и фона на изображениях, а также описывает принцип работы алгоритма и критерии подбора его параметров.

Для кластеризации динамических наборов точек авторы работы \cite{inc} разработали модель, названную инкрементной кластеризацией. Она основана на тщательном анализе требований поисковых систем, целью модели является сохранить кластеры малого диаметра при добавлении новых точек.

Для кластеризации категоричных данных в работе \cite{huang} применяется обобщение парадигмы метода k-means~--- метод k-modes. В нем используется особая метрика, вместо среднего значения набора данных находится их мода и используется частотный метод обновления мод.

В работе \cite{birch} описывается алгоритм BIRCH~--- иерархичесий алгоритм кластеризации, особенно эффективный при обработке очень больших выборок. Алгоритм является локальным~--- принятие решений относительно одного кластера производится без перебора всех данных; алгоритм используют на данных с четко выраженными кластерами; алгоритм является иерархическим, благодаря чему может находить скрытые субкластеры, а время работы линейно маштабируемо.

Проблема кластеризации геораспределенных данных с учетом препятствий была описана в работах \cite{presence, cod}. В качестве наглядного примера был приведен пример расстановки банкоматов в области, через которую протекает река.

Для решения этой задачи в \cite{presence} предлагалось использовать измененный алгоритм CLARANS (более подробно рассмотрен в работе \cite{cod}), в \cite{cod} перед основной кластеризацией алгоритмом CLARANS предлагается сделать предварительную кластеризацию алгоритмом BIRCH. Сам алгоритм CLARANS был немного изменен для применения к решению задачи кластеризации геораспределенных данных.

В работе \cite{koperski} так же рекомендуется использовать алгоритм CLARANS, поскольку он эффективно обрабатывает большие пространственные базы данных. В качестве альтернативы рассматривается алгоритм BIRCH из-за более быстрой работы.

В работах \cite{estivill} и \cite{obstacles} данные о препятствиях рассчитываются из диаграмм Воронова и Делоне с помощью алгоритмов AUTOCLUST и AUTO\-CLUST+ соответственно.

\section{Требования к решаемой задаче}
Таким образом, если рассматривать задачу создания остановочных\linebreak пунктов методом кластеризации геораспределенных данных с учетом рельефа, то рассмотренные способы и подходы не решают сформулированные задачи и имеют следующие недостатки: (а) не учитывают предпочтения жителей по перемещениям в городской среде; (б) не адаптированы для работы с реальными геораспределенными данными, получаемых из картографических сервисов.

Для решения задачи предлагается подход, состоящий из следующих шагов:
\begin{enumerate}
    \item сбор и/или моделирование геораспределенных данных о перемещениях жителей в городе;
    \item кластеризация полученных данных с целью выявления узлов (остановочных пунктов) для дальнейшего построения маршрутной сети.
    \begin{enumerate}
        \item Задание характеристик производимой кластеризации.
        \item Формирование сети остановочных пунктов.
    \end{enumerate}
\end{enumerate}

\section{Заключение}
В данной главе было рассмотрено общее состояние существующей проблемы в кластеризации геораспределенных данных. Были рассмотрены различные методы, предлагаемые для кластеризации, способы ускорения широко используемых алгоритмов, а также способы учета препятствий при кластеризации.
