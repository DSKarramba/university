%% change page number %%
% \setcounter{page}{70}
%% specification moved to .. %%
\APPENDIX{Приложение А}{Техническое задание}
\addtocounter{page}{10}

\APPENDIX{Приложение Б}{Исходный код}

\noindentСсылка на исходный код \url{https://github.com/vstu-cad-stuff/clustering/tree/master}.\\
Ссылка на демонстрационные примеры \url{https://vstu-cad-stuff.github.io/clustering/}.

\noindentМодуль по настройке кластеризации и ее запуску:
\lstinputlisting{../source/kmeans.py}

\noindentМодуль кластеризации:
\lstinputlisting{../source/KMeansMachine.py}

\noindentМодуль метрики \emph{Route}:
\lstinputlisting{../source/routelib.py}

\noindentМодуль вспомогательного класса:
\lstinputlisting{../source/ClusteringMachine.py}

\noindentМодуль по загрузке данных из файлов:
\lstinputlisting{../source/DataCollector.py}

\noindentМодуль инициализации начального положения центров кластеров:
\lstinputlisting{../source/InitMachine.py}

\pagebreak
\noindentМодуль для преобразования данных в формат для визуализации. Для расчета выпуклых оболочек используется алгоритм Джарвиса:
\lstinputlisting{../source/converter.py}
