\chapter{Испытание и обоснование эффективности предлагаемых подходов}
\section{Проектирование ПО}
Разработанное программное обеспечение дожно соответствовать техническому заданию представленное в Приложении А и должна обеспечивать возможность выполнения следующих ниже функций:
\begin{enumerate}
    \item Предоставлять возможность сохранять и загружать данные используемые для работы программы:
    \begin{itemize}
        \item загрузка данных пользователей о перемещении;
        \item преобразование загруженных данных во внутренний формат программы;
        \item сохранение расчётных данных для последующей обработки;
    \end{itemize}
    \item Предоставлять функцию кластеризцаии по заданным параметрам, включающая следующие пункты:
    \begin{itemize}
        \item использование заданной метрики;
        \item ограничение на количество итераций алгоритма;
        \item способ расстановки начальных кластеров.
    \end{itemize}
    \item Предоставлять возможность визуализировать выходные данные.
\end{enumerate}

Предложенные алгоритмы из разделов \ref{sec:kmeans} и \ref{sec:distance} были реализованы с использованием языка программирования Python и сервиса построения маршрутов \emph{Open Source Routing Machine} для расчета расстояния между узлами графа по городским дорогам. Программный код опубликован на хостинге Github (подробнее в Приложении Б).

\section{Методика проведения эксперимента}
