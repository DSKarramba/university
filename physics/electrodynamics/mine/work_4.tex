\input{../../../../.preambles/01-semester_work}
\input{../../../../.preambles/10-russian}
\input{../../../../.preambles/20-math}
\input{../../../../.preambles/22-vectors}
\input{../../../../.preambles/30-physics}

\newcommand{\ds}{\displaystyle}

\begin{document}
\maketitlepage{Факультет электроники и вычислительной техники}{физики}
{Электродинамика}{4}{}{студент группы Ф-369\\Чечеткин~И.~А.}{m}
{доцент Грецов~М.~В.}{m}

\newpage
% ------------------------------------------------------------------------------
Задача 3.29: \emph{Показать, что при распространении радиоволн в ионосфере
следует ожидать  резонансных явлений вблизи длины волны
\( \lambda \approx 210 \)~м. Принять напряженность магнитного поля Земли
\( H = 0,5 \)~э.}

\vspace*{2em}
\emph{Решение:}

\vspace*{2em}   
\emph{Ответ:}

\newpage
%-------------------------------------------------------------------------------
Задача 3.51: \emph{В вакууме распространяется плоская монохроматическая волна,
которая под углом \( \alpha \) падает на плоскую границу ионосферы.
Рассматривая ионосферу как разреженный электронный газ (наличие тяжелых ионов
можно не учитывать), вычислить коэффициенты отражения и прохождения. Показать,
что если \( \omega^2 < \cfrac{4\pi Ne^2}{m} \) (\( N \) -- число электронов в
единице объема; \( \omega \) -- частота световой волны), то имеет место полное
отражение.}

\vspace*{2em}
\emph{Решение:}

\vspace*{2em}   
\emph{Ответ:}

\pagebreak
%-------------------------------------------------------------------------------
Задача 3.72: \emph{Вследствие конечности проводимости стенок волновода энергия
волны, распространяющейся вдоль волновода, частично проникает в стенки и
поглощается ими (диссипирует). Предполагая, что поверхностный импеданс
\( \zeta \) стенок волновода достаточно мал, показать, что амплитуда волны
убывает по закону \( A = A_0\e^{-\alpha} \) и что для \( E \)-волны коэффициент
поглощения
\[
    \alpha = \frac{\eps\omega\zeta'}{2\kappa^2 kc} \frac{\oint\limits_l
    \bigr|\nabla_2 E_z\bigl|^2\,dl}{\oint\limits_S \bigr|E_z\bigl|^2\,dS},
\]
где \( l \) -- контур, а \( S \) -- площадь сечения волновода.}

\vspace*{2em}
\emph{Решение:}

\vspace*{2em}
\emph{Ответ:}

\newpage
%-------------------------------------------------------------------------------
Задача 3.76: \emph{В полом резонаторе имеется одновременно несколько типов
собственных колебаний. Показать, что полная энергия этих колебаний равна сумме
энергий отдельных собственных колебаний. Стенки резонатора считать идеально
проводящими. Среду, заполняющую резонатор, однородной.}
\vspace*{-1em}

\singlespacing
{\footnotesize{\verb Указание. } \emph{Воспользоваться тождеством
\[
    \int\limits_V \bigr(\vec{A}\,\rot\rot\vec{B} - \vec{B}\,\rot\rot\vec{A}\bigl)\,
    dV = \oint\limits_S \left(\bigr[\vec{B}\times\rot\vec{A}\,\bigl] -
    \bigr[\vec{A}\times\rot\vec{B}\,\bigl]\right)\,d\vec{S},
\]
где \( S \) -- замкнутая поверхность, ограничивающая объем \( V \);
\( \vec{A} \) и \( \vec{B} \) -- произвольные векторы, удовлетворяющие обычным
условиям дифференцируемости. Доказать это тождество.}}

\vspace*{1em}
\onehalfspacing
\emph{Решение:}

\vspace*{2em}
\emph{Решение:}

\vspace*{2em}
\emph{Ответ:}
\end{document}
