\input{../../.preambles/01-semester_work}
\input{../../.preambles/10-russian}
\input{../../.preambles/20-math}
\input{../../.preambles/30-physics}

\renewcommand{\labelenumi}{\asbuk{enumi})}
\newcommand{\mr}[1]{\mathrm{#1}}

\begin{document}
\maketitlepage{Факультет электроники и вычислительной техники}{физики}
{Квантовая теория}{}{15}{студент группы Ф-369\\Чечеткин~И.~А.}{m}
{доцент Жуков~С.~С.}{m}
\newpage

%-------------------------------------------------------------------------------
Задача 1.15: \emph{Воспользовавшись формулой распределения Максвелла, найти 
функцию распределения молекул газа по дебройлевским длинам волн, 
а также их наиболее вероятную длину волны. Масса каждой молекулы 
т, температура газа Т. Вычислить наиболее вероятную длину волны 
молекул водорода при Т=300 К. }

\vspace*{2em}
\emph{Решение:}

\vspace*{2em}        
\emph{Ответ:}

\newpage

%-------------------------------------------------------------------------------
Задача 2.15: \emph{.}

\vspace*{2em}
\emph{Решение:}

\vspace*{2em}
\emph{Ответ:}

\newpage

%-------------------------------------------------------------------------------
Задача 3.6: \emph{.}

\vspace*{2em}
\emph{Решение:}

\vspace*{2em}
\emph{Ответ:}

\newpage

%-------------------------------------------------------------------------------
Задача 4.15: \emph{.}

\vspace*{2em}
\emph{Решение:}

\vspace*{2em}
\emph{Ответ:}

\newpage

%-------------------------------------------------------------------------------
Задача 5.13: \emph{.}

\vspace*{2em}
\emph{Решение:}

\vspace*{2em}
\emph{Ответ:}

\end{document}
