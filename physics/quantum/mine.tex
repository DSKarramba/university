\input{../../../.preambles/01-semester_work}
\input{../../../.preambles/10-russian}
\input{../../../.preambles/20-math}

\newcommand{\ds}{\displaystyle}
\newcommand{\h}{\hbar}
\newcommand{\comm}[2]{\left[#1,\ #2\right]}

\begin{document}
\maketitlepage{Факультет электроники и вычислительной техники}{физики}
{Квантовая теория}{}{15}{студент группы Ф-369\\Чечеткин~И.~А.}{m}
{доцент Жуков~С.~С.}{m}

\newpage

%-------------------------------------------------------------------------------
Задача 1.15: \emph{Воспользовавшись формулой распределения Максвелла, найти 
функцию распределения молекул газа по дебройлевским длинам волн, 
а также их наиболее вероятную длину волны. Масса каждой молекулы 
т, температура газа Т. Вычислить наиболее вероятную длину волны 
молекул водорода при Т=300 К.}

\vspace*{2em}
\emph{Решение:}

\vspace*{2em}        
\emph{Ответ:}

\newpage

%-------------------------------------------------------------------------------
Задача 2.15: \emph{Прямолинейная траектория частицы в камере Вильсона
представляет собой цепочку малых капелек тумана, размер которых \( d \). Можно
ли, наблюдая след электрона с кинетической энергией \( K = 1 \)~кэВ, обнаружить
отклонение в его движении от классических законов?}

\vspace*{2em}
\emph{Решение:}

\vspace*{2em}
\emph{Ответ:}

\newpage

%-------------------------------------------------------------------------------
Задача 3.6: \emph{Найти с помощью уравнения Шредингера энергию гармонического
осциллятора с частотой \( \omega \) в стационарном состоянии:\\
а) \( \ds \psi(x) = A\exp\big(-a^2x^2\big) \); б) \( \ds \psi(x) = Bx\exp\big(
-a^2x^2\big) \), где \( A \), \( B \), \( a \) -- постоянные.}

\vspace*{2em}
\emph{Решение:}

\vspace*{2em}
\emph{Ответ:}

\newpage

%-------------------------------------------------------------------------------
Задача 4.15: \emph{Доказать, что}
\vspace*{-1em}
\[
    \e^{\hat{A}}\hat{B}\e^{-\hat{A}} = \hat{B} + \comm{\hat{A}}{\hat{B}} +
    \frac{1}{2!}\comm{\hat{A}}{\comm{\hat{A}}{\hat{B}}} + \ldots
\]

\vspace*{2em}
\emph{Решение:}

\vspace*{2em}
\emph{Ответ:}

\newpage

%-------------------------------------------------------------------------------
Задача 5.13: \emph{Найти собственные значения оператора \( \hat{L}^2 \) (квадрат момента 
импульса), соответствующие его собственной функции}
\vspace*{-1em}
\[
    Y(\theta, \phi) = A\big(\cos\theta + 2\sin\theta\cos\phi\big).
\]

\vspace*{2em}
\emph{Решение:}

\vspace*{2em}
\emph{Ответ:}

\end{document}
