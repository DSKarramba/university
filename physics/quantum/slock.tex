\documentclass[a4paper, 14pt]{extarticle}
\usepackage[utf8]{inputenc}
\usepackage[paper=a4paper, top=1cm, right=1cm, bottom=1.5cm, left=2cm]{geometry}
\usepackage{setspace}
\onehalfspacing

\usepackage{graphicx}
\graphicspath{{plots/}, {images/}}

\parindent=1.25cm

\usepackage{titlesec}

\titleformat{\section}
    {\normalsize\bfseries}
    {\thesection}
    {1em}{}

\titleformat{\subsection}
    {\normalsize\bfseries}
    {\thesubsection}
    {1em}{}

% Настройка вертикальных и горизонтальных отступов
\titlespacing*{\chapter}{0pt}{-30pt}{8pt}
\titlespacing*{\section}{\parindent}{*4}{*4}
\titlespacing*{\subsection}{\parindent}{*4}{*4}

\usepackage[square, numbers, sort&compress]{natbib}
\makeatletter
\bibliographystyle{unsrt}
\renewcommand{\@biblabel}[1]{#1.} 
\makeatother


\newcommand{\maketitlepage}[6]{
    \begin{titlepage}
        \singlespacing
        \newpage
        \begin{center}
            Министерство образования и науки Российской Федерации \\
            Федеральное государственное бюджетное образовательное \\
            учреждение высшего профессионального образования \\
            <<Волгоградский государственный технический университет>> \\
            #1 \\
            Кафедра #2
        \end{center}


        \vspace{14em}

        \begin{center}
            \large Семестровая работа #6 по дисциплине
            \\ <<#3>>
        \end{center}

        \vspace{5em}

        \begin{flushright}
            \begin{minipage}{.35\textwidth}
                Выполнила:\\#4
                \vspace{1em}\\
                Проверил:\\#5
                \\
                \\ Оценка \underline{\ \ \ \ \ \ \ \ \ \ \ \ \ \ \ \ }
            \end{minipage}
        \end{flushright}

        \vspace{\fill}

        \begin{center}
            Волгоград, \the\year
        \end{center}

    \end{titlepage}
    \setcounter{page}{2}
}

\newcommand{\maketitlepagewithvariant}[7]{
    \begin{titlepage}
        \singlespacing
        \newpage

        \begin{center}
            Министерство образования и науки Российской Федерации \\
            Федеральное государственное бюджетное образовательное \\
            учреждение высшего профессионального образования \\
            <<Волгоградский государственный технический университет>> \\
            #1 \\
            Кафедра #2
        \end{center}


        \vspace{8em}

        \begin{center}
            \large Семестровая работа #6 по дисциплине
            \\ <<#3>>
        \end{center}

        \vspace{1em}
        \begin{center}
            Вариант №#7
        \end{center}
        \vspace{4em}

        \begin{flushright}
            \begin{minipage}{.35\textwidth}
                Выполнила:\\#4
                \vspace{1em}\\
                Проверил:\\#5
                \\
                \\ Оценка \underline{\ \ \ \ \ \ \ \ \ \ \ \ \ \ \ \ }
            \end{minipage}
        \end{flushright}

        \vspace{\fill}

        \begin{center}
            Волгоград, \the\year
        \end{center}

    \end{titlepage}
    \setcounter{page}{2}
}

\input{../../../.preambles/10-russian}
\input{../../../.preambles/20-math}
\input{../../../.preambles/22-vectors}

\newcommand{\comm}[2]{\left[#1,\ #2\right]}
\newcommand{\h}{\hbar}
\newcommand{\ds}{\displaystyle}

\begin{document}
\maketitlepage{Факультет электроники и вычислительной техники}{физики}
{Квантовая теория}{}{14}{студентка группы Ф-369\\Слоква~В.~И.}{f}
{к.ф.-м.н., доцент Жуков~С.~С.}{m}
\newpage

% ------------------------------------------------------------------------------
Задача 1.14: \emph{Вычислить длину волны релятивистских электронов, подлетающих
к антикатоду рентгеновской трубки, если длина волны коротковолновой границы
сплошного рентгеновского спектра равна \( \lambda_k = 10,\!0 \)~пм.}

\vspace*{2em}
\emph{Решение:}

\vspace*{2em}
\emph{Ответ:}

\newpage

% ------------------------------------------------------------------------------
Задача 2.14: \emph{Ускоряющее напряжение на электронно-лучевой трубке
\( U = 10 \)~кВ. Расстояние от электронной пушки до экрана \( l = 20 \)~см.
Оценить неопределенность координаты электрона на экране, если след электронного
пучка на экране имеет диаметр \( d = 0,\!5 \)~мм.}

\vspace*{2em}
\emph{Решение:}

\vspace*{2em}
\emph{Ответ:}

\newpage

% ------------------------------------------------------------------------------
Задача 3.7: \emph{Частица массы \( m \) падает слева на прямоугольный
потенциальный барьер высотой \( U_0 \) (рис. \ref{pic3_7}). Энергия частицы
равна \( E \), причем \( E < U_0 \). Найти эффективную глубину
\( x_\text{эф} \) проникновения частицы под барьер, т.е. расстояние от границы
барьера до точки, в которой плотность вероятности \( w \) нахождения частицы
уменьшается в \( \e \) раз. Вычислить \( x_\text{эф} \) для электрона, если
\( U_0 - E = 1 \)~эВ.}

\vspace*{2em}
\emph{Решение:}

\vspace*{2em}
\emph{Ответ:}

\newpage

% ------------------------------------------------------------------------------
Задача 4.14: \emph{Доказать тождества:
а) \( \ds \comm{f(\vec{r})}{\hat{\vec{L}}} = i\h\left[\vec{r}\times\grad
f(\vec{r})\right] \);\\ б) \( \ds \comm{\vec{A}(\vec{r})}{\hat{\vec{L}}} =
i\h\,\vec{r}\,\rot\vec{A} \).}

\vspace*{2em}
\emph{Решение:}

\vspace*{2em}
\emph{Ответ:}

\newpage

% ------------------------------------------------------------------------------
Задача 5.12: \emph{Найти общую собственную функцию следующих операторов:
а) \( \hat{x} \) и \( \hat{p}_y \); б) \( \hat{p}_x \), \( \hat{p}_y \) и
\( \hat{p}_z \); в) \( \hat{p}_x \) и \( \hat{p}_x^{\,2} \).}

\vspace*{2em}
\emph{Решение:}

\vspace*{2em}
\emph{Ответ:}
\end{document}