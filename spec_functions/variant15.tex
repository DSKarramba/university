\documentclass[a4paper, 14pt]{extarticle}
\usepackage[utf8]{inputenc}
\usepackage[paper=a4paper, top=1cm, right=1cm, bottom=1.5cm, left=2cm]{geometry}
\usepackage{setspace}
\onehalfspacing

\usepackage{graphicx}
\graphicspath{{plots/}, {images/}}

\parindent=1.25cm

\usepackage{titlesec}

\titleformat{\section}
    {\normalsize\bfseries}
    {\thesection}
    {1em}{}

\titleformat{\subsection}
    {\normalsize\bfseries}
    {\thesubsection}
    {1em}{}

% Настройка вертикальных и горизонтальных отступов
\titlespacing*{\chapter}{0pt}{-30pt}{8pt}
\titlespacing*{\section}{\parindent}{*4}{*4}
\titlespacing*{\subsection}{\parindent}{*4}{*4}

\usepackage[square, numbers, sort&compress]{natbib}
\makeatletter
\bibliographystyle{unsrt}
\renewcommand{\@biblabel}[1]{#1.} 
\makeatother


\newcommand{\maketitlepage}[6]{
    \begin{titlepage}
        \singlespacing
        \newpage
        \begin{center}
            Министерство образования и науки Российской Федерации \\
            Федеральное государственное бюджетное образовательное \\
            учреждение высшего профессионального образования \\
            <<Волгоградский государственный технический университет>> \\
            #1 \\
            Кафедра #2
        \end{center}


        \vspace{14em}

        \begin{center}
            \large Семестровая работа #6 по дисциплине
            \\ <<#3>>
        \end{center}

        \vspace{5em}

        \begin{flushright}
            \begin{minipage}{.35\textwidth}
                Выполнила:\\#4
                \vspace{1em}\\
                Проверил:\\#5
                \\
                \\ Оценка \underline{\ \ \ \ \ \ \ \ \ \ \ \ \ \ \ \ }
            \end{minipage}
        \end{flushright}

        \vspace{\fill}

        \begin{center}
            Волгоград, \the\year
        \end{center}

    \end{titlepage}
    \setcounter{page}{2}
}

\newcommand{\maketitlepagewithvariant}[7]{
    \begin{titlepage}
        \singlespacing
        \newpage

        \begin{center}
            Министерство образования и науки Российской Федерации \\
            Федеральное государственное бюджетное образовательное \\
            учреждение высшего профессионального образования \\
            <<Волгоградский государственный технический университет>> \\
            #1 \\
            Кафедра #2
        \end{center}


        \vspace{8em}

        \begin{center}
            \large Семестровая работа #6 по дисциплине
            \\ <<#3>>
        \end{center}

        \vspace{1em}
        \begin{center}
            Вариант №#7
        \end{center}
        \vspace{4em}

        \begin{flushright}
            \begin{minipage}{.35\textwidth}
                Выполнила:\\#4
                \vspace{1em}\\
                Проверил:\\#5
                \\
                \\ Оценка \underline{\ \ \ \ \ \ \ \ \ \ \ \ \ \ \ \ }
            \end{minipage}
        \end{flushright}

        \vspace{\fill}

        \begin{center}
            Волгоград, \the\year
        \end{center}

    \end{titlepage}
    \setcounter{page}{2}
}

\input{../../.preambles/10-russian}
\input{../../.preambles/20-math}
\input{../../.preambles/22-vectors}
\input{../../.preambles/30-physics}

\renewcommand{\L}{\,\Delta\,}

\begin{document}
\maketitlepage{Факультет электроники и вычислительной техники}{физики}
{Специальные функции}{студентка группы Ф-369\\Слоква~В.~И.}
{к.~ф.-м.~н., доцент\\Никулин~Р.~Н.}{\!\!}

№15. Решить задачу о распространении ТЕ-волн в полости проводника,
представляющей круглый цилиндр неограниченной длины (круглая труба).
    
[Найти составляющие \( H_{z, nm}(\rho, \phi)\).]

\vspace*{2em}
\emph{Решение}. Данный волновод представляет собой полую бесконечно
протяженную трубку с внутренним радиусом \( \rho_0 \).
На границе волновода \( \left.\pder{H_z}{\rho}\right|_{\rho_0} = 0 \).

Запишем уравнения Максвелла для нашей задачи:
\begin{align*}
    & \rot\vec{H} = \lambda\vec{E} + \eps\Ezero\pder{\vec{E}}{t}, \\
    & \rot\vec{E} = -\mu\mu_0\pder{\vec{H}}{t}, \\
    & \div\vec{E} = 0, \\
    & \div\vec{H} = 0.
\end{align*}

Найдем ротор от обеих частей первого уравнения, учитывая, что порядок операций
ротора и производной по времени можно менять
\( \left(\rot\pder{\vec{E}}{t}= \pder{\rot\vec{E}}{t}\right) \):
\begin{equation}
    \rot\!\rot\vec{H} = \lambda\rot\vec{E} + \eps\Ezero\pder{\rot\vec{E}}{t}.
    \label{eq:1}
\end{equation}

Продифференцируем по \( t \) второе уравнение:
\begin{equation}
    \pder{\rot\vec{E}}{t} = -\mu\mu_0\ppder{\vec{H}}{t}.
    \label{eq:2}
\end{equation}

Подставляя \eqref{eq:2} в \eqref{eq:1}, получим:
\[
    \rot\!\rot\vec{H} = -\lambda\mu\mu_0\pder{\vec{H}}{t} - \eps\Ezero\mu\mu_0
    \ppder{\vec{H}}{t}.
\]

С учетом того, что \( \rot\!\rot\vec{H} = \grad\div\vec{H} - \L\vec{H} \) и
\( \div\vec{H} = 0 \), получим:
\begin{equation}
    \L\vec{H} = \lambda\mu\mu_0\pder{\vec{H}}{t} + \eps\Ezero\mu\mu_0
    \ppder{\vec{H}}{t}.
    \label{eq:4}
\end{equation}

Представим \( \vec{H} \) в виде волны, распространяющейся вдоль оси \( z \):
\begin{equation}
    \vec{H} =\vec{H}(r, \phi) e^{i(kz -\omega t)},
    \label{eq:7}
\end{equation}
где \( \vec{H}(r, \phi) \) -- функция, зависящая только от \( r \) и \( \phi \).

Лапласиан в цилиндрических координатах имеет вид:
\[
    \L = \ppder{}{\rho} + \frac{1}{\rho} \pder{}{\rho} + \frac{1}{\rho^2} \ppder{}{\phi}
    + \ppder{}{z}.
\]

Учитывая, что
\begin{align*}
    & \pder{\vec{H}}{z} = \vec{H}(r, \phi)\, ike^{i(kz - \omega t)} = ik\vec{H}, \\
    & \pder{\vec{H}}{t} = -i\omega\vec{H}, \; \ \ppder{\vec{H}}{z} = -k^2\vec{H}, \;\ 
    \ppder{\vec{H}}{t} = \omega^2\vec{H};
\end{align*}
получим, перенося производную по \( z \) в правую часть:
\[
    \L_{\rho\phi}\vec{H} = [\omega^2\eps\Ezero\mu\mu_0 - i\lambda\mu\mu_0\omega
    + k^2]\vec{H}
    = \gamma\vec{H},
\]
где \( \gamma = -i\lambda\mu\mu_0\omega + \omega^2\eps\Ezero\mu\mu_0 + k^2 \).

Нас интересует только проекция на ось \( z \), поэтому:
\[
    \L_{\rho\phi} H_z = \gamma H_z.
\]

Расписывая радиальную и полярную части лапласиана, получим уравнение:
\begin{equation}
    \ppder{H_z}{\rho} + \frac{1}{\rho} \pder{H_z}{\rho} + \frac{1}{\rho^2} \ppder{H_z}{\phi}
    = \gamma H_z.
    \label{eq:extra_0}
\end{equation}

Используем метод разделения переменных:
\begin{equation}
    \tilde{H}(r, \phi) = v(r)\cdot w(\phi).
    \label{eq:extra_1}
\end{equation}

Подставляя \eqref{eq:extra_1} в уравнение \eqref{eq:extra_0}, получим:
\[
    w\dder{v}{\rho} + w\frac{1}{\rho}\der{v}{\rho} + \frac{1}{\rho^2}v\dder{w}{\phi}
    = \gamma vw.
\]

Домножив это уравнение на \( \rho^2 \) и разделив на \( vw \), получим:
\[
    \rho^2\frac{v''}{v} + \rho\frac{v'}{v} - \gamma\rho^2 = -\frac{w''}{w}.
\]

Левая часть зависит только от \( \rho \), а правая -- только от \( \phi \). Следовательно,
справедливы равенства:
\begin{align*}
    & \rho^2\frac{v''}{v} + \rho\frac{v'}{v} - \gamma\rho^2 = \beta, \\
    & -\frac{w''(\phi)}{w(\phi)} = \beta.
\end{align*}

Таким образом, получаем систему дифференциальных уравнений:
\begin{align}
    & \rho^2v'' + \rho v' + (\gamma\rho^2 - \beta)v = 0, \label{eq:extra_2} \\
    & w'' + \beta w = 0. \label{eq:extra_3}
\end{align}

Из условия периодичности этой функции \( w(\phi) = w(\phi + 2\pi) \) следует, что
\( \beta = n^2 \), где \( n = 1, 2, \ldots \). Тогда
\[
    w = A\cos n\phi + B\sin n\phi.
\]

Подставляя \( \beta \) в уравнение \eqref{eq:extra_2}, получим:
\[
    \rho^2v'' + \rho v' + (\gamma\rho^2 - n^2)v = 0.
\]

Проведем замену: \( -\gamma = \alpha \), \( \sqrt{\alpha}\rho = r \),
\( \sqrt{\alpha}\d\rho = \d r \). Тогда уравнение примет вид уравнения Бесселя:
\[
    r^2\dder{v}{r} + r\der{v}{r} + (r^2 - n^2)v = 0.
\]

Его общим решением является функция вида
\[
    v(r) = C_1 J_n(r) + C_2 Y_n(r),
\]
где \( C_1 \) и \( C_2 \) -- некоторые постоянные, \( J_n(r) \) -- функция Бесселя
первого рода, \( Y_n(r) \) -- функция Неймана.

Так как \( |Y_n(0)| \to \infty \), то постоянную \( C_2 \) полагаем равной нулю.
Таким образом, функция \( v(r) \):
\[
    v(r) = C_1 J_n(\sqrt{\alpha_{nm}}\rho),
\]
где значения \( \sqrt{\alpha_{nm}} \) находятся из уравнения, определяемого
граничным условием:
\[
    \left.\pder{H_z}{\rho}\right|_{\rho_0} = 0\ \Rightarrow
    \left.\der{v}{\rho}\right|_{\rho_0} = 0\ \Rightarrow
    \left.\der{J_n(\sqrt{\alpha_{nm}}\rho_0)}{\rho}\right|_{\rho_0} = 0.
\]

Таким образом, частные решения уравнения \eqref{eq:extra_0}, удовлетворяющие
граничному условию, имеют вид:
\[
    H_{z, nm}(\rho, \phi) = J_n(\sqrt{\alpha_{nm}}\rho)
    \left[A_{nm}\cos n\phi + B_{nm}\sin n\phi \right],
\]
где \( A_{nm} \) и \( B_{nm} \) -- произвольные постоянные.

\newpage
\addcontentsline{toc}{section}{Список литературы}
\begin{thebibliography}{9}
    \bibitem{Koshljakov}Кошляков,~Н.~С. Уравнения в частных производных
    математической физики [Текст] / Кошляков~Н.~С., Глинер~Е.~Б., Смирнов~М.~М.
    Учебное пособие. -- М.: <<Высшая школа>>, 1970.-- 712с.
    \bibitem{Tarabrin}Тарабрин,~Г.~Т.  Методы математической физики [Текст] /
    Тарабрин~Г.~Т. Учебное пособие. -- М.: <<АСВ>>, 2009.-- 208с.
\end{thebibliography}
\end{document}