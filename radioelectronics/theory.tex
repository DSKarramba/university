\documentclass[14pt,a4paper]{extarticle}
\usepackage[utf8]{inputenc}
\usepackage[russian]{babel}

\usepackage{geometry}
\geometry{top=2cm, right=1cm, bottom=2cm, left=2cm}

\usepackage{amsmath}
\usepackage{graphicx}
\graphicspath{{images/}}

\usepackage{setspace}
\usepackage{pscyr}

\begin{document}
  \begin{titlepage}
    \singlespacing
    \begin{center}
      Министерство образования и науки Российской Федерации \\
      Федеральное государственное бюджетное образовательное \\
      учреждение высшего профессионального образования \\
      <<Волгоградский государственный технический университет>> \\
      Факультет электроники и вычислительной техники \\
      Кафедра физики
    \end{center}

    \vspace{9em}

    \begin{center}
      \large Семестровая работа по дисциплине \\
      <<Радиоэлектроника>>
    \end{center}

    \vspace{5em}

    \begin{flushright}
      \begin{minipage}{.40\textwidth}
        Выполнили \\
        студенты группы Ф-469 \\
        Слоква В. И., \\
        Чечеткин И. А. \\

        \vspace{1em}

        Проверил профессор, \\
        доктор физ.-мат. наук \\
        Смоляр В. А.
      \end{minipage}
    \end{flushright}

    \vspace{\fill}

    \begin{center}
      Волгоград, \the\year
    \end{center}
  \end{titlepage}

  \setcounter{page}{2}
  \tableofcontents
  \newpage

  Концентрация электронов в зоне проводимости равна:
  \begin{equation}
    n = 2\cdot\int\limits_{E_C}^\infty N(E) f(E, T) dE.
    \label{G1.9}
  \end{equation}
  
  Если уровень Ферми \( F \) лежит в запрещенной зоне энергий и удален от края
  зоны \( Е_C \) хотя бы на \( 2kT \). Тогда в распределении Ферми-Дирака
  \eqref{G1.7}
  \begin{equation}
    f(E, T) = \frac{1}{1 + e^{\cfrac{E - F}{kT}}}
    \label{G1.7}
  \end{equation}
  единицей в знаменателе можно пренебречь и оно переходит в распределение
  Максвелла-Больцмана классической статистики (для кремния показаны на
  рис.~\ref{picF}, рис.~\ref{picF2}). Это случай невырожденного полупроводника:
  \begin{equation}
    f(E, T) = e^{-\cfrac{E - F}{kT}}.
    \label{G1.8}
  \end{equation}

  \begin{figure}[ht]
    \center
    \includegraphics[width=.75\textwidth]{f1500K}\\
    \caption{Функции Ферми-Дирака \( f \) и Больцмана \( f_\emph{Б} \) для
      температуры 1500 К}
    \label{picF}
  \end{figure}

  Плотность состояний в зоне проводимости \( N(E) \) выражается формулой
  \begin{equation}
    N(E) = \frac{4\pi m_n^{3/2}\cdot\sqrt{2(E - E_C)}}{h^3},
    \label{G1.6}
  \end{equation}
  где \( m_n^{3/2} \)~-- эффективная масса электрона, \( E_C \)~-- энергия,
  соответствующая дну зоны проводимости.

  Если вместо \( (E - E_C) \) подставить \( (E_V - E) \), а вместо \( m_n \)~--
  эффективную массу дырки \( m_p \), то получим формулу плотности состояний в
  валентной зоне.

  Для кремния \( N(E) \) выглядит так, как показано на рисунке~\ref{picNE}.
  \begin{figure}[ht]
    \center
    \includegraphics[width=.75\textwidth]{N(E)}
    \caption{Функция распределения плотности состояний в зоне проводимости
      \( N(E) \)}
    \label{picNE}
  \end{figure}

  Подставив \eqref{G1.8} и \eqref{G1.6} в \eqref{G1.9}, получим:
  \begin{equation}
    n = N_C\cdot e^{-\cfrac{E_C - F}{kT}},
    \label{G1.10}
  \end{equation}
  где \( N_C \) -- эффективная плотность состояний в зоне проводимости:
  \[
    N_C = 2\cdot\left( \frac{2\pi m_n kT}{h^2} \right)^{3/2}.
  \]

  Концентрация дырок в валентной зоне:
  \begin{equation}
    p = N_V\cdot e^{-\cfrac{F - E_V}{kT}},
    \label{G1.13}
  \end{equation}
  где \( N_V \) -- эффективная плотность состояний в валентной зоне:
  \[
    N_V = 2\cdot\left( \frac{2\pi m_p kT}{h^2} \right)^{3/2}.
  \]

  Для расчета \( n \) и \( p \) по уравнениям \eqref{G1.10} и \eqref{G1.13}
  необходимо знать положение уровня Ферми \( F \). Однако произведение
  концентраций электронов и дырок для невырожденного полупроводника не зависит
  от уровня Ферми, хотя зависит от температуры:
  \begin{equation}
    n\cdot p = (n_i)^2 = N_C\cdot N_V\cdot e^{-\cfrac{E_g}{kT}}.
    \label{G1.14}
  \end{equation}

  Концентрацию собственных носителей заряда в зоне проводимости и в валентной
  зоне рассчитаем из формулы \eqref{G1.14}. Для кремния зависимость
  \( n_i(T) \), \( N_C(T) \), \( N_V(T) \) отображена на рисунке~\ref{picNi}.
  \begin{figure}[ht]
    \includegraphics[width=.75\textwidth]{n(i)}
    \caption{Зависимость \( n_i(T) \)}
    \label{picNi}
  \end{figure}
\end{document}