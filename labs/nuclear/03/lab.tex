\input{../../../../.preambles/02-lab_work}
\newgeometry{top=1.5cm, bottom=1.5cm, left=1cm, right=1cm}
\usepackage{epstopdf}
\begin{document}
    \begin{table}[h!]
        \center
        \begin{tabular}{|C{.5}|C{.2}|C{.25}|}
            \hline
            \multicolumn{1}{|c|}{\multirow{4}{*}{Лабораторная работа № 3}} &
            Студент, группа & {{ student }}, Ф-369 \\ \cline{2-3}
            & Дата выполнения & 16.02.2013 \\ \cline{2-3}
            & Подпись &  \\ \cline{2-3}
            Прохождение гамма-излучения через вещество & Дата отчёта & \\ \cline{2-3}
            & Оценка &  \\ \cline{2-3}
            & Подпись &  \\ \hline
        \end{tabular}
    \end{table}

    \emph{Цель работы:} исследовать зависимость линейного коэффициента
    поглощения от материала, через который проходит пучок, а также от величины
    энергии налетающего гамма-кванта.

    \subsection{Вольфрам}
    \begin{table}[h!]
        \center
        \caption{Результаты эксперимента для вольфрама}
        \begin{tabular}{|C{.11}|C{.14}||C{.13}|*{6}{C{.06}|}} \hline
            Энергия кванта, МэВ & Коэффициент линейного поглощения \( \tau \)
            & Толщина \( x \),~см & 1,0 & 2,0 & 3,0 & 4,0 & 5,0 & 6,0 \\ \hline
            \multirow{2}{*}{10,0} & \multirow{2}{*}{0,895} & \( I(l)/I(0) \) &
            0,427 & 0,164 & 0,065 & 0,027 & 0,011 & 0,005 \\ \cline{3-9}
            & & \( \ln\bigl(I(l)/I(0)\bigr) \) &
            -0,851 & -1,808 & -2,752 & -3,612 & -4,510 & -5,298 \\ \hline
            \multirow{2}{*}{8,0} & \multirow{2}{*}{0,852} & \( I(l)/I(0) \) &
            0,430 & 0,180 & 0,080 & 0,032 & 0,015 & 0,006 \\ \cline{3-9}
            & & \( \ln\bigl(I(l)/I(0)\bigr) \) &
            -0,844 & -1,715 & -2,826 & -3,492 & -4,200 & -5,116 \\ \hline
        \end{tabular}
    \end{table}
    
    \begin{figure}[h!]
        \includegraphics[width=.47\textwidth]{w-10} \hfill
        \includegraphics[width=.47\textwidth]{w-8}
        \parbox{.47\textwidth}{\caption{При энергии кванта в 10,0 МэВ}} \hfill
        \parbox{.47\textwidth}{\caption{При энергии кванта в 8,0 МэВ}}
    \end{figure}
    
    \pagebreak
    
    \subsection{Бетон}
    \begin{table}[h!]
        \center
        \caption{Результаты эксперимента для бетона}
        \begin{tabular}{|C{.11}|C{.14}||C{.13}|*{6}{C{.06}|}} \hline
            Энергия кванта, МэВ & Коэффициент линейного поглощения \( \tau \)
            & Толщина \( x \),~см & 1,0 & 2,0 & 3,0 & 4,0 & 5,0 & 6,0 \\ \hline
            \multirow{2}{*}{1,5} & \multirow{2}{*}{0,122} & \( I(l)/I(0) \) &
            0,895 & 0,773 & 0,680 & 0,594 & 0,555 & 0,492 \\ \cline{3-9}
            & & \( \ln\bigl(I(l)/I(0)\bigr) \) &
            -0,111 & -0,257 & -0,376 & -0,521 & -0,589 & -0,709 \\ \hline
            \multirow{2}{*}{3,0} & \multirow{2}{*}{0,081} & \( I(l)/I(0) \) &
            0,949 & 0,811 & 0,784 & 0,721 & 0,679 & 0,617 \\ \cline{3-9}
            & & \( \ln\bigl(I(l)/I(0)\bigr) \) &
            -0,052 & -0,209 & -0,243 & -0,327 & -0,387 & -0,483 \\ \hline
        \end{tabular}
    \end{table}
    
    \begin{figure}[h!]
        \includegraphics[width=.47\textwidth]{c-1_5} \hfill
        \includegraphics[width=.47\textwidth]{c-3}
        \parbox{.47\textwidth}{\caption{При энергии кванта в 1,5 МэВ}} \hfill
        \parbox{.47\textwidth}{\caption{При энергии кванта в 3,0 МэВ}}
    \end{figure}
    
    \vspace*{-2em}

    \subsection{Фосфор}
    \begin{table}[h!]
        \center
        \caption{Результаты эксперимента для фосфора}
        \begin{tabular}{|C{.11}|C{.14}||C{.13}|*{6}{C{.06}|}} \hline
            Энергия кванта, МэВ & Коэффициент линейного поглощения \( \tau \)
            & Толщина \( x \),~см & 1,0 & 2,0 & 3,0 & 4,0 & 5,0 & 6,0 \\ \hline
            \multirow{2}{*}{0,15} & \multirow{2}{*}{0,250} & \( I(l)/I(0) \) &
            0,747 & 0,606 & 0,454 & 0,357 & 0,284 & 0,237 \\ \cline{3-9}
            & & \( \ln\bigl(I(l)/I(0)\bigr) \) &
            -0,292 & -0,501 & -0,790 & -1,020 & -1,259 & -1,440 \\ \hline
            \multirow{2}{*}{0,8} & \multirow{2}{*}{0,126} & \( I(l)/I(0) \) &
            0,853 & 0,784 & 0,706 & 0,584 & 0,520 & 0,483 \\ \cline{3-9}
            & & \( \ln\bigl(I(l)/I(0)\bigr) \) &
            -0,159 & -0,243 & -0,348 & -0,538 & -0,654 & -0,728 \\ \hline
        \end{tabular}
    \end{table}
    
    \begin{figure}[h!]
        \includegraphics[width=.47\textwidth]{p-0_15} \hfill
        \includegraphics[width=.47\textwidth]{p-0_8}
        \parbox{.47\textwidth}{\caption{При энергии кванта в 0,15 МэВ}} \hfill
        \parbox{.47\textwidth}{\caption{При энергии кванта в 0,8 МэВ}}
    \end{figure}
    
    \emph{Вывод:} провел опыт, моделирующий прохождение гамма-излучения через
    вещество, и исследовал зависимость линейного коэффициента поглощения от
    материала среды, через которую проходит пучок, а также энергии налетающего
    гамма-кванта.
\end{document}
