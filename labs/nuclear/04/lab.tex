\input{../../../.preambles/02-lab_work}
\newgeometry{top=1.5cm, bottom=1.5cm, left=1cm, right=1cm}
\begin{document}
    \begin{table}[h!]
        \center
        \begin{tabular}{|C{.5}|C{.2}|C{.25}|}
            \hline
            \multicolumn{1}{|c|}{\multirow{4}{*}{Лабораторная работа № 4}} &
            Студент, группа & {{ student }}, Ф-369 \\ \cline{2-3}
            & Дата выполнения & 06.03.2013 \\ \cline{2-3}
            & Подпись &  \\ \cline{2-3}
            Изучение явления радиоактивности & Дата отчёта & \\ \cline{2-3}
            & Оценка &  \\ \cline{2-3}
            & Подпись &  \\ \hline
        \end{tabular}
    \end{table}

    \emph{Цель работы:} Изучение явления радиоактивности с помощью компьютерного
    моделирования процессов \( \alpha \)- и \( \beta \)-распадов радиоактивных ядер.
    
    % \emph{Вывод:}
\end{document}
