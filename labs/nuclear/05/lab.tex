\input{../../../.preambles/02-lab_work}
\newgeometry{top=1.5cm, bottom=1.5cm, left=1cm, right=1cm}
\begin{document}
    \begin{table}[h!]
        \center
        \begin{tabular}{|C{.5}|C{.2}|C{.25}|}
            \hline
            \multicolumn{1}{|c|}{\multirow{4}{*}{Лабораторная работа № 5}} &
            Студент, группа & {{ student }}, Ф-369 \\ \cline{2-3}
            & Дата выполнения & 02.03.2013 \\ \cline{2-3}
            & Подпись &  \\ \cline{2-3}
            Энергетические соотношения для бета-распада & Дата отчёта & \\ \cline{2-3}
            & Оценка &  \\ \cline{2-3}
            & Подпись &  \\ \hline
        \end{tabular}
    \end{table}

    \emph{Цель работы:} Ознакомиться с энергетическими соотношениями для
    \( \beta \)-распада и на основе экспериментальных значений энергий связи
    построить семейства парабол, иллюстрирующих \( \beta \)-превращения для
    чётных и нечётных значений массового числа \( A \).
    
    % \emph{Вывод:}
\end{document}
