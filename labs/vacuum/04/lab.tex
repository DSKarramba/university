\input{../../../.preambles/02-lab_work}
\newgeometry{top=1.5cm, bottom=1.5cm, left=1cm, right=1cm}
\begin{document}
    \begin{table}[h!]
        \center
        \begin{tabular}{|C{.5}|C{.2}|C{.25}|}
            \hline
            \multicolumn{1}{|c|}{\multirow{4}{*}{Лабораторная работа № 4}} &
            Студент, группа & {{ student }}, Ф-369 \\ \cline{2-3}
            & Дата выполнения & 02.03.2013 \\ \cline{2-3}
            & Подпись &  \\ \cline{2-3}
            Исследование характеристик & Дата отчёта & \\ \cline{2-3}
            электронно-лучевой трубки & Оценка &  \\ \cline{2-3}
            & Подпись &  \\ \hline
        \end{tabular}
    \end{table}

    \emph{Цель работы:} Ознакомиться с принципом работы, конструкциями и
    характеристиками электронно-лучевых трубок (ЭЛТ).
    
    \begin{table}[h!]
        \center
        \caption{Зависимость смещения луча по горизонтали от величины
        отклоняющего напряжения}
        \begin{tabular}{|C{.08}|*{13}{C{.04}|}} \hline
            \( X \), мм & -30 & -25 & -20 & -15 & -10 & -5 & 0 & 5 & 10 & 15
            & 20 & 25 & 30 \\ \hline
            \( U \), \ В &&&&&&&&&&&&& \\ \hline
            \( U \), \ В &&&&&&&&&&&&& \\ \hline
            \( U \), \ В &&&&&&&&&&&&& \\ \hline
            \( U \), \ В &&&&&&&&&&&&& \\ \hline
        \end{tabular}
    \end{table}
    
    \begin{table}[h!]
        \center
        \caption{Зависимость смещения луча по вертикали от величины
        отклоняющего напряжения}
        \begin{tabular}{|C{.08}|*{13}{C{.04}|}} \hline
            \( Y \), мм & -30 & -25 & -20 & -15 & -10 & -5 & 0 & 5 & 10 & 15
            & 20 & 25 & 30 \\ \hline
            \( U \), \ В &&&&&&&&&&&&& \\ \hline
            \( U \), \ В &&&&&&&&&&&&& \\ \hline
            \( U \), \ В &&&&&&&&&&&&& \\ \hline
            \( U \), \ В &&&&&&&&&&&&& \\ \hline
        \end{tabular}
    \end{table}
    
    \begin{table}[h!]
        \center
        \caption{Зависимость тока эмиссии с катода от напряжения на модуляторе}
        \begin{tabular}{|*{8}{C{.08}|}} \hline
            \( J_K \), мкА &&&&&&& \\ \hline
            \( U_M \), В &&&&&&& \\ \hline
        \end{tabular}
    \end{table}
    % \emph{Вывод:}
\end{document}
