\documentclass[12pt,pscyr]{hedwork}
\usepackage[russian]{babel}

\pagestyle{empty}

\begin{document}

  \begin{flushright}
    Выполнила студентка группы Ф-469 Слоква В. И.
  \end{flushright}
  \vspace{-2em}
  \begin{center}
    \bfseries <<Музыка обретает свои краски>>
  \end{center}

  11-го мая в моей жизни произошло событие, которое надолго останется в моей
  памяти. Я впервые услышала музыку в исполнении симфонического оркестра.

  Концерт состоял из двух частей. В первой части прозвучали композиции русских
  классиков: Римского-Корсакова, Рахманинова и Чайковского.

  Выражение <<первый блин комом>> было неуместным. Первым озвученным
  произведением было произведение Римского-Корсакова, посвященное <<Сказке о
  царе Салтане>>.

  Впечатления начались с первых нот. <<Живая>> музыка~-- это то, что, на мой
  взгляд, заставляет трепетать сердце. Изобилие музыкальных инструментов
  вызывают целую палитру эмоций и ощущений.

  На смену Римскому-Корсакову пришел Рахманинов. Его композиция была насквозь
  пропитана природой, которой он часто вдохновлялся. Подобную информацию о
  произведении сообщал сам дирижер перед тем, как оркестр начинал играть.
  Однако, на мой взгляд, эту информацию можно было немного сократить или
  перефразировать, поскольку его речи казались слишком затянутыми.

  <<Закрыли>> классическую часть концерта увертюрой Чайковский, посвященной
  Отечественной войне 1812~года; отличное, на мой взгляд, завершение первой
  части.

  Я на протяжении всего концерта испытывала самые разнообразные чувства. Раньше
  мне приходилось слышать или читать выражения <<музыка обретает свои краски>>,
  <<музыка имеет цвет/чувства/вкус/запах>>. Я понимала их смысл, но на концерте
  я прочувствовала, так сказать, все чувства и эмоции, которые музыка может
  вызвать в человеке.

  Во второй части выступления была исполнена сюита Коумена <<Симфонический Led
  Zeppelin>>. Сыграли 4 отрывка из нее~-- 4 песни известной на весь мир группы
  в оркестровой обработке, что позволило взглянуть на рок-музыку с другой
  стороны.

  Но были и недостатки, главным из которых, на мой взгляд, является выступление
  скрипача-солиста во время исполнения баллады <<Kashmir>>. В некоторых местах
  оркестр просто заглушал партию соло-исполнителя, которая заглушаться не
  должна, так как является, по сути, вокальной.

  Но, не смотря на этот недочет, впечатления от посещенного концерта остались
  только положительные.

\end{document}
