\documentclass[12pt,pscyr]{hedwork}
\usepackage[russian]{babel}

\pagestyle{empty}

\begin{document}

  \begin{flushright}
    Выполнил студент группы Ф-469 Чечеткин И. А.
  \end{flushright}
  \vspace{-2em}
  \begin{center}
    \bfseries <<Классика встречает Рок>>
  \end{center}

  Вчера я впервые посетил Волгоградский ЦКЗ, придя на концерт под
  многообещающим названием <<Классика встречает Рок>>.

  Поскольку подобные мероприятия я посещаю первый раз, то меня поразило большое
  количество музыкальных инструментов, собранных в одном месте, исполняющих
  единое произведение. Однако, из названия я ожидал, что помимо исполнения
  рок-музыки симфоническим оркестром, будет часть, в которой классические
  мелодии исполняются в рок-обработке, но, увы, такой части не было. Хотя, может
  быть, оно и к лучшему.

  Концерт состоял из двух частей. В первой из них исполнялись композиции великих
  русских классиков~-- Римского-Корсакова, Рахманинова и Чайковского. Во второй
  была исполнена сюита Джеза~Коумена <<Симфонический Led Zeppelin>>. В общем,
  впечатления остались положительными. Была пара моментов, которые не совсем
  понравились:
  \begin{itemize}
    \item слишком долгие вступительные комментарии к композициям. Конечно,
      интересно узнать что-то новое об истории произведения, но, на мой взгляд,
      если и делать это вступлением перед игрой, то можно было сделать покороче
      и полаконичнее;
    \item во второй части, во время исполнения <<Kashmir>>, практически не было
      слышно партии вокала, которую играл солист-скрипач. Мелодия просто
      <<утопала>> в фоновой музыке, что сказалось на восприятии произведения.
  \end{itemize}

  Однако, два этих мелких недочета покрываются полученными эмоциями во время
  прослушивания как классических, так и рок-композиций в новой для меня
  обработке. Пожалуй, за одну только <<Stairway to Heaven>> в такой обработке
  можно было бы слушать дирижера-ведущего хоть полчаса кряду.

  В целом, я остался доволен тем, что сходил, и считаю, что точно не потратил
  время зря.

\end{document}
