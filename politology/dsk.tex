\documentclass[pscyr]{hedreport}
\usepackage[utf8]{inputenc}
\usepackage[russian]{babel}

\usepackage{setspace}

\begin{document}
  \onehalfspacing
  \begin{center}
    <<Какая из перечисленных идеологий: либерализм, консерватизм,
      социал-демократия,~-- наиболее подходит современной России?>>
  \end{center}  
  \vspace{-2em}
  \begin{flushright}
    Чечеткин И. А., Ф-469
  \end{flushright}

  За последнее десятилетие ситуация в России значительно изменилась. Но совсем
  не факт, что в сторону улучшения общественной жизни. История показывает, что
  Россия неоднократно переживала различные кризисы, реформы, революции и войны.
  Однако, так как я могу оценивать изменение ситуации в стране лишь за последние
  20--30 лет, то, на мой взгляд, положение отдельного гражданина страны не такое
  <<радужное>>, как может показаться: маленькие пенсии и зарплаты, растущие
  налоги и цены на товары и услуги, малый прожиточный минимум, проблемы с
  различными муниципальными услугами.

  Я считаю, что нашей стране наиболее полно подойдет идеология консерватизма; на
  мой взгляд, именно эта идеология будет способствовать стабильному
  функционированию страны без особого перекроя устоев общества.

  В современной России под все б\'{о}льшим влиянием современного понятия
  <<толерантности>> ко всему создаются условия для разрушения моральных устоев и
  института семьи. Основной из идей консерватизма является уважение традиций,
  мудрости предков, ведь именно традициях и строятся семейные отношения, крепкие
  ячейки общества. Немаловажная роль семье отдается и по той причине, что
  государство воспринимается как единый организм, а разрушение отдельных
  составляющих организма ведет к нарушению работы всего организма, к его
  болезни. В текущий момент институт семьи пытаются сохранить, но, судя по
  западным либерально-демократическим странам, в скором будущем он будет
  разрушен.

  Другим немаловажным вопросом является место человека в политической жизни
  страны. В настоящее время граждане нашей страны не принимают активного участия
  в политической жизни государства, как и должно быть при ведении консервативной
  политики. В идеологии консерватизма людям вообще не надо думать о политике, а
  заниматься более полезными вещами~-- еще один принцип консерватизма,
  прагматизм.

  Следующей идеей идеологии консерватизма является негативное отношение к
  революциям и радикальным изменениям в общественной жизни. Принятие радикальных
  мер не всегда дает положительный результат. Из истории можно привести пример
  распада СССР. В настоящее время в стране наблюдаются нехватка рабочих мест,
  растущие цены даже на товары повседневного пользования при мизерных или
  отсутствующих повышениях заработных плат, рынок услуг заполнен либо
  монополистами, что дает им возможность выставлять какие угодно цены за их
  услуги, либо компаниями-однодневками, у которых нет возможности предоставлять
  услуги приемлемого качества; отсутствие <<доступного>> жилья; растущие налоги,
  ставки на кредиты и ипотеки; дорогостоящее <<бесплатное>> образование;
  введение законов, противоречащих Конституции РФ; низкое качество услуг в
  различных сферах (медицина, образование и т.~д.), связанное с плохой
  подготовкой кадров; повсеместная коррупция и т.~д., и т.~п. Естественно,
  страна стала более открытой, но этого можно было добиться простыми реформами,
  которые не изменят строй государства в корне.

  Консерватизм строится на ценностях, традициях, сформированных поколениями, и
  моральных устоях общества. Одной из таких ценностей, приходящих в помощь
  правителю государства, является религия. В современной России церковь также
  приближена к правительству; законодательные органы принимают различные законы,
  связанные с духовной жизнью населения и жизнью духовенства в частности. Но на
  мой взгляд, если развивать политику современной страны, то она не должна
  опираться на религиозные догматы, а на то, как жили предыдущие поколения,
  внося в их жизнь поправки, связанные с техническим прогрессом.

  На мой взгляд, оставшиеся две политические идеологии~-- либерализм и
  социал-демократия~-- присущи России в меньшей степени.

  Идеология либерализма предполагает полное невмешательство государства в
  частную и экономическую жизнь общества. В России же и внутренний, и внешний
  рынки регулируются государством. Помимо того, в либерализме все равны перед
  законом и каждый реализует свои возможности как хочет сам; а в нашей стране
  ярко проявляются привилегии чиновников: за административное правонарушение
  против должностного лица нарушитель получит особо строгое наказание (вплоть
  до наказания за уголовное преступление), а за уголовное преступление чиновника
  накажут как за административное.

  Социал-демократия предполагает зависимость аппарата власти от населения,
  социальную солидарность и утверждение ценностей свободы, справедливости,
  солидарности и равенства. В России аппарат власти меняется тогда, когда это
  необходимо самой власти: на смену одним чиновникам приходят новые; из-за
  огромного количества частных предприятий социальная солидарность
  неосуществима; а свобода, справедливость и равенство либо жестко ограничены
  аппаратом власти, как в идеологии консерватизма, либо контролируются
  специальными службами.
  
  Таким образом, я считаю, что современной России более всего подходит идеология
  консерватизма.

\end{document}
