\documentclass[pscyr,notitlepage]{hedwork}
\usepackage[utf8]{inputenc}
\usepackage[russian]{babel}

\usepackage{setspace}

\begin{document}
  \onehalfspacing
  \begin{center}
    <<Какая из перечисленных идеологий: либерализм, консерватизм,
      социал-демократия,~-- наиболее подходит современной России?>>
  \end{center}  
  \vspace{-2em}
  \begin{flushright}
    Слоква В. И., Ф-469
  \end{flushright}

  В настоящее время, на мой взгляд, современная Россия должна придерживаться
  идеологии консерватизма. В таких странах как США, Великобритания, Франция,
  ФРГ довольно длительное время консервативные партии находились у власти.
  Глядя на опыт этих стран и политическую атмосферу настоящего времени можно
  сказать, что данный опыт оказал положительное влияние на эти страны.

  Идеология консерватизма предполагает социальную стабильность, так как новые
  социальные порядки могут оказаться хуже старых. В истории России, на мой
  взгляд, развал СССР является неудачным опытом для страны. Идеология
  консерватизма противостоит радикальным изменениям и я считаю в данное время
  нашей стране необходима стабильность в экономике, политике и социальных сферах
  жизни граждан.

  Идеология консерватизма уделяет большую роль традициям и обычаям в жизни
  общества, уважение мудрости предков. В наше время часто встречаются неполные
  семьи, можно заметить деградацию моральных устоев и ценностей в обществе. Я
  считаю, что приход к власти консерваторов со временем способно изменить это.

  С точки зрения консерватизма, государство должно работать как единый слаженный
  механизм. На мой взгляд, государство должно существовать лишь как единое
  целое, только тогда страна достигнет процветания. Четко поставленное
  сотрудничество городов, слаженная работа на предприятиях, человек должен
  осознавать зависимость от существующих отношений. Только так такой сложный
  механизм как государство сможет функционировать минуя кризисы, четко и
  стабильно (стабильность~-- залог успеха).

  Не каждый человек может быть лидером, не каждый может встать у власти, поэтому
  во все времена люди условно делились на ведущих и ведомых. В современной
  России многие граждане равнодушно относятся к политической жизни страны.
  Каждому человеку в консервативном обществе отведена своя роль, в связи с чем
  простым обывателям не нужно думать о политической деятельности правящих
  партий.

  <<Свобода~-- как тот ремень от штанов, когда ремень большой~-- штаны
  падают\ldots>>. Политическая свобода, с точки зрения консерватизма
  отождествляется с ограничением государственной властью. Я считаю, что наличие
  большой свободы либо приводит к произволу и хаосу, либо никак не сказывается
  на части населения в связи с политической неграмотностью.

  Мне кажется, что консерватизм~-- это то, чего не хватает современной России.

  Уход от традиционного образования привел к поколению неквалифицированных
  работников социальной сферы: врачей, учителей, воспитателей, военных и т.~п.
  В наше время с распространением сферы влияния СМИ распространяется <<реклама>>
  нездорового образа жизни в фильмах, сериалах, мультфильмах, рекламе.
  Подобная пропаганда разрушает моральные устои и ценности общества, которые
  консерватизм призван сохранить.

  На мой взгляд, либерализм слабо проявляется в политике современной России и
  не совсем подходит для нее.

  Экономическая ситуация в стране на данный момент не является особо
  стабильной: рост налогов, нестабильная ценовая политика на рынке товаров
  общего потребления и т.~п. Либерализм предполагает невмешательство государства
  в регуляцию внутреннего рынка страны. Я же считаю, что правительство должно
  принимать активное участие в экономической жизни страны во избежание ее резких
  изменений, кризисов.

  В государстве, ведущем либеральную политику, предполагается наличие прав,
  свобод и равенства граждан в их выборе жизненного пути и перед законом. В
  современной России неформально проявляется классовое разделение, люди условно
  делятся на бедных и богатых, на рабочих и чиновников. Также не наблюдается и
  равенства перед законом, что проявляется в использовании привилегий
  чиновниками в аппаратах судебной власти и регулирующих и управляющих органов.
  Равенство возможностей проявляется в огромном числе создания частных компаний,
  но они либо быстро закрываются, либо приносят очень малый доход, либо
  поглощаются более большими и устойчивыми на рынке компаниями.

  Социал-демократия также, на мой взгляд, не подходит современной России.

  В настоящее время можно наблюдать классовую борьбу, что не соответствует
  принципу социальной солидарности идеологии социал-демократии, что также
  выражено в создании и скоропостижном закрытии частных фирм. Помимо этого,
  в нашей стране состав политического аппарата зависит от общества в малой
  степени, что также недопустимо для идеологии социал-демократии.

  Таким образом, по-моему мнению, современной России наиболее подходит идеология
  консерватизма.

\end{document}
