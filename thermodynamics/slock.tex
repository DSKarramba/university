\documentclass[pscyr]{hedwork}
\usepackage[russian]{babel}
\usepackage[derivative,root,shortcuts,environments]{hedmaths}

\faculty{Факультет электроники и вычислительной техники}
\department{<<Физика>>}
\subject{Термодинамика и статистическая физика}
\variant{16}
\student[f]{студентка группы Ф-469 \\ Слоква В. И.}
\teacher[m]{профессор, д. физ.-мат. н. \\ Крючков С. В.}

\renewcommand{\labelenumi}{\asbuk{enumi})}

\begin{document}

  \maketitle % 10, 86, 127, 206; 329, 404, 452, 474

  \begin{task}{10 (ТД)}{
    Получите выражение критических параметров \( V_\text{к} \),
    \( p_\text{к} \), \( T_\text{к} \) через константы уравнения состояния,
    предложенного Бертло для описания поведения реальных газов:
    \[
      \left(p + \frac{a}{TV^2}\right)\cdot\Bigl(V - b\Big) = RT.
    \]
  }

  \end{task}

  \begin{task}{86 (ТД)}{
    Покажите, что сжатие газа по политропе, идущей на диаграмме \( p \) и
    \( V \) круче адиабаты, сопровождается поглощением тепла.
  }
    
  \end{task}
  
  \begin{task}{127 (ТД)}{
    Определите к.~п.~д. цикла Карно, рабочим веществом в котором является газ
    Ван-дер-Ваальса, и покажите, что он равен к.~п.~д. цикла Карно с идеальным
    газом.
  }
  
  \end{task}
  
  \begin{task}{206 (ТД)}{
    Вычислите разность молярных теплоемкостей \( C_p - C_V \) газа, состояние
    которого описывается уравнением Бертло, оставляя лишь линейные члены по
    отношению к \( a \) и \( b \).
  }
  
  \end{task}
  
  \begin{task}{329 (СФ)}{
    Найти положение \( E_\text{вер} \), ширину \( \D E \), отношение
    \( \D E / E_\text{вер} \) и высоту \( w_{\max} \) максимума плотности
    вероятности \( w(E) \) канонического распределения Гиббса для системы с
    большим числом невзаимодействующих частиц \( N \).\\    
    \emph{Указание.} Воспользоваться выражением
    \[
      w(E) = Be^{-\frac{E}{k_0T}}E^{\frac{3N}{2} - 1}
    \]
    для плотности вероятности. Для получения окончательного результата применить
    формулу Стирлинга: \( N! \approx \bigl(N / e\big)^N \).
  }
  
  \end{task}
  
  \begin{task}{404 (СФ)}{
    Вычислить поправку к уравнению состояния для разреженного газа, частицы
    которого взаимодействуют по закону
    \[
      U(r) = \left\{
        \begin{array}{r @{\text{ при }} l}
          \infty        & 0 \le r < d, \\
          -\alpha / r^n & r \ge d,
        \end{array}
      \right.
    \]
    где \( d \)~-- диаметр частицы, \( \alpha > 0 \), \( n > 3 \).
  }
  
  \end{task}
  
  \begin{task}{452 (СФ)}{
    Для ультрарелятивистского электронного газа найдите:
    \vspace{-.5ex}
    \begin{enumerate}
      \itemsep -.5ex
      \item полную и среднюю энергию одной частицы при \( T = 0 \)~К;
      \item связь между давлением и полной энергией.
    \end{enumerate}
    \vspace{-2.5ex}
  }
  
  \end{task}
  
  \begin{task*}{474 (СФ)}{
    Найти зависимость числа фотонов равновесного излучения от полной энергии и
    объема.
  }
  
  \end{task*}

\end{document}
