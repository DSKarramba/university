\documentclass[pscyr]{hedwork}
\usepackage[russian]{babel}
\usepackage[derivative,root,shortcuts,environments]{hedmaths}

\faculty{Факультет электроники и вычислительной техники}
\department{<<Физика>>}
\subject{Термодинамика и статистическая физика}
\variant{18}
\student[m]{студент группы Ф-469 \\ Чечеткин И. А.}
\teacher[m]{профессор, д. физ.-мат. н. \\ Крючков С. В.}

\renewcommand{\labelenumi}{\asbuk{enumi})}

\makeatletter
  \@addtoreset{equation}{task}
  \renewcommand{\theequation}{\thetask.\arabic{equation}}
\makeatother

\begin{document}

  \maketitle

  \begin{task}{8 (ТД)}{
    Вычислите критические параметры \( V_\text{к} \), \( p_\text{к} \),
    \( T_\text{к} \) газа Ван-дер-Ваальса, выражая их через постоянные
    \( a \) и \( b \) для этого газа.
  }
  
  Уравнение Ван-дер-Ваальса:
  \[
    \left(P + \frac{a}{V^2}\right)\Bigl(V - b\Big) = RT.
  \]
  
  Домножим на \( V^2 \) и раскроем скобки:
  \begin{equation}
    PV^3 - (RT + Pb)V^2 + aV - ab = 0.
    \label{eq:1.1}
  \end{equation}

  По определению критических параметров, это уравнение можно записать в виде:
  \begin{gather}
    P_\text{к} (V - V_\text{к})^3 = 0, \text{ раскрывая куб, получим:}
      \nonumber \\
    P_\text{к} (V^3 - 3V^2 V_\text{к} + 3V V_\text{к}^2 - V_\text{к}^3),
      \text{ или,}\nonumber \\
    P_\text{к} V^3 - 3P_\text{к} V^2 V_\text{к} + 3P_\text{к} V V_\text{к}^2 -
      P_\text{к} V_\text{к}^3 = 0. \label{eq:1.2}
  \end{gather}

  Сравнивая \eqref{eq:1.2} с \eqref{eq:1.1}

  \end{task}

  \begin{task}{88 (ТД)}{
    Воздух сжимается по политропе, описываемой уравнением
    \( pV^{1,45} = \const \). Как при этом будет изменяться его температура?
  }
    
  \end{task}
  
  \begin{task}{130 (ТД)}{
    Найдите выражение для к.~п.~д. карбюраторного четырехтактного двигателя
    внутреннего сгорания, работающего по циклу Отто, состоящему из двух
    адиабатических и двух изохорических процессов. Параметром цикла является
    величина \( \eps = V_1 / V_2 \)~-- степень сжатия горючей смеси, которую
    можно считать идеальным газом.
  }
  
  \end{task}
  
  \begin{task}{246 (ТД)}{
    Докажите, что для однородной изотропной системы теплоемкость при постоянном
    давлении равна
    \[
      C_p = T\left[\left(\ppder{I}{S}\right)_{\!\!p}\right]^{-1} \!\!,
    \]
    где \( I \)~-- энтальпия.
  }
  
  \end{task}
  
  \begin{task}{321 (СФ)}{
    Для линейного гармонического осциллятора с энергией \( \eps \) вычислить
    фазовый объем \( \Gamma \), ограниченный гиперповерхностью энергии. Оценить
    объем элементарной фазовой ячейки, используя формулу энергетического спектра
    \[
      \eps_n = h\nu \left(n + \frac{1}{2}\right),
    \]
    где \( n = 0, 1, 2, \ldots \).
  }
  
  \end{task}
  
  \begin{task}{410 (СФ)}{
    Пользуясь теоремами о равномерном распределении кинетической энергии по
    степеням свободы и о вириале в виде
    \[
      \overline{q_i\pder{H}{q_i}} = \overline{p_i\pder{H}{p_i}},
    \]
    вычислить среднюю энергию линейного гармонического осциллятора.
  }
    
  \end{task}
  
  \begin{task}{456 (СФ)}{
    Химический потенциал \( \eta \) бозе-газа определяется равенством
    \[
      \frac{N}{V} = 2\pi(2s + 1)(2mk_0 T)^{3/2} h^{-3}
        \int\lni \frac{\sqrt{z}\,dz}{e^{z - \eta / (k_0 T)} - 1},
    \]
    где \( s \)~-- спин частицы и \( z = \eps / (k_0 T) \). Определить
    температуру бозе-конденсации.
  }
  
  \end{task}
  
  \begin{task*}{450 (СФ)}{
    Оценить удельную электронную теплоемкость (на единицу массы) для лития и
    натрия, предполагая, что валентные электроны в обоих случаях можно
    рассматривать как свободные. Плотности лития и натрия равны соответственно
    \( 0,\!534 \) и \( 0,\!97 \) г/см\( ^3 \).
  }
  
  \end{task*}

\end{document}
