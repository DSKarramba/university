\documentclass[12pt,pdf]{beamer}
\usepackage[utf8]{inputenc}
\usepackage[russian]{babel}
\usepackage{pscyr}

\usepackage{hedmaths}
\usepackage[quantum]{hedphysics}

\usepackage{graphicx}
\graphicspath{{images//}}

\setbeamertemplate{caption}[numbered]
\usetheme[minimal, numbers, nonav, nologo]{Statmod}
\setbeamerfont{footline}{series=\tiny\bfseries}

\usepackage{tikz}
\renewcommand{\~}[1]{\widetilde{#1}}

\begin{document}
  \begin{frame}
    \begin{center}
    \vspace{3.0cm}
    \normalsize
    \textbf{Квантовая информатика.} \\
    \vspace{1.5cm}
    \raggedleft\small\textbf{Выполнил:}\\Чечеткин~И.~А.\\САПР-1.1п\\
    \vspace{1.8cm}
    \vspace{\fill}
    \centering Волгоград \the\year
    \end{center}
  \end{frame}

  \begin{frame}
    \frametitle{Цели и задачи проекта}
    Цель проекта -- ознакомить пользователя с воздействием некоторых квантовых
    логических вентилей на кубит.
    \vspace{1ex}
    
    Задачи проекта:
    \begin{itemize}
      \item ознакомить пользователя с элементарными квантовыми носителями
        информации;
      \item ознакомить пользователя с принципом работы квантовых вентилей;
      \item визуально показать воздействие гейтов на кубит, то есть показать
        изменение коэффициентов \( \alpha \) и \( \beta \), а так же положения
        вектора кубита на сфере Блоха.
    \end{itemize}
  \end{frame}

  \begin{frame}
    \frametitle{Описание объекта автоматизации}
    Объектом автоматизации является кубит.
    
    Его состояние задается в виде волновой функции вида
    \[
    	\ket{\psi} = \alpha\ket{0} + \beta\ket{1}.
    \]
    
    В квантовой информатике существуют квантовые логические элементы,
    изменяющие состояние кубита:
    \[
      \ket{\~{\psi}} = \hat{U}\ket{\psi} = \alpha\hat{U}\ket{0} +
        \beta\hat{U}\ket{1}.
    \]
    
    Кубит можно изобразить в виде единичного вектора на сфере Блоха, углы
    сферической системы координат связаны с параметрами кубита следующими
    соотношениями:
    \[
      \alpha = \cos\frac{\theta}{2}\,; \quad
      \beta = e^{i\phi}\sin\frac{\theta}{2}.
    \]
  \end{frame}

  \begin{frame}
    \frametitle{Функциональная структура АС}
    В составе АС выделяются следующие функциональные подсистемы:
    \begin{itemize}
      \item подсистема сбора данных;
      \item подсистема обработки данных;
      \item подсистема визуализации.
    \end{itemize}
    
    Подсистема сбора данных выполняет процесс сбора данных с формы.
    
    Подсистема обработки данных выполняет обработку данных, полученных с
    подсистемы сбора данных, и формирует выходные данные для подсистемы
    визуализации.
    
    Подсистема визуализации предназначена для визуализации выходных данных.
  \end{frame}
  
  \begin{frame}
    \frametitle{Описание входных и выходных данных АС}
    Входными данными являются начальные параметры кубита \( \alpha_0 \) и
    \( \beta_0 \) и действия пользователя.
    \vspace{1ex}
    
    Выходными данными являются параметры кубита \( \alpha \) и \( \beta \),
    а так же положение кубита на сфере Блоха, т.е. углы \( \phi \) и
    \( \theta \) сферической системы координат.
  \end{frame}
  
  \begin{frame}
    \frametitle{Логическая структура АС}
    \small
    АС выполнена в виде web-приложения, написанного с использованием
    html5 и языка javascript.
    
    Кубиты представлены в виде объектов с четырьмя параметрами: углы \( \phi \)
    и \( \theta \) и коэффициенты \( \alpha \) и \( \beta \).
    
    При выборе какого-либо гейта вектор кубита умножается слева на матрицу
    квантового гейта, производя квантовое вычисление. Измененные коэффициенты
    \( \alpha \) и \( \beta \) передаются в подсистему визуализации.
    
    После этого, производится перерасчет углов \( \phi \) и \( \theta \),
    значения которых так же передаются в подсистему визуализации.
    
    Подсистема визуализации переводит углы сферической системы координат в
    координаты вектора в трехмерном декартовом пространстве, после чего
    производит ортогональную проекцию трехмерного вектора на двумерную
    плоскость экрана; после чего строит проекции сферы и вектора на canvas'е.
  \end{frame}

  \begin{frame}
    \frametitle{Архитектура АС}
    АС содержит:
    \begin{itemize}
      \item модуль ввода, исполненный в виде списка гейтов и формы для задания
        начального значения коэффициентов кубита;
      \item модуль вывода, исполненный в виде html-элемента \emph{canvas};
      \item обработчик данных, исполненный в виде функции \emph{gate};
      \item преобразователь данных, исполненный в виде функции \emph{calc}.
    \end{itemize}
  \end{frame}

  \begin{frame}
    \frametitle{Выводы}
    Данная АС дает возможность ознакомить пользователя с понятием кубита,
    с различными квантовыми гейтами, а так же наглядно показать воздействие
    этих гейтов на кубит.
  \end{frame}
\end{document}
