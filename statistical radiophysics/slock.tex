\documentclass[pscyr]{hedwork}
\usepackage[russian]{babel}
\usepackage[derivative,root,shortcuts,environments]{hedmaths}

\faculty{Факультет электроники и вычислительной техники}
\department{<<Физика>>}
\subject{Статистическая радиофизика}
\student[f]{студентка группы Ф-469 \\ Слоква В. И.}
\teacher[m]{доцент, к. физ.-мат. наук \\ Поляков И. В.}
\variant{17}

\begin{document}

  \maketitle

  \begin{task}{4}{
    Определите непосредственным интегрированием
    \( m_k(\xi) = \int\lii x^k w(x)\,dx \) начальные моменты непрерывной
    случайной величины, имеющей гамма-распределение:
    \[
      w(x) = \frac{\lambda^n}{\Gamma(n)}\,r^{n - 1} e^{-\lambda r},
    \]
    при \( r > 0 \), \( \lambda > 0 \), \( n > 0 \).
  }

    Математическое ожидание:
  \end{task}

  \begin{task}{14}{
    Случайная величина \( x \) распределена с плотностью вероятности
    \[
      w(x) = \frac{x}{\sigma^2}\exp\left(-\frac{x^2}{2\sigma^2}\right),
        \quad x \ge 0
    \].
    Найдите распределение величины \( y = x^2 \). Определите непосредственным
    интегрированием начальные моменты \( m_k(\eta) \).
  }
    
  \end{task}
  
  \begin{task}{37}{
    Корреляционная функция случайного процесса \( x(t) \) имеет вид
    \( B_x(\tau) = D_x e^{-\alpha \tau^2} \). Найти корреляционную функцию
    случайного процесса \( \ds y(t) = x(t) + \der{x(t)}{t} \).
  }
  
  \end{task}
  
  \begin{task*}{47}{
    На вход синхронного детектора подается сумма амплитудно модулированного
    колебания \( s_1(t) = A_0 \bigl(1 + m\cos\Omega t\big) \cos\omega_0 t \)
    и узкополосного стационарного шума \( \xi(t) \) с дисперсией
    \( \sigma_\xi^2 \), а на другой~-- опорное колебание
    \( s_2(t) = A_1\cos\omega_0 t \). Определить отношение сигнал/шум на выходе
    детектора (полезным сигналом считать колебание с частотой модуляции
    \( \Omega \)).
  }
  
  \end{task*}

\end{document}
