\documentclass[pscyr]{hedwork}
\usepackage[russian]{babel}
\usepackage[derivative,root,shortcuts,environments]{hedmaths}

\faculty{Факультет электроники и вычислительной техники}
\department{<<Физика>>}
\subject{Статистическая радиофизика}
\student[m]{студент группы Ф-469 \\ Чечеткин И. А.}
\teacher[m]{доцент, к. физ.-мат. наук \\ Поляков И. В.}
\variant{19}

\begin{document}

  \maketitle

  \begin{task}{6}{
    Определите непосредственным интегрированием
    \( m_k(\xi) = \int\lii x^k w(x)\,dx \) начальные моменты непрерывной
    случайной величины, имеющей закон распределения Накагами
    (\( m \)-распределения):
    \[
      w(x) = \frac{2m^m r^{2m - 1}}{m_2^m\Gamma(m)}\,e^{-\frac{m}{m_2}r^2},
    \]
    при \( r \ge 0 \), \( m \ge 1 / 2 \).
  }

    Математическое ожидание:
  \end{task}

  \begin{task}{16}{
    Найдите плотность вероятности и начальные моменты случайной величины
    \( \eta = e^\xi \), где \( \xi \)~-- случайная гауссовская величина с
    математическим ожиданием \( a \) и дисперсией \( \sigma^2 \).
  }
    
  \end{task}
  
  \begin{task}{39}{
    Вычислить корреляционную функцию \( B_x(t_1, t_2) \) и дисперсию
    \( D_x(t) \) случайного процесса Винера
    \[
      x(t) = \int\limits_0^t n(\theta)\,d\theta,
    \]
    где \( n(t) \)~-- стационарный белый шум с нулевым математическим ожиданием
    и корреляционной функцией \( B_x(\tau) = D\d(\tau) \).
  }
  
  \end{task}
  
  \begin{task*}{49}{
    Для равномерного прямоугольного распределения на отрезке
    \( \bigl[-\lambda / 2, \lambda / 2 \big] \), \( \lambda > 0 \), найти
    характеристическую функцию, а с ее помощью получить общее выражения для
    моментов.
  }
  
  \end{task*}

\end{document}
