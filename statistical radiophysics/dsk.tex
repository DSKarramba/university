\documentclass[pscyr]{hedwork}
\usepackage[russian]{babel}
\usepackage[derivative,root,shortcuts,environments]{hedmaths}

\faculty{Факультет электроники и вычислительной техники}
\department{<<Физика>>}
\subject{Статистическая радиофизика}
\student[m]{студент группы Ф-469 \\ Чечеткин И. А.}
\teacher[m]{доцент, к. физ.-мат. наук \\ Поляков И. В.}
\variant{19}

\begin{document}

  \maketitle

  \begin{task}{6}{
    Определите непосредственным интегрированием
    \( m_k(\xi) = \int\lii x^k w(x)\,dx \) начальные моменты непрерывной
    случайной величины, имеющей закон распределения Накагами
    (\( m \)-распределения):
    \[
      w(x) = \frac{2m^m x^{2m - 1}}{\mu^m\Gamma(m)}\,e^{-\frac{m}{\mu}x^2},
    \]
    при \( x \ge 0 \), \( m \ge 1 / 2 \).
  }
  
    Математическое ожидание:
    \[
      M = m_1 = \int\lni \frac{2m^m x^{2m - 1}}{\mu^m\Gamma(m)}\,
        e^{-\frac{m}{\mu}x^2} x\,dx = \frac{2m^m}{\mu^m\Gamma(m)}
        \int\lni x^{2m}\,e^{-\frac{m}{\mu}x^2}\,dx.
    \]
    Введем замену \( y = mx^2 / \mu \) (\( dy = 2mx / \mu\,dx \)), тогда:
    \[
      M = \frac{m^{-1/2}}{\mu^{-1/2}}\frac{1}{\Gamma(m)}
        \int\lni y^{m - \frac{1}{2}}\,e^{-y}\,dy =
        \sqrt{\frac{\mu}{m}} \frac{\Gamma(m + 1/2)}{\Gamma(m)}.
    \]
    
    Второй момент:
    \[
      m_2 = \int\lni \frac{2m^m x^{2m - 1}}{\mu^m\Gamma(m)}\,
        e^{-\frac{m}{\mu}x^2} x^2\,dx = \frac{2m^m}{\mu^m\Gamma(m)}
        \int\lni x^{2m + 1}\,e^{-\frac{m}{\mu}x^2}\,dx.
    \]
    Введем замену \( y = mx^2 / \mu \) (\( dy = 2mx / \mu\,dx \)), тогда:
    \[
      m_2 = \frac{m^{-1}}{\mu^{-1}}\frac{1}{\Gamma(m)}
        \int\lni y^m\,e^{-y}\,dy =
        \frac{\mu}{m} \frac{\Gamma(m + 1)}{\Gamma(m)} =
        \mu\frac{m\Gamma(m)}{m\Gamma(m)} = \mu.
    \]
    
    Дисперсия:
    \[
      D = \mu_2 = m_2 - m_1^2 = \mu - \frac{\mu}{m}
        \left(\frac{\Gamma(m + 1/2)}{\Gamma(m)}\right)^{\!\!2} =
        \mu\left[1 - \frac{1}{m}
        \left(\frac{\Gamma(m + 1/2)}{\Gamma(m)}\right)^{\!\!2}\right].
    \]
  \end{task}

  \begin{task}{16}{
    Найдите плотность вероятности и начальные моменты случайной величины
    \( \eta = e^\xi \), где \( \xi \)~-- случайная гауссовская величина с
    математическим ожиданием \( a \) и дисперсией \( \sigma^2 \).
  }
  
    Плотность вероятности \( w_\xi(x) \):
    \( \ds
      w_\xi(x) = \frac{1}{\sqrt{2\pi}\sigma}\,e^{-\frac{(x - a)^2}{2\sigma^2}}
    \).
    
    Зависимость \( x(y) \): \( x(y) = \ln y \). Производная от \( y \) по
    \( x \): \( \ds \der{y}{x} = e^x = y \).
    
    Таким образом, плотность вероятности \( w_\eta(y) \):
    \[
      w_\eta(y) = \frac{1}{\sqrt{2\pi}\sigma}\,
        e^{-\frac{(\ln y - a)^2}{2\sigma^2}} \frac{1}{y}.
    \]
    
    Проверим нормировку в пределах от \( y_1 = e^{-\infty} = 0 \) до
    \( y_2 = e^\infty = \infty \):
    \[
      \int\lni \frac{1}{\sqrt{2\pi}\sigma y}
        e^{-\frac{(\ln y - a)^2}{2\sigma^2}}\,dy =
        \int\lii \frac{1}{\sqrt{2\pi}\sigma}
        e^{-\frac{(\ln y - a)^2}{2\sigma^2}}\,d(\ln y - a) =
        \frac{1}{\sqrt{2\pi}\sigma} \sqrt{\pi}\sqrt{2\sigma^2} = 1.
    \]
    
    Математическое ожидание:
    \[
      M_\eta = m_1 = \frac{1}{\sqrt{2\pi\sigma^2}} \int\lni y
        e^{-\frac{(\ln y - a)^2}{2\sigma^2}}\frac{1}{y}\,dy =
        \frac{1}{\sqrt{2\pi\sigma^2}}
        \int\lni e^{-\frac{(\ln y - a)^2}{2\sigma^2}}\,dy.
    \]
    Делая замену \( \ln y = z \) (\( dy = e^z\,dz \)), получим
    \[
      m_1 = \frac{1}{\sqrt{2\pi\sigma^2}}
        \int\lii e^{-\frac{(z - a)^2}{2\sigma^2} + z}\,dz.
    \]
    Приводя к полному квадрату в степени экспоненты, получим окончательно
    \[
      m_1 = \frac{e^{a + \sigma^2 / 2}}{\sqrt{2\pi\sigma^2}}
        \int\lii e^{-\frac{\left[z - \left(a + \sigma^2\right)\right]^2}
        {2\sigma^2}}\,d\left[z - \left(a + \sigma^2\right)\right] =
        e^{a + \sigma^2 / 2}
        \frac{\sqrt{\pi}\sqrt{2\sigma^2}}{\sqrt{2\pi\sigma^2}} =
        e^{a + \sigma^2 / 2}.
    \]
    
    Производя аналогичные действия, получим для \( m_2 \):
    \[
      m_2 = \frac{1}{\sqrt{2\pi\sigma^2}}
        \int\lni y^2 e^{-\frac{(\ln y - a)^2}{2\sigma^2}}\frac{1}{y}\,dy =
        e^{2(\sigma^2 + a)}.
    \]
    
    Тогда дисперсия: \( D_\eta = m_2 - m_1^2 = e^{2(\sigma^2 + a)} -
    e^{2a + \sigma^2} \).
  
  \end{task}
  
  \begin{task}{39}{
    Вычислить корреляционную функцию \( B_x(t_1, t_2) \) и дисперсию
    \( D_x(t) \) случайного процесса Винера
    \[
      x(t) = \int\limits_0^t n(\theta)\,d\theta,
    \]
    где \( n(t) \)~-- стационарный белый шум с нулевым математическим ожиданием
    и корреляционной функцией \( B_n(\tau) = D\d(\tau) \).
  }
  
    
  
  \end{task}
  
  \begin{task*}{49}{
    Для равномерного прямоугольного распределения на отрезке
    \( \bigl[-\lambda / 2, \lambda / 2 \big] \), \( \lambda > 0 \), найти
    характеристическую функцию, а с ее помощью получить общее выражения для
    моментов.
  }
  
    Плотность распределения:
    \[
      w_x = \frac{1}{\left(\frac{\lambda}{2} + \frac{\lambda}{2}\right)} =
        \frac{1}{\lambda}.
    \]
    
    Тогда характеристическая функция равна
    \begin{align*}
      \theta_x & = \average{e^{jux}} = \int\limits_{-\lambda/2}^{\lambda/2}
        \frac{1}{\lambda}\,e^{jux}\,dx = \frac{1}{ju\lambda}
        e^{jux}\bigg|_{-\lambda/2}^{\lambda/2} = \\
      & = \frac{1}{ju\lambda} \Bigl(e^{ju\frac{\lambda}{2}} -
        e^{-ju\frac{\lambda}{2}}\Big) =
        \frac{2}{u\lambda}\sin\left(\frac{u\lambda}{2}\right).
    \end{align*}
    
    Момент \( k \)-порядка:
    \[
      m_k = \frac{1}{j^k} \left(\pkder{\theta_x(u)}{u}{k}\right)\Biggr|_{u=0} =
        \frac{2}{\lambda j^k} \pkder{}{u}{k}\left( \frac{1}{u}
        \sin\left[\frac{u\lambda}{2}\right] \right)\Biggr|_{u=0}.
    \]
  
  \end{task*}

\end{document}
