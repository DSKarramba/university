\documentclass[a4paper, 14pt]{extarticle}
\usepackage[utf8]{inputenc}
\usepackage[paper=a4paper, top=1cm, right=1cm, bottom=1.5cm, left=2cm]{geometry}
\usepackage{setspace}
\usepackage{ifthen}
\usepackage{array}
\usepackage{bm}
\onehalfspacing
\input{../../.preambles/10-russian}
\input{../../.preambles/20-math}

\usepackage{graphicx}
\graphicspath{{plots/}, {images/}}

\parindent=1.25cm

\renewcommand{\thesection}{\arabic{section}}
\renewcommand{\thesubsection}{\arabic{section}.\arabic{subsection}}
\renewcommand{\theequation}{\arabic{section}.\arabic{equation}}

\usepackage{caption}
\DeclareCaptionLabelFormat{figure}{Рисунок #2}
\DeclareCaptionLabelFormat{table}{Таблица #2}
\DeclareCaptionLabelSeparator{sep}{~---~}
\captionsetup{labelsep=sep,justification=centering,font=small}
\captionsetup[figure]{labelformat=figure}
\captionsetup[table]{labelformat=table}

\usepackage{titlesec}
\titleformat{\section}
    {\centering\normalsize\bfseries}
    {\thesection}
    {1em}{}
\titleformat{\subsection}
    {\normalsize\bfseries}
    {\thesubsection}
    {1em}{}

% Настройка вертикальных и горизонтальных отступов
\titlespacing*{\section}{\parindent}{*4}{*4}
\titlespacing*{\subsection}{\parindent}{*4}{*4}

\usepackage[square, numbers, sort&compress]{natbib}
\makeatletter
\bibliographystyle{unsrt}
\renewcommand{\@biblabel}[1]{#1.} 
\makeatother
\addto\captionsrussian{\def\bibname{Список использованных источников}}
\addto\captionsrussian{\def\refname{Список использованных источников}}

\newcolumntype{C}[1]{>{\centering\arraybackslash}m{#1\textwidth}}
\renewcommand{\arraystretch}{1.2}

\usepackage{color}
\definecolor{darkgreen}{rgb}{0,.5,0}
\usepackage[colorlinks,linkcolor=black,filecolor=blue,citecolor=darkgreen,urlcolor=black]{hyperref}

\newcommand{\maketitlepage}[1]{
    \begin{titlepage}
        \singlespacing
        \newpage
        \begin{center}
            Министерство образования и науки Российской Федерации \\
            Федеральное государственное бюджетное образовательное \\
            учреждение высшего профессионального образования \\
            <<Волгоградский государственный технический университет>> \\
            Факультет электроники и вычислительной техники \\
            Кафедра физики
        \end{center}
        \vspace{9em}
        \begin{center}
           { \large\bfseries ОТЧЕТ }
            \\ О научно-исследовательской практике на \( \underset{\text{наименование организации}}{\rule{.35\textwidth}{.5pt}\hrulefill} \)
        \end{center}
        \vspace{4em}
        \begin{table}[h!]
            \center         
            \begin{tabular}{b{.3\textwidth}ccl}
                Руководитель практики от организации & \( \underset{\text{должность}}{\rule{3cm}{.5pt}\hrulefill} \) & \( \underset{\text{подпись}}{\rule{3cm}{.5pt}\hrulefill} \) & Виснер~С.~В. \\
                Руководитель практики от университета & доцент & \( \underset{\text{подпись}}{\rule{3cm}{.5pt}\hrulefill} \) & Поляков~И.~В. \\
                Студент группы Ф-369 & \multicolumn{2}{c}{\( \underset{\text{подпись}}{\rule{6.5cm}{.5pt}\hrulefill} \)} & #1
            \end{tabular}
        \end{table}
        \vspace{5em}

        \begin{flushright}
            \begin{minipage}{.5\textwidth}
                Отчет защищен с оценкой \hrulefill
            \end{minipage}
        \end{flushright}
        \vspace{\fill}
        \begin{center}
            Волгоград, \the\year
        \end{center}

    \end{titlepage}
    \setcounter{page}{2}
}
